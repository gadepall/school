Since   $Q = PAP^T$,
%
\begin{align}
P^TQ^{2015}P &= P^T(PAP^T)^{2015}P
\\
& = \cbrak{(P^TP)A}^{2015}
\\
&=A^{2015}
\end{align}
since $PP^T = I$.
Since,
A=$\begin{pmatrix}
\displaystyle1&\displaystyle1\\
\displaystyle0&\displaystyle1
\end{pmatrix}$,
$A^2=\begin{pmatrix}
\displaystyle1&\displaystyle2\\
\displaystyle0&\displaystyle1
\end{pmatrix}$,
$A^3=\begin{pmatrix}
\displaystyle1&\displaystyle3\\
\displaystyle0&\displaystyle1
\end{pmatrix}$,
$A^4=\begin{pmatrix}
\displaystyle1&\displaystyle4\\
\displaystyle0&\displaystyle1
\end{pmatrix}$,
$$
P^TQ^{2015}P=\begin{pmatrix}
\displaystyle1&\displaystyle2015\\
\displaystyle0&\displaystyle1
\end{pmatrix}
=
A^{2015}=
\begin{pmatrix}
\displaystyle1&\displaystyle2015\\
\displaystyle0&\displaystyle1
\end{pmatrix}
$$
