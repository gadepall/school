\item In a rectangle ABCD, E is the midpoint of AB; F is a point on AC such that BF is perpendicular to AC; and FE perpendicular to BD. Suppose $BC = 8\sqrt{3}$. Find AB?

\item Suppose in the plane 10 pairwise nonparallel lines intersect one another. What is the maximum possible number of polygons (with finite areas) that can be formed?

\item  Let P be an interior point of a triangle ABC whose side lengths are 26, 65, 78. The line through P parallel to BC meets AB in K and AC in L. The line through P parallel to CA meets BC in M and BA in N. The line through P parallel to AB meets CA in S and CB in T. If KL, MN, ST are of equal lengths, find this common length.

\item  Let ABCD be a rectangle and let E and F be points on CD and BC respectively such that area $(ADE) = 16$,  area 
$(CEF) = 9$ and area$(ABF) = 25$. What is the area of triangle AEF? 

\item  Let AB and CD be two parallel chords in a circle with radius 5 such that the centre O lies between these chords. Suppose $AB = 6 , CD = 8$. Suppose further that the area of the part of the circle lying between the chords AB and CD is 
$\frac{m \pi + n}{k}$, where m, n, k are positive integers with gcd(m, n, k) = 1. What is the value of m + n + k? 

\item Consider the areas of the four triangles obtained by drawing the diagonals AC and BD of a trapezium ABCD. The product of these areas, taken two at time, are computed. If among the six products so obtained, two products are 1296 and 576, determine the square root of the maximum possible area of the trapezium to the nearest integer.


