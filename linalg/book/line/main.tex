\begin{enumerate}[label=\arabic*]
\numberwithin{equation}{enumi}

    \item The area enclosed within the curve $\vert x \vert + \vert y \vert = 1 $ is..........
    \item y=$10^x$ is the reflection of y=$\log_10x$ in the line whose equation is...........

    \item The set of lines \begin{align} 
    \myvec{a & b}\vec{x} + c &= 0
    \end{align}, where 3a+2b+4c=0 is a concurrent at 
    the point......
    \item Given the points $A\myvec{0 \\ 4}$ and $B\myvec{0 \\ -4}$, the equation of the locus of the point $P\myvec{x \\ y}$ such that $\vert AP-BP \vert$=6  is......
    \item If a,b and c are in A.P,then the straight line \begin{align} 
    \myvec{a & b}\vec{x} + c &= 0
    \end{align}will always pass through a fixed point whose coordinates are.......... 
    \item The orthocentre of the triangle formed by the lines \begin{align} 
    \myvec{1 & 1}\vec{x}  &= 1
    \end{align}\begin{align} 
    \myvec{2 & 3}\vec{x}  &= 6
    \end{align} and \begin{align} 
    \myvec{2 & 3}\vec{x} + 4 &= 0
    \end{align} lies in quadrant number..........
    \item Let the algebraic sum of the perpendicular distances from the points $\myvec{2 \\ 0}$,  $\myvec{0 \\ 2}$ and $\myvec{1 \\ 1}$ to a variable straight line be zero; then the line passes through a fixed points whose coordinates are .........\\
    \item The vertices of a triangle are A$\myvec{-1 \\ -7}$,B$\myvec{5 \\ 1}$ and c$\myvec{1 \\ 4}$. The equation of the bisector of the $\angle$ ABC is .............\\
\end{enumerate}

{\large \textbf{B True/False}}
\begin{enumerate}[label=\arabic*]
\numberwithin{equation}{enumi}
    \item The Straight line \begin{align} 
    \myvec{5 & 4}\vec{x} &= 0
    \end{align} passes through the point of intersection of the straight lines \begin{align} 
    \myvec{1 & 2}\vec{x} - 10 &= 0
    \end{align} and \begin{align} 
    \myvec{2 & 1}\vec{x} + 5 &= 0
    \end{align}
    \item The lines \begin{align} 
    \myvec{2 & 3}\vec{x} + 19 &= 0
    \end{align} and \begin{align} 
    \myvec{9 & 6}\vec{x} - 17 &= 0
    \end{align} cut the coordinate axes in concyclic points
\end{enumerate}

{\large \textbf{C} \textbf{MCQs with one Correct Answer}}
\begin{enumerate}
    \item The points $\myvec{-a\\b}$,\myvec{0\\0},\myvec{a\\b} and \myvec{a^2\\ab} are:
    \begin{enumerate}
    \item  Collinear
    \item  Vertices of a parallelogram
    \item  Vertices of a rectangle
    \item  None of these
    \end{enumerate}
    \item The point $\myvec{4 \\ 1}$ undergoes the following three transformations successively
    (i) Reflection about the line \begin{align} 
    \myvec{-1 & 1}\vec{x} &= 0
    \end{align}
    (ii)Translation through a distance 2 units along the positive direction of x-axis
    (iii) Rotation through an angle p/4 about the origin in counter clockwise direction.
    Then the final position of the point is given by the coordinates
    \begin{enumerate}
     \item  \myvec{1/\sqrt{2}\\7/\sqrt{2}}
    
     \item  \myvec{\sqrt{-2}\\7/\sqrt{2}}
    
     \item  \myvec{-1/\sqrt{2}\\7/\sqrt{2}}
    
     \item  \myvec{\sqrt{2}\\7/\sqrt{2}}
    \end{enumerate}
\item The straight line \begin{align} 
    \myvec{1 & 1}\vec{x} &= 0
    \end{align} \begin{align} 
    \myvec{3 & 1}\vec{x} - 4 &= 0
    \end{align}\begin{align} 
    \myvec{1 & 3}\vec{x} - 4 &= 0
    \end{align}form a triangle which is
    \begin{enumerate}
     \item isosceles
     \item Equilateral
     \item right angled
     \item none of these
     \end{enumerate}
    \item If P=\myvec{1\\0}, Q=\myvec{-1\\0}and R=\myvec{2\\0} are three given points, then locus of the point S satisfying the relation $SQ^2+SR^2=2SP^2$, is
    \begin{enumerate}
     \item  a straight line parallel to X-axis
     \item a circle passing through the origin
     \item a circle with centre at the origin
     \item  a straight line parallel to Y-axis.
     \end{enumerate}
    \item Line L has intercepts a and b on the coordinate axes.When the axes are rotated through a given angle, keeping the origin fixed, the same line L has intercepts p and q, then
    \begin{enumerate}
     \item  $a^2+b^2=p^2+q^2$
     \item  $1/a^2+1/b^2=1/p^2+1/q^2$
     \item  $a^2+p^2=b^2+q^2$
     \item  $1/a^2+1/p^2=1/b^2+1/q^2$
     \end{enumerate}
    \item If the sum of the distances of a point from two perpendicular lines in a plane is 1, then its locus is
    \begin{enumerate}
     \item  Square
     \item  Circle
     \item  Straight line
     \item  Two intersecting lines
     \end{enumerate}
    \item The locus of a variable point whose distances from\myvec{-2\\0}is 2/3 times its distance from the line x=-9/2 is
    \begin{enumerate}
     \item  Ellipse
     \item  Parabola
     \item  Hyperbola
     \item  None of these
    \end{enumerate}
    \item The equation to a pair of opposite sides of a parallelogram are  \begin{align} 
    \vec{x}^T\myvec{1 & 0 \\ 0 & 0}\vec{x} + \myvec{-5 & 0}\vec{x} + 5 &= 0\end{align} and \begin{align} 
    \vec{x}^T\myvec{0 & 0 \\ 0 & 1}\vec{x} + \myvec{0 & -6}\vec{x} + 5 &= 0\end{align}  the equation to its diagonals are
    \begin{enumerate}
     \item  $\myvec {1 & 4} \vec x =13$ , $\myvec{-4 & 1} \vec x= -7$
     \item   $\myvec {4 & 1} \vec x =13$ , $\myvec{-1 & 4} \vec x= 7$
     \item   $\myvec {4 & 1} \vec x =13$ , $\myvec{-4 & 1} \vec x= -7$
     \item   $\myvec {-4 & 1} \vec x =13$ , $\myvec{4 & 1} \vec x= 13$
     \end{enumerate}
    \item The orthocentre of the triangle formed by the lines \begin{align}
    \vec {x}^T\myvec{0 & 0 \\ 1 & 0} \vec{x} &=0
    \end{align} and $\myvec{1 & 1} \vec x =1$ is 
    \begin{enumerate}
     \item \myvec{1/2\\1/2}
     \item \myvec{1/3\\1/3}
     \item \myvec{0\\0}
     \item \myvec{1/4\\1/4}
     \end{enumerate}
    \item Let PQR be a right angled isosceles triangle, right angled at  P\myvec{2\\1}. If the equation of the line QR is \begin{align}\myvec{2 & 1}\vec{x} = 3\end{align},then the equation representing the pair of lines PQ and PR is
    \begin{enumerate}
     \item  $\vec{x}^T\myvec{3 & 8 \\ 0 & -3}\vec{x} + \myvec{20 & 10}\vec{x} + 25 = 0$
     \item  $\vec{x}^T\myvec{3 & 8 \\ 0 & -3}\vec{x} + \myvec{-20 & -10}\vec{x} + 25 = 0$
     \item  $\vec{x}^T\myvec{3 & 8 \\ 0 & -3}\vec{x} + \myvec{10 & 15}\vec{x} + 20 = 0$
     \item  $\vec{x}^T\myvec{3 & -8 \\ 0 & -3}\vec{x} + \myvec{10 & -15}\vec{x} - 20 = 0$
     \end{enumerate}
    \item If $x_1,x_2,x_3$ as well as $y_1,y_2,y_3$ are in G.P with the same common ratio, then the points\myvec{x_1\\y_1},\myvec{x_2\\y_2} and \myvec{x_3\\y_3}
    \begin{enumerate}
     \item  lie on a straight line
     \item  lie on a ellipse
    \item  lie on a circle
    \item  vertices of a triangle
    \end{enumerate}
    \item Let PS be the median of the triangle with vertices P\myvec{2\\2}, Q\myvec{6\\-1} and R\myvec{7\\3}. The equation of the line passing through\myvec{1\\-1} and Parallel to PS is
    \begin{enumerate}
     \item $\myvec{2 & -9}\vec{x} - 7 = 0$
     \item $\myvec{2 & -9}\vec{x} - 11 = 0$
     \item $\myvec{2 & 9}\vec{x} - 11 = 0$
     \item $\myvec{2 & -9}\vec{x} + 7 = 0$
    \end{enumerate}
    \item The incentre of the triangle with vertices $\myvec{1\\\sqrt{3}},\myvec{0\\0}$ and $\myvec{2\\0}$ is 
    \begin{enumerate}
     \item  $\myvec{1\\\sqrt3/2}$
     \item  $\myvec{2/3\\1/\sqrt3}$
     \item  $\myvec{2/3\\ \sqrt3/2}$
     \item  $\myvec{1\\1\sqrt3}$
     \end{enumerate}
    \item The number of integer values of m, for which the x coordinate of the point of intersection of the lines\myvec{3\\4}$\vec {x}$=9 and \myvec{-m\\1}$\vec {x}$-1=0  is also an integer,is
    \begin{enumerate}
     \item  2
     \item  0
     \item  4
     \item  1
     \end{enumerate}
    \item The area of the parallelogram formed by the lines \myvec{-m\\1}$\vec {x}$=0, \myvec{-m\\1}$\vec {x}$+1=0 ,\myvec{-n\\1}$\vec {x}$=0 and \myvec{-n\\1}$\vec {x}$+ 1 =0 equals
    \begin{enumerate}
     \item $\vert m+n\vert /(m-n)^2$
     \item $2/\vert m+n\vert$
     \item $1/(\vert m+n\vert)$
     \item $1/(\vert m-n\vert)$ 
    \end{enumerate}
    \item Let 0\textless$\alpha<\pi$  be fixed angle. If
    $P=\myvec{cos\theta\\sin\theta}$ and $Q=\myvec{cos(\alpha-\theta)\\sin(\alpha-\theta)}$
    then the Q is obtained from P by
    \begin{enumerate}
     \item  clockwise rotation around origin through an angle$\alpha$
     \item  anticlockwise rotation around origin through an angle$\alpha$
     \item  reflection in the line through origin with slope tan$\alpha$
     \item  reflection in the line through origin with slope tan$\alpha/2$
     \end{enumerate}
    \item Let $P=\myvec{-1\\0} Q=\myvec{0\\0}$ and $R=\myvec{3\\\sqrt3}$ be three points.Then the equation of the bisector of the angle PQR is
    \begin{enumerate}
     \item $\myvec{{\frac{\sqrt3}{2}} & 1} \vec x= 0$
     \item $\myvec{1 & \sqrt3} \vec x= 0$
     \item $\myvec{\sqrt3 & \sqrt1} \vec x= 0$
     \item $\myvec{1 & \frac{\sqrt3}{2}} \vec x= 0$
     \end{enumerate}
    \item A straight line through the origin O meets the parallel lines \begin{align} \myvec{4 & 2} \vec {x} &= 9\end{align} and \begin{align} \myvec{2 & 1} \vec {x} + 6 &= 0\end{align} ar points P and Q respectively. Then the point O divides the segment PQ in the ratio
    \begin{enumerate}
     \item 1:2
     \item 3:4
     \item  2:1
     \item 4:3
     \end{enumerate}
    \item The number of integral points (integral points means both the coordinate should be integer)exactly in the interior of the triangle with the vertices \myvec{0\\0} , \myvec{0\\21} and  \myvec{21\\0} is 
    \begin{enumerate}
     \item 133
     \item 190
     \item 233
     \item 105
     \end{enumerate}
    \item Orthocentre of a triangle with vertices  \myvec{0\\0}, \myvec{0\\0},\myvec{3\\4},\myvec{4\\0} is
    \begin{enumerate}
     \item \myvec{3\\\frac{5}{4}}
     \item \myvec{3\\12}
     \item \myvec{3\\\frac{3}{4}}
     \item \myvec{3\\9}
     \end{enumerate}
    \item Area of the triangle formed by the line \begin{align} \myvec{1 & 1} \vec {x} &= 3\end{align} and angle bisectors of the pair of straight lines \begin{align}\vec{x}^T \myvec{1 & 0 \\ 0 & -1} \vec {x} + \myvec{0 & 2} \vec {x} &= 1\end{align}
    \begin{enumerate}
     \item  2 sq.units
     \item  4 sq.units
     \item  6 sq.units
     \item  8 sq.units
     \end{enumerate}
    \item Let O\myvec{0\\0},P\myvec{3\\4},Q\myvec{6\\0} be the vertices of the triangle OPQ. The point R inside the triangle OPQ is such that the triangles OPR,PQR, OQR are of equal area. The coordinates of R are
    \begin{enumerate}
     \item  \myvec{\frac{4}{3}\\3}
     \item  \myvec{3\\\frac{2}{3}}
     \item  \myvec{3\\\frac{4}{3}}
     \item  \myvec{\frac{4}{3}\\\frac{2}{3}}
     \end{enumerate}
    \item A straight line L through the point \myvec{3\\-2} is inclined at an angle $60^0$ to the line\begin{align} \myvec{3 & 1} \vec {x} &= 1\end{align}. If L also intersects the x-axis, then the equation of L is
    \begin{enumerate}
     \item  $\myvec{\sqrt3 & 1} \vec {x} + 2 - 3\sqrt3=0$
     \item  $\myvec{-\sqrt3 & 1} \vec{x} + 2 + 3\sqrt3=0$
     \item  $\myvec{-1 & \sqrt3 } \vec x + 3 + 2\sqrt3 =0$
     \item  $\myvec{1 & \sqrt3 } \vec x - 3 + 2\sqrt3 =0$
     \end{enumerate}
\end{enumerate}

{\Large \textbf{C MCQs with one or More than one correct}}
\begin{enumerate}
    \item Three lines \begin{align}\myvec{p & q} \vec {x} + r &= 0\end{align}, \begin{align}\myvec{q & r} \vec {x} + p &= 0\end{align} and \begin{align}\myvec{r & p} \vec {x} + q &= 0\end{align} are concurrent if
    \begin{enumerate}
     \item  $p+q+r=0$
     \item  $p^2+q^2+r^2=qr+rp+pq$
     \item  $p^3+q^3+r^3=3pqr$
     \item  none of these
     \end{enumerate}
    \item The points \myvec{0\\\frac{8}{3}},\myvec{1\\3} and \myvec{82\\30} are vertices of 
    \begin{enumerate}
     \item  an obtuse angled triangle
     \item  an acute angled triangle
     \item  a right angled triangle
     \item  none of these
     \end{enumerate}
    \item All points lying inside the triangle formed by the points \myvec{1\\3},\myvec{5\\0} and\myvec{-1\\2}satisfy
    \begin{enumerate}
     \item  $\myvec{3 & 2} \vec {x} \geq 0$
     \item  $\myvec{2 & 1} \vec {x}-13\geq 0$
     \item  $\myvec{2 & -3} \vec {x}-12\leq 0$
     \item  $\myvec{-2 & 1} \vec {x} \geq 0$
     \item none of these
     \end{enumerate}
    \item A vector $\vec a$ has components 2p and 1 with respect to a rectangluar cartesian system.The system is rotated through a certain angle about the origin in the counter clockwise sense. If, with respect to the new system,  $\vec a$ has components p+1 and 1, then
    \begin{enumerate}
     \item  p=0
     \item  p=1 or p=$-\frac{1}{3}$
     \item  p=-1 or p=$\frac{1}{3}$
     \item  p=1 or p=-1
     \item none of these
     \end{enumerate}
    \item If P\myvec{1\\2}, Q\myvec{4\\6}, R\myvec{5\\7} and S\myvec{a\\b} are the vertices of a parallelogram PQRS,then
    \begin{enumerate}
     \item  a=2, b=4
     \item  a=3, b=4
     \item  a=2, b=3
     \item  a=3, b=5
     \end{enumerate}
    \item The diagonals of a parallelogram PQRS are along the lines \begin{align}\myvec{1 & 3} \vec {x} &= 4\end{align} and \begin{align}\myvec{6 & -2} \vec {x} &= 7\end{align}. Then PQRS must be a.
    \begin{enumerate}
     \item  rectangle
     \item  square
     \item  cyclic quadrilateral
     \item  rhombus
     \end{enumerate}
    \item If the vertices P, Q , R of a triangle PQR are rational points, which of the following points of the triangle PQR is(are) always rational point(s)?
    \begin{enumerate}
     \item  centroid
     \item  incentre
     \item  circumcentre
     \item  orthocentre
     \end{enumerate}
    \item Let $L_1$ be a straight line passing through the origin and $L_2$ be the straight line \begin{align}\myvec{1 & 1} \vec {x} &= 1\end{align}.If the intercepts made by the circle  (\begin{align}\vec{x}^T\vec{x} + \myvec{-1 & 3} \vec {x} &= 0\end{align} on $L_1$ and $L_2$ are equal, then which of the following equations can represent $L_1$?
    \begin{enumerate}
     \item  $\myvec{1 & 1} \vec {x} = 0$
     \item  $\myvec{1 & -1} \vec {x} = 0$
     \item  $\myvec{1 & 7} \vec {x} = 0$
     \item  $\myvec{1 & -7} \vec {x} = 0$
     \end{enumerate}
    \item For a\textgreater b\textgreater c\textgreater 0, the distance between(1,1) and the point of intersection of the lines (a , b)$\vec {x}$+c=0 and (b , a)$\vec {x}$+c=0 is less than 2$\sqrt2$. Then 
    \begin{enumerate}
     \item  a+b-c\textgreater 0
     \item  a-b+c\textless 0
     \item  a-b+c\textgreater 0
     \item  a+b-c\textless 0
    \end{enumerate}
\end{enumerate}
{\Large \textbf{E Subjective Problems}}
\begin{enumerate}
    \item  A straight line segment of length l, moves with its ends on two mutually perpendicular lines. Find the locus of the points which divides the line segment in the ration 1:2.
    \item The area of triangle is 5. Two of its vertices are A\myvec{2\\1}, B\myvec{3\\-2}. The third vertex C lies on \myvec{-1\\1}$\vec {x}$ = 3. Find C.
    \item one side of a rectangle lies along the line \begin{align}\myvec{4 & 7} \vec {x} + 5 &= 0\end{align}. Two of its vertices are \myvec-{3\\1} and \myvec{1\\1}. Find the equations of the other three sides.
   \item \item  Two vertices of a triangle are \myvec{5\\-1} and\myvec{2\\-3}. If  the orthocentre of the triangle is the origin, find the coordinates of the third point.
     \item  Find the equation of the line which bisects the obtuse angle between the lines \begin{align}\myvec{1 & -2} \vec {x} + 4 &= 0\end{align} and \begin{align}\myvec{4 & -3} \vec {x} -2 &= 0\end{align}.
    \item A straight line L is perpendicular to the line \begin{align}\myvec{5 & -1} \vec {x} &= 1\end{align}. The area of the triangle formed by the line L and the coordinate axes is 5. Find the equation of the line L.
    \item The end A,B of a straight line segment of constant length c slide upon the fixed rectangular axis OX,OY respectively. If the rectangle OAPB be completed, then the show that the locus of the foot of the perpendicular drawn from P to AB is $x^{2/3}+y^{2/3}=c^{2/3}$.
    \item The vertices of a triangle are [a $t_1t_2$, a($t_1+t_2$)],[a $t_2t_3$, a($t_2+t_3$)],[a $t_3t_1$, a($t_3+t_1$)]. Find the orthocentre of the triangle.
    \item The coordinates of A,B,C are \myvec{6\\3},\myvec{-3\\5},\myvec{4\\-2} respectively, and P is any point \myvec{x\\y}. Show that the ratio of the area of the triangles $\triangle PBC$ and $\triangle ABC$ is $\vert {\frac{(1 , 1)\vec {x}-2}{7}}\vert$
    \item Two equal sides of an isosceles triangles are given by the equations \begin{align}\myvec{7 & -1} \vec {x} + 3 &= 0 \end{align} and \begin{align}\myvec{1 & 1} \vec {x} - 3 &= 0 \end{align} and its third side passes through the point \myvec{1\\10}. Determine the equation of third side.
    \item One of the diameters of the circle circumscribing the rectangle ABCD is \begin{align}\myvec{-1 & 4} \vec {x}  &= 7 \end{align}. If A and B are the points \myvec{-3\\4} and\myvec{5\\4}respectively, then find the area of the rectangle.
    \item Two sides of a rhombus ABCD are parallel to the lines \begin{align}\myvec{-1 & 1} \vec {x}  &= 2 \end{align} and \begin{align}\myvec{-7 & 1} \vec {x}  &= 3 \end{align}. If the diagonals of the rhombus intersects at the point(1,2) under vertex A is on the y axis. Find possible coordinates of A.
    \item Lines \begin{align}L_1\equiv\myvec{a & b} \vec {x} + c &= 0 \end{align} \begin{align}L_2\equiv\myvec{1 & m} \vec {x} + n &= 0 \end{align} intersect at the point P and make an angle $\theta$ with each other. Find the equation of a line L different from $L_2$ which passes through P and makes the same angle $\theta$ with $L_1$
    \item Let ABC be a traingle with AB=AC. If D is the mid point of BC,E is the foot of the perpendicular drawn from D to AC and F the mid-point of DE, Prove that AF perpendicular to BE.
    \item Straight lines\begin{align}\myvec{3 & 4} \vec {x}  &= 5 \end{align} and \begin{align}\myvec{4 & -3} \vec {x}  &= 15 \end{align} intersect at the point A. Points B and C are chosen on these two lines such that AB=AC. Determine the possible equations of the lines BC passing through the point\myvec{1\\2}
    \item A line cuts the x-axis at A\myvec{7\\0} and the y-axis at B\myvec{0\\-5}. A variable line PQ is draw perpendicular to AB cutting the x-axis in P and the y-axis in Q. If AQ and BP intersect at R, find the locus of R.
    \item Find the equation of the line passing through the point\myvec{2\\3} and making intercept of a length 2 units between the lines\myvec{2\\1}$\vec {x}$=3 and\myvec{2\\1}$\vec {x}$=5.
%\includegraphics[scale=1.5]{sample}
    \item Show that all chords of the curve  \begin{align}\vec{x}^T\myvec{3 & 0 \\ 0 & -1} \vec {x} + \myvec{2 & 4} \vec {x} &= 0 \end{align} which subtend a right angle at the origin, Pass through a fixed point. Find the coordinates of the point.
    \item Determine all values of $\alpha$ for which the point $(\alpha, \alpha^2)$ lies inside the triangle formed by the lines 
    \begin{align}\myvec{2 & 3} \vec {x} -1 &= 0 \end{align} \begin{align}\myvec{1 & 2} \vec {x} -3 &= 0 \end{align}  \begin{align}\myvec{5 & 6} \vec {x} -1 &= 0 \end{align}
    \item Tangent at a point $P_1$ [other than \myvec{0\\0}] on the curve 
    \begin{align}\myvec{0 & 1}\vec{x} - (\vec{x}^T \myvec{1 & 0 \\ 0 & 0} \vec {x})\vec {x}  &= 0 \end{align} meets the curve again at $P_2$. The tangent at $P_2$ meets the curve at $P_3$ and so on. Show that the abscissae of $P_1,P_2,P_3........P_n$,form a G.P. Also find the ratio. [area($\triangle(P_1,P_2,P_3)$/area($\triangle(P_2,P_3,P_4)$]
    \item A line through A\myvec{-5\\-4} meets the line\myvec{1\\3}$\vec {x}$+2=0,\myvec{2\\1} $\vec {x}$ +4=0 and\myvec{1\\-1}$\vec {x}$ -5=0 at the points B,C and D respectively. If$(15/AB)^2+(10/AC)^2=(6/AD)^2$,find the equation of the line.
    \item A rectangle PQRS has its side PQ parallel to the line \begin{align}\myvec{-m & 1} \vec {x}  &= 0 \end{align} and vertices P,Q and S on the lines \begin{align}\myvec{0 & 1} \vec {x}  &= a \end{align}, \begin{align}\myvec{1 & 0} \vec {x}  &= b \end{align} and \begin{align}\myvec{1 & 0} \vec {x}  &= -b \end{align} respectively. Find the locus of vertex R.
    \item Using coordinate geometry, prove that the three altitudes of any triangle are concurrent.
    \item For points $P=\myvec{x_1\\y_1}$ and $Q=\myvec{x_2\\y_2}$of the coordinate plane, a new distance d(P,Q) is defined by d\myvec{P\\Q}=$\vert x_1-x_2\vert+ \vert y_1-y_2\vert$. Let O=\myvec{0\\0} and A=\myvec{3\\2}. Prove that the set of points in the first quadrant which are equidistant (with respect to the new distance) from O and A consists of the union of a line segment of finite length and infinite ray. Sketch this set in a labelled diagram.
    \item Let ABC and PQR be any two triangles in the same plane. Assume that the perpendiculars from the points A,B,C to the sides QR,RP,PQ respectively are concurrent. Using vector methods or otherwise, prove that the perpendiculars from P,Q,R to BC,CA,AB respectively are also concurrent.
    \item Let a,b,c be real numbers with $a^2+b^2+c^2=1$. show that the equation
    \begin{align}\myvec{\myvec{a & -b}\vec {x} - c & \myvec{b & a}\vec {x} & \myvec{c & 0}\vec {x} + a \\ \myvec{b & a}\vec {x} &  \myvec{-a & b}\vec {x} - c & \myvec{0 & c}\vec {x} + b \\ \myvec{c & 0}\vec {x} + a & \myvec{0 & c}\vec {x} + b & \myvec{-a & -b}\vec {x} + c} \end{align} represents a straight line.
    \item A straight line L through the origin meets the line and \begin{align}\myvec{1 & 1} \vec {x} &= 3 \end{align} at P and Q respectively. Through P is straight lines $L_1$and $L_2$ are drawn parallel to \begin{align}\myvec{2 & -1} \vec {x} &= 5 \end{align}, \begin{align}\myvec{3 & 1} \vec {x}  &= 5 \end{align} respectively. Lines $L_1$ and $L_2$ intersect at that the locus of R, as L varies, is a straight line.
    \item A straight line L with negative slope passes through Point\myvec{8\\2} and cuts the positive coordinates are P and Q. Find the absolute minimum value of OP varies, where O is origin.
    \item The area of the triangle formed by the intersection of parallel to x-axis and passing through P(h,k) with \begin{align}\myvec{1 & 1} \vec {x} &= 0 \end{align}  and \begin{align}\myvec{1 & 1} \vec {x} &= 2 \end{align} is 4$h^2$.Find the locus of the point.
\end{enumerate}
    {\large\textbf{H: Assertion Reason Type Questions}}

\begin{enumerate}
    \item Lines \begin{align}L_1: \myvec{-1 & 1} \vec {x} &=0\end{align} and \begin{align}L_2: \myvec{2 & 1} \vec {x} &=0\end{align} intersect the line \begin{align}L_1: \myvec{0 & 1} \vec {x} + 2 &=0\end{align} at P and Q respectively. The bisector of the acute angle between $L_1$ and $L_2$ intersects $L_3$ at R.\\
     {\textbf{STATEMENT-1:}}The ratio PR:RQ equals 2$\sqrt2:\sqrt5$ because\\
    
     {\textbf{STATEMENT-2:}}If any triangle, bisector of an angle divides the triangle into two similar triangles
     \item  Statement-1 is true, Statement-2 is true ; Statement-2 is not a correct explanation for Statement-1
     \item  Statement-1 is true, Statement-2 is true ;  Statement-2 is not a correct explanation for Statement-1
     \item  Statement-1 is True,Statement False
     \item  Statement-1 is False,Statement True\\
\end{enumerate}
{\Large \textbf{I Integer Value Correct Type}}
\begin{enumerate}


\item For a point P in the plane, let $d_1(P)$ and $d_2(P)$ be the distance of the point P from the lines \begin{align} \myvec{1 & -1} \vec {x} &=0\end{align} and \begin{align} \myvec{1 & 1} \vec {x} &=0\end{align} respectively. The area of the region R consisting of all points P lying in the first quadrant of the plane and satisfying 2$\leq d_1(P)+d_2(P) \leq 4$, is
\end{enumerate}
\section{\LARGE Section-B}
\begin{enumerate}
    \item A triangle with vertices \myvec{4\\0},\myvec{-1\\-1} and \myvec{3\\5} is
    \begin{enumerate}
     \item  isosceles and right angled
     \item  isosceles but not right angled
     \item  right angled but not isosceles
     \item  neither right angled nor isosceles
     \end{enumerate}
    \item Locus of mid points of the portion between the axis of \begin{align}\myvec{\cos\alpha & \sin\alpha}\vec {x} &= 0\end{align} where p is constant
    \begin{enumerate}
     \item  $\vec{x}^T\vec{x} ={\frac{4}{p^2}}$ 
     \item  $\vec{x}^T\vec{x} =4p^2$ 
     \item  ${\frac{1}{x^2}}+{\frac{1}{y^2}}={\frac{2}{p^2}}$
     \item  ${\frac{1}{x^2}}+{\frac{1}{y^2}}={\frac{4}{p^2}}$
     \end{enumerate}
    \item If the pair of the lines \begin{align} \vec{x}^T\myvec{a & 2h \\ 0 & b} \vec {x} + 2\myvec{g & f}\vec {x} + c &=0\end{align}  intersects the y-axis then
    \begin{enumerate}
     \item  2fgh=b$g^2$+c$h^2$
     \item  b$g^2\neq$c$h^2$
     \item  abc=2fgh
     \item  none of these
     \end{enumerate}
    \item A pair of lines represented by \begin{align} \vec{x}^T\myvec{3a & 5 \\ 0 & (a^2-2)} \vec {x} &=0\end{align} are perpendicular to each other for
    \begin{enumerate}
     \item  two values of a 
     \item  $\forall$ a
     \item  for one value of a 
     \item  for no values of a 
     \end{enumerate}
    \item A square of side of a lies above the x-axis and has one vertex at the origin. The side passing through the origin makes an angle $\alpha(0\textless\alpha\textless\pi$/4) with the positive direction of the x-axis. The equation of its diagonal not passing through the origin is
    \begin{enumerate}
     \item  $\myvec{(\cos\alpha-\sin\alpha) & (\cos\alpha+\sin\alpha)}\vec {x} = a$
     \item  $\myvec{-(\sin\alpha-\cos\alpha) & (\cos\alpha-\sin\alpha)}\vec {x} =a$
     \item  $\myvec{(\sin\alpha-\cos\alpha)& (\cos\alpha+\sin\alpha)}\vec {x}  =a$
     \item  $\myvec{(\cos\alpha+\sin\alpha) & (\cos\alpha+\sin\alpha)} \vec {x}  =a$\\
     \end{enumerate}
    \item If the pair of straight lines \begin{align}\vec{x}^T \myvec{1 & -2p \\ 0 & -1}\vec {x} &= 0\end{align} and \begin{align}\vec{x}^T \myvec{1 & -2q \\ 0 & -1}\vec {x} &= 0\end{align} be such that each pair bisects the angle between the other pair, then
    \begin{enumerate}
     \item  pq=-1
     \item  p=q
     \item  p=-q
     \item  pq=1
     \end{enumerate}
    \item Locus of centroid of the triangle whose vertices are \myvec{a \cos t\\a \sin t} , \myvec{b \sin t\\-b \cos t}and \myvec{1\\0}, where t is the parameter is
    \begin{enumerate}
     \item  $9(\vec{x}^T\vec {x}) + \vec{x}^T \myvec{6 & 0}\vec {x} = 1 - a^2 - b^2$
     \item  $9(\vec{x}^T\vec {x}) + \vec{x}^T \myvec{-6 & 0}\vec {x} = 1 - a^2 - b^2$
     \item  $9(\vec{x}^T\vec {x}) + \vec{x}^T \myvec{-6 & 0}\vec {x} = 1 - a^2 + b^2$
     \item  $9(\vec{x}^T\vec {x}) + \vec{x}^T \myvec{6 & 0}\vec {x} = 1 - a^2 + b^2$
     \end{enumerate}
    \item If $x_1$,$x_2$,$x_3$ and $y_1$,$y_2$,$y_3$ are both in G.P. with the common ratio, then the common points $\myvec{x_1\\y_1},\myvec{x_2\\y_2}$ and $\myvec{x_3\\y_3}$
    \begin{enumerate}
     \item  are verices of a triangle
     \item  lie on a straight line
     \item  lie on a ellipse
     \item  lie on a circle
     \end{enumerate}
    \item If the equation of the locus of a point equidistant from the point($a_1$,$b_1$) and ($a_1$,$b_2$) is \begin{align}\myvec{a_1-b_2 & a_1-b_2}\vec {x} &= 0 \end{align} then the value of c is
    \begin{enumerate}
     \item $\sqrt {a_1^2+b_1^2-a_2^2-b_2^2}$\\
     \item  $\frac{1}{2(a_2^2+b_2^2-a_1^2-b_1^2)}$\\
     \item  $a_1^2-a_2^2+b_1^2-b_2^2$\\
     \item   $\frac{1}{2(a_1^2+b_1^2-a_2^2-b_2^2)}$\\
     \end{enumerate}
    \item Let A\myvec{2\\-3} and B\myvec{-2\\3} be verices of a triangle ABC. If the centroid of the triangle moves on the line \begin{align}\myvec{2 & 3}\vec {x} &= 1 \end{align} then the locus of the vertex C is the line
    \begin{enumerate}
     \item  $\myvec{3 & -2}\vec {x} =3$
     \item  $\myvec{2 & -3}\vec {x} = 7$  
     \item  $\myvec{3 & 2}\vec {x} = 5$
     \item  $\myvec{2 & 3}\vec {x} = 9$
     \end{enumerate}
    \item The equation of the straight line passing through the point (4,3) and making intercepts on the coordinate axis whose sum is -1 is
    \begin{enumerate}
     \item $\myvec{{\frac{1}{2}} & {\frac{-1}{3}}} \vec {x} = 1$ and$ \myvec{{\frac{-1}{2}} & 1} \vec {x} = 1$
     \item $\myvec{{\frac{1}{2}} & {\frac{-1}{3}}} \vec {x} = -1$ and$ \myvec{{\frac{-1}{2}} & 1} \vec {x} = -1$
     \item $\myvec{{\frac{1}{2}} & {\frac{1}{3}}} \vec {x} = 1 $and$ \myvec{{\frac{1}{2}} & 1} \vec {x} = 1$
     \item $\myvec{{\frac{1}{2}} & {\frac{1}{3}}} \vec {x} = -1 $and$ \myvec{{\frac{-1}{2}} & 1} \vec {x} =-1$
     \end{enumerate}
    \item If the sum of the slopes of the lines given by $\vec {x}^T \myvec{1 & -2c \\ 0 & -1} \vec {x} = 0$ is 4 times their product c has the value
    \begin{enumerate}
     \item  -2 
     \item  -1 
     \item  2
     \item  1
     \end{enumerate}
    \item If one of the lines given by \begin{align}\vec {x}^T \myvec{6 & -1 \\ 0 & 4c} \vec {x} &= 0\end{align} is \begin{align}\myvec{ 3 & 4} \vec {x} &= 0\end{align} then c equals 
    \begin{enumerate}
     \item  -3 
     \item  -1 
     \item  3
     \item  1
     \end{enumerate}
    \item The line parallel to x-axis and passing through the intersection of the lines \begin{align}\myvec{a & 2b} \vec {x} + 3b &= 0\end{align} and \begin{align}\myvec{b & -2a} \vec {x} - 3a &= 0\end{align} , where \myvec{a\\b}$\neq$\myvec{0\\0}
    \begin{enumerate}
     \item  below the x-axis at a distance of 3/2 from it
     \item  below the x-axis at a distance of 2/3 from it
     \item  above the x-axis at a distance of 3/2 from it
     \item  above the x-axis at a distance of 2/3 from it
     \end{enumerate}
    \item If a vertex of a triangle is \myvec{1\\1}and the mid points of two sides through this vertex are\myvec{-1\\2}and \myvec{3\\2} then the centroid of the triangle is
    \begin{enumerate}
     \item  $\myvec{-1\\ \frac{7}{3}}$ 
     \item $\myvec{\frac{-1}{3}\\\frac{7}{3}}$
     \item  $\myvec{1\\ \frac{7}{3}}$
     \item  $\myvec{\frac{1}{3}\\\frac{7}{3}}$
     \end{enumerate}
    \item A straight line through the point A\myvec{3\\4} is such that its intercepts between the axes is bisected at A. Its equation is
    \begin{enumerate}
     \item $\myvec{1 & 1} \vec {x} = 7$
     \item  $\myvec{3 & -4} \vec {x} = -7$
     \item  $\myvec{4 & 3} \vec {x} = 24$
     \item  $\myvec{3 & 4} \vec {x} = 25$
     \end{enumerate}
    \item If $\myvec{a\\a^2}$ falls inside the angle made by the lines \begin{align}\myvec{-1 & 2} \vec {x} &= 0\end{align}, x$\textgreater$0 and \begin{align}\myvec{-3 & 1} \vec {x} &= 0\end{align}, $\vec {x} \textgreater 0$,then a belong to
    \begin{enumerate}
     \item  $\myvec{0\\\frac{1}{2}}$
     \item  $\myvec{3\\\infty}$
     \item  $\myvec{\frac{1}{2}\\3}$
     \item  $\myvec{-3\\-\frac{1}{2}}$
     \end{enumerate}
    \item Let $A=\myvec{h\\k}, B=\myvec{1\\1}$ and  $C=\myvec{2\\1}$ be the vertices of a right angled triangle with AC has its hypotenuse. If the area of the triangle is 1 sq.unit, then the set of values which 'k' can take is given by\\
    \begin{enumerate}
     \item  \myvec{-1\\3}
     \item  \myvec{-3\\-2}
     \item  \myvec{1\\3}
     \item  \myvec{0\\2}\\
     \end{enumerate}
    \item Let $P=\myvec{-1\\0}, Q=\myvec{0\\0}$ and $R=\myvec{3\\\sqrt3}$ be three point. The equation of the bisector of the angle PQR is\\ 
    \begin{enumerate}
     \item  $\myvec{\sqrt3/2 & 1} \vec {x} = 0$
     \item  $\myvec{1 & \sqrt3} \vec {x} = 0$
     \item  $\myvec{\sqrt3 & 1} \vec {x} = 0$
     \item  $\myvec{1 & \sqrt3/2} \vec {x} = 0$\\
     \end{enumerate}
    \item If one of the lines of \begin{align}\vec{x}^T\myvec{-m & (1-m^2) \\ 0 & m} \vec {x} &= 0\end{align} is a bisector of the angle between the lines  \begin{align}\vec{x}^T\myvec{0 & 0 \\ 1 & 0} \vec {x} &= 0\end{align} , then m is\\ 
    \begin{enumerate}
     \item  1 
     \item  2
     \item  -1/2
     \item  -2
     \end{enumerate}
    \item The perpendicular bisector of the line segment joining P\myvec{1\\4} and Q\myvec{k\\3} has y-intercept -4. Then a possible value of k is
    \begin{enumerate}
     \item  1 
     \item  2
     \item  -2
     \item  -4
     \end{enumerate}
    \item The shortest distance between the line \begin{align}\myvec{-1 & 1}\vec{x} &= 1\end{align} and the curve \begin{align}\vec{x}^T\myvec{0 & 0 \\ 0 & 1} \vec {x} - \myvec{-1 & 0} &= 0\end{align}  is:\\
    \begin{enumerate}
     \item  $\frac{2\sqrt3}{8}$ 
     \item  $\frac{3\sqrt2}{5}$
     \item  $\frac{\sqrt3}{4}$ 
     \item  $\frac{3\sqrt2}{8}$\\
     \end{enumerate}
    \item The lines \begin{align}\myvec{(p(p^2+1) & -1}\vec{x} + q &= 0\end{align} and \begin{align}\myvec{(p^2+1) & (p^2+1)}\vec{x} + 2q &= 0\end{align} are perpendicular to a common line for:
    \begin{enumerate}
     \item  exactly one values of p
     \item  exactly two values of p
     \item  more than two values of p
     \item  no value of p
     \end{enumerate}
    \item Three distinct points A,B and C are given in the two dimensional coordinates plane such that the ratio of the distance of any one of them from the point\myvec{1\\0} to the distance from the point\myvec{-1\\0} is equal to $\frac{1}{3}$. Then the circumcentre of the triangle ABC is at the point:
    \begin{enumerate}
     \item  $\myvec{\frac{5}{4}\\0}$
     \item  $\myvec{\frac{5}{2}\\0}$
     \item  $\myvec{\frac{5}{3}\\0}$
     \item  $\myvec{0\\0}$
     \end{enumerate}
    \item The line L given by \begin{align}\myvec{1/5 & 1/b}\vec{x} &= 1\end{align}  passes through the point\myvec{13\\32}. The line K is parallel to L and has the equation \begin{align}\myvec{\frac{1}{c} & \frac{1}{3}}\vec{x} = 1\end{align}. Then the distance between L and K is:
    \begin{enumerate}
     \item  $\sqrt17$
     \item  $\frac{17}{\sqrt15}$
     \item $\frac{23}{\sqrt17}$
     \item  $\frac{23}{\sqrt15}$
     \end{enumerate}
    \item The lines \begin{align}L_1:\myvec{-1 & 1}\vec{x} &= 0\end{align}  and \begin{align}L_2:\myvec{2 & 1}\vec{x} &= 0\end{align} intersect the line \begin{align}L_3:\myvec{0 & 2}\vec{x} + 2 &= 0\end{align} at P and Q respectively. The bisector of the acute angle between $L_1$ and $L_2$ intersects $L_3$ at R.
     Statement-1: The ratio PR:RQ equals 2$\sqrt2$:$\sqrt5$
     Statement-2: In any triangle, bisector of an angle divides the triangle into two similar triangles
     \begin{enumerate}
     \item  Statement-1 is True, Statement-2 is True; Statement-2 is not a correct explanation for Statement-1
     \item  Statement-1 is True, Statement-2 is False
     \item  Statement-1 is False, Statement-2 is True
     \item  Statement-1 is True, Statement-2 is True; Statement-2 is a correct explanation for Statement-1
     \end{enumerate}
    \item If the line \begin{align}\myvec{2 & 1}\vec{x} &= k\end{align} passes through the point which divides the line segment joining the points\myvec{1\\1} and\myvec{2\\4}in the ration 3:2, then k equals:\\
    \begin{enumerate}
     \item $\frac{29}{5}$ 
     \item  5
     \item  6
     \item $\frac{11}{5}$\\
     \end{enumerate}
    \item A ray of light along \begin{align}\myvec{1 & \sqrt3y}\vec{x} &= \sqrt3\end{align} gets reflected upon reaching x-axis, then the equation of the reflected ray is:\\
    \begin{enumerate}
     \item  $\myvec{-1 & 1}\vec{x} = \sqrt3$
     \item  $\myvec{-1 & \sqrt3}\vec{x} = -\sqrt3$
     \item  $\myvec{-\sqrt3 & 1}\vec{x} = -\sqrt3$
     \item  $\myvec{-1 & \sqrt3}\vec{x} = -1$\\
     \end{enumerate}
    \item The x coordinate of the incentre of the triangle that has the coordinates of mid points of its sides as \myvec{0\\1},\myvec{1\\1} and \myvec{1\\0}:\\
    \begin{enumerate}
     \item  2+$\sqrt2$
     \item  2-$\sqrt2$
     \item  1+$\sqrt2$
     \item  1-$\sqrt2$\\
     \end{enumerate}
    \item Let PS be the median of the triangle with vertices P\myvec{2\\2},Q\myvec{6\\-1} and R\myvec{7\\3}. The equation of the line passing through \myvec{1\\-1} and parallel to PS is :
    \begin{enumerate}
     \item  $\myvec{4 & 7}\vec{x} + 3 = 0$
     \item  $\myvec{2 & -9}\vec{x} - 11 = 0$
     \item  $\myvec{4 & -7}\vec{x} - 11 = 0$
     \item  $\myvec{2 & 9}\vec{x} + 7 = 0$
     \end{enumerate}
    \item Let a,b,c and d be non zero numbers. If the point of the intersection of the lines $\myvec{4a & 2a}\vec{x} + c = 0$ and $\myvec{5b & 2b}\vec{x} + c = 0$ lies in the fourth quadrant and is equidistant from the two axes, then:\\
    \begin{enumerate}
     \item  3bc-2ad=0
     \item  3bc+2ad=0
     \item  2bc-3ad=0
     \item  2bc+3ad=0\\
     \end{enumerate}
    \item The numbers of points, having both coordinates as integers, that lie in the interior of the triangle with vertices \myvec{0\\0},\myvec{0\\41} and\myvec{41\\0} is:\\
    \begin{enumerate}
     \item  820 
     \item  780
     \item  901
     \item  861\\
     \end{enumerate}
    \item Two sides of rhombus are along the lines,\myvec{1&-1} $\vec {x}$+1=0 and\myvec{7&-1}$\vec {x}$-5=0. If its diagonals intersect at\myvec{-1\\-2} then which one of the following is a vertex of the rhombus?\\
    \begin{enumerate}
     \item  \myvec{\frac{1}{3}\\\frac{8}{3}}
     \item  \myvec{\frac{10}{3}\\\frac{7}{3}}
     \item  \myvec{-3\\-9}
     \item  \myvec{-3\\-8}\\
     \end{enumerate}
    \item A straight the through a fixed point(2,3) intersects the coordinate axes at distinct point P and Q. If O is the origin and the rectangle OPRQ is completed, then the locus of R is\\
    \begin{enumerate}
     \item  \myvec{2&3} $\vec {x}$=$\vec {x}^T \myvec{ 0 & 0  \\0 & 1}\vec {x}$
     \item  \myvec{3&2} $\vec {x}$=$\vec {x}^T\myvec{ 0 & 0  \\0 & 1}\vec {x}$
     \item  \myvec{3&2}$\vec {x}$=6($\vec {x}^T \myvec{ 0 & 0  \\0 & 1}\vec {x})$
     \item  \myvec{3&2}$\vec {x}$=6
     \end{enumerate}
    \item Consider the set of all lines\myvec{p&q} $\vec {x}$+r=0 such that 3p+2q+4r=0. Which one of the following statement is true?
    \begin{enumerate}
     \item  The lines are concurrent at the point\myvec{\frac{3}{4}\\\frac{1}{2}}
     \item  Each line passes through the origin
     \item  The lines are all parallel
     \item  The lines are not concurrent
     \end{enumerate}
    \item Slope of a line passing through P\myvec{2\\3} and intersecting the line \myvec{1&1}
    $\vec {x}$=7 at a distance of 4 units from P, is:
    \begin{enumerate}
     \item {$\frac{1-\sqrt5}{1+\sqrt5}$}
     \item {$\frac{1-\sqrt7}{1+\sqrt7}$}
     \item {$\frac{\sqrt7-1}{\sqrt7+1}$}
     \item {$\frac{\sqrt5-1}{\sqrt5+1}$}
     \end{enumerate}
\end{enumerate}

%\end{document}
