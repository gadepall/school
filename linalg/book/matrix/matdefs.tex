\renewcommand{\theequation}{\theenumi}
\begin{enumerate}[label=\arabic*.,ref=\thesubsection.\theenumi]
\numberwithin{equation}{enumi}
\item Show that 
\begin{align}
\min_{a,b,c} \abs{a + b\omega + c\omega^2}^2
\end{align}
%
where $\omega^3 = 1, \omega \ne 1$ and $a,b,c$ are distinct nonzero integers
can be expressed as
\begin{align}
\label{eq:13_opt}
\min_{\vec{x}} \frac{1}{2}\vec{x}^T\vec{A}\vec{x}
\end{align}
%
where 
\begin{align}
\vec{x}&= \myvec{a \\ b \\c}, 
\vec{A}= 2\vec{P}^T\vec{P},
\\
\vec{P}&= \myvec{1 & \cos \theta & -\cos \theta \\ 0 & \sin \theta & \sin \theta},
\theta = \frac{\pi}{3} 
\end{align}
\item Show that 
\begin{align}
\label{eq:13_A}
\vec{A}  = \myvec{
2 & 1 & -1 \\ 
1 & 2 & 1
\\
-1 & 1 & 2
}
\end{align}
%
\solution 
\begin{align}
\vec{A}&= \myvec{1 &  0 \\ \cos \theta & \sin \theta  \\ -\cos \theta  & \sin \theta} 
\myvec{1 & \cos \theta & -\cos \theta \\ 0 & \sin \theta & \sin \theta} 
\nonumber \\
&=\myvec{
1 & \cos \theta & -\cos \theta \\ 
\cos \theta & 1 & -\cos 2\theta
\\
-\cos \theta & -\cos 2\theta & 1
},
\end{align}
resulting in \eqref{eq:13_A}.
\begin{align}
\because \cos 2 \theta = -\cos \theta = -\frac{1}{2}
\end{align}
\item Show that the characteristic equation of $\vec{A}$ is 
\begin{align}
f(\lambda) = \lambda^3 -6\lambda^2 + 9\lambda
\end{align}
\item Show that the eigenvalues of $\vec{A}$ are 0 and 3.
\item Verify that $tr\brak{\vec{A}}$ is the sum of its eigenvalues.
\item Verify that $\det\brak{\vec{A}}$ is the product of  its eigenvalues.
\item Show that $\vec{A}$ is positive definite.
\item Show that $\vec{x}^{T}\vec{A}\vec{x}$ is convex.
\item Show that the unconstrained $\vec{x}$ that minimizes $\vec{x}^{T}\vec{A}\vec{x}$ is given by the line
\begin{align}
\vec{x} = k \myvec{1 \\ -1 \\ 1}
\end{align}
\item Find $\vec{y}$ such that 
\begin{align}
\vec{A}\vec{y} = \lambda\vec{y}
\end{align}
where $\lambda$ is an eigenvalue of $\vec{A}$.
\item Show that 
\begin{align}
\vec{A} = \vec{P}^{-1} \vec{D}\vec{P}
\end{align}
%
where $\vec{D}$  is a diagonal matrix comprising of the eigenvalues of $\vec{A}$
and the columns of $\vec{P}$ are the corresponding eigenvectors.
\item Find $\vec{U}$ such that 
\begin{align}
\vec{A} = \vec{U}^{T} \vec{D}\vec{U}, \vec{U}^{T} \vec{U} = \vec{I}
\end{align}
\item Show that 
\begin{align}
\vec{x}^T \vec{A}\vec{x}  = 3 \vec{v}^{T} \vec{v},
\end{align}
where
\begin{align}
\vec{v} = \vec{U}\vec{x}
\end{align}
%
\item Show that when the entries of $\vec{x}$ are unequal and integers, the solution of \eqref{eq:13_opt} can be expressed as
\begin{align}
\vec{x} = \myvec{1 \\ -1 \\ 0} + c\myvec{1 \\ -1 \\ 1}
\end{align}

%\nonumber \\
%\\
%\solution The desired $\vec{x}$ is obtained by 
%\begin{align}
%\frac{d}{d\vec{x}}\vec{x}^{T}\vec{A}\vec{x} &= \vec{0}
%\nonumber \\
%\implies \vec{A}\vec{x} &= \vec{0}
%\end{align}
\end{enumerate}
