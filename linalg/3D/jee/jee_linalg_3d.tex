\documentclass[journal,12pt,twocolumn]{IEEEtran}
\usepackage{setspace}
\usepackage{gensymb}
\usepackage{caption}
%\usepackage{multirow}
%\usepackage{multicolumn}
%\usepackage{subcaption}
%\doublespacing
\singlespacing
\usepackage{csvsimple}
\usepackage{amsmath}
\usepackage{multicol}
%\usepackage{enumerate}
\usepackage{amssymb}
%\usepackage{graphicx}
\usepackage{newfloat}
%\usepackage{syntax}
\usepackage{listings}
\usepackage{iithtlc}
\usepackage{color}
\usepackage{tikz}
\usetikzlibrary{shapes,arrows}



%\usepackage{graphicx}
%\usepackage{amssymb}
%\usepackage{relsize}
%\usepackage[cmex10]{amsmath}
%\usepackage{mathtools}
%\usepackage{amsthm}
%\interdisplaylinepenalty=2500
%\savesymbol{iint}
%\usepackage{txfonts}
%\restoresymbol{TXF}{iint}
%\usepackage{wasysym}
\usepackage{amsthm}
\usepackage{mathrsfs}
\usepackage{txfonts}
\usepackage{stfloats}
\usepackage{cite}
\usepackage{cases}
\usepackage{mathtools}
\usepackage{caption}
\usepackage{enumerate}	
\usepackage{enumitem}
\usepackage{amsmath}
%\usepackage{xtab}
\usepackage{longtable}
\usepackage{multirow}
%\usepackage{algorithm}
%\usepackage{algpseudocode}
\usepackage{enumitem}
\usepackage{mathtools}
\usepackage{hyperref}
%\usepackage[framemethod=tikz]{mdframed}
\usepackage{listings}
    %\usepackage[latin1]{inputenc}                                 %%
    \usepackage{color}                                            %%
    \usepackage{array}                                            %%
    \usepackage{longtable}                                        %%
    \usepackage{calc}                                             %%
    \usepackage{multirow}                                         %%
    \usepackage{hhline}                                           %%
    \usepackage{ifthen}                                           %%
  %optionally (for landscape tables embedded in another document): %%
    \usepackage{lscape}     


\usepackage{url}
\def\UrlBreaks{\do\/\do-}


%\usepackage{stmaryrd}


%\usepackage{wasysym}
%\newcounter{MYtempeqncnt}
\DeclareMathOperator*{\Res}{Res}
%\renewcommand{\baselinestretch}{2}
\renewcommand\thesection{\arabic{section}}
\renewcommand\thesubsection{\thesection.\arabic{subsection}}
\renewcommand\thesubsubsection{\thesubsection.\arabic{subsubsection}}

\renewcommand\thesectiondis{\arabic{section}}
\renewcommand\thesubsectiondis{\thesectiondis.\arabic{subsection}}
\renewcommand\thesubsubsectiondis{\thesubsectiondis.\arabic{subsubsection}}

% correct bad hyphenation here
\hyphenation{op-tical net-works semi-conduc-tor}

%\lstset{
%language=C,
%frame=single, 
%breaklines=true
%}

%\lstset{
	%%basicstyle=\small\ttfamily\bfseries,
	%%numberstyle=\small\ttfamily,
	%language=Octave,
	%backgroundcolor=\color{white},
	%%frame=single,
	%%keywordstyle=\bfseries,
	%%breaklines=true,
	%%showstringspaces=false,
	%%xleftmargin=-10mm,
	%%aboveskip=-1mm,
	%%belowskip=0mm
%}

%\surroundwithmdframed[width=\columnwidth]{lstlisting}
\def\inputGnumericTable{}                                 %%
\lstset{
%language=C,
frame=single, 
breaklines=true,
columns=fullflexible
}
 

\begin{document}
%
\tikzstyle{block} = [rectangle, draw,
    text width=3em, text centered, minimum height=3em]
\tikzstyle{sum} = [draw, circle, node distance=3cm]
\tikzstyle{input} = [coordinate]
\tikzstyle{output} = [coordinate]
\tikzstyle{pinstyle} = [pin edge={to-,thin,black}]

\theoremstyle{definition}
\newtheorem{theorem}{Theorem}[section]
\newtheorem{problem}{Problem}
\newtheorem{proposition}{Proposition}[section]
\newtheorem{lemma}{Lemma}[section]
\newtheorem{corollary}[theorem]{Corollary}
\newtheorem{example}{Example}[section]
\newtheorem{definition}{Definition}[section]
%\newtheorem{algorithm}{Algorithm}[section]
%\newtheorem{cor}{Corollary}
\newcommand{\BEQA}{\begin{eqnarray}}
\newcommand{\EEQA}{\end{eqnarray}}
\newcommand{\define}{\stackrel{\triangle}{=}}

\bibliographystyle{IEEEtran}
%\bibliographystyle{ieeetr}

\providecommand{\nCr}[2]{\,^{#1}C_{#2}} % nCr
\providecommand{\nPr}[2]{\,^{#1}P_{#2}} % nPr
\providecommand{\mbf}{\mathbf}
\providecommand{\pr}[1]{\ensuremath{\Pr\left(#1\right)}}
\providecommand{\qfunc}[1]{\ensuremath{Q\left(#1\right)}}
\providecommand{\sbrak}[1]{\ensuremath{{}\left[#1\right]}}
\providecommand{\lsbrak}[1]{\ensuremath{{}\left[#1\right.}}
\providecommand{\rsbrak}[1]{\ensuremath{{}\left.#1\right]}}
\providecommand{\brak}[1]{\ensuremath{\left(#1\right)}}
\providecommand{\lbrak}[1]{\ensuremath{\left(#1\right.}}
\providecommand{\rbrak}[1]{\ensuremath{\left.#1\right)}}
\providecommand{\cbrak}[1]{\ensuremath{\left\{#1\right\}}}
\providecommand{\lcbrak}[1]{\ensuremath{\left\{#1\right.}}
\providecommand{\rcbrak}[1]{\ensuremath{\left.#1\right\}}}
\theoremstyle{remark}
\newtheorem{rem}{Remark}
\newcommand{\sgn}{\mathop{\mathrm{sgn}}}
\providecommand{\abs}[1]{\left\vert#1\right\vert}
\providecommand{\res}[1]{\Res\displaylimits_{#1}} 
\providecommand{\norm}[1]{\lVert#1\rVert}
\providecommand{\mtx}[1]{\mathbf{#1}}
\providecommand{\mean}[1]{E\left[ #1 \right]}
\providecommand{\fourier}{\overset{\mathcal{F}}{ \rightleftharpoons}}
%\providecommand{\hilbert}{\overset{\mathcal{H}}{ \rightleftharpoons}}
\providecommand{\system}{\overset{\mathcal{H}}{ \longleftrightarrow}}
	%\newcommand{\solution}[2]{\textbf{Solution:}{#1}}
\newcommand{\solution}{\noindent \textbf{Solution: }}
\newcommand{\myvec}[1]{\ensuremath{\begin{pmatrix}#1\end{pmatrix}}}
\providecommand{\dec}[2]{\ensuremath{\overset{#1}{\underset{#2}{\gtrless}}}}
\DeclarePairedDelimiter{\ceil}{\lceil}{\rceil}
%\numberwithin{equation}{subsection}
%\numberwithin{equation}{section}
%\numberwithin{problem}{subsection}
%\numberwithin{definition}{subsection}
\makeatletter
\@addtoreset{figure}{section}
\makeatother

\let\StandardTheFigure\thefigure
%\renewcommand{\thefigure}{\theproblem.\arabic{figure}}
\renewcommand{\thefigure}{\thesection}


%\numberwithin{figure}{subsection}

%\numberwithin{equation}{subsection}
%\numberwithin{equation}{section}
%\numberwithin{equation}{problem}
%\numberwithin{problem}{subsection}
\numberwithin{problem}{section}
%%\numberwithin{definition}{subsection}
%\makeatletter
%\@addtoreset{figure}{problem}
%\makeatother
\makeatletter
\@addtoreset{table}{section}
\makeatother

\let\StandardTheFigure\thefigure
\let\StandardTheTable\thetable
\let\vec\mathbf
%%\renewcommand{\thefigure}{\theproblem.\arabic{figure}}
%\renewcommand{\thefigure}{\theproblem}

%%\numberwithin{figure}{section}

%%\numberwithin{figure}{subsection}



\def\putbox#1#2#3{\makebox[0in][l]{\makebox[#1][l]{}\raisebox{\baselineskip}[0in][0in]{\raisebox{#2}[0in][0in]{#3}}}}
     \def\rightbox#1{\makebox[0in][r]{#1}}
     \def\centbox#1{\makebox[0in]{#1}}
     \def\topbox#1{\raisebox{-\baselineskip}[0in][0in]{#1}}
     \def\midbox#1{\raisebox{-0.5\baselineskip}[0in][0in]{#1}}

\vspace{3cm}

\title{ 
	\logo{
JEE Problems in Linear Algebra: 3D
	}
}

%\author{ G V V Sharma$^{*}$% <-this % stops a space
%	\thanks{*The author is with the Department
%		of Electrical Engineering, Indian Institute of Technology, Hyderabad
%		502285 India e-mail:  gadepall@iith.ac.in. All content in this manual is released under GNU GPL.  Free and open source.}
	
%}	

\maketitle

%\tableofcontents

\bigskip

\renewcommand{\thefigure}{\theenumi}
\renewcommand{\thetable}{\theenumi}


\begin{abstract}
	A  collection of problems from JEE mains papers related to 3D geometry  are available in 
this document.  Students are expected to solve these using linear algebra.
\end{abstract}
\begin{enumerate}[label=\arabic*.]
\item  $\vec{A} = \myvec{a_1 \\ a_2 \\ a_3}$ is a solution of
\begin{align}
\myvec{1 & -8 & 7 \\ 9 & 2 & 3 \\ 1 & 1 & 1}\vec{x} &= \vec{0}
\end{align}
%
such that $\vec{A}$ lies on the plane
\begin{align}
\myvec{1 & 2 & 1}\vec{x} &= 6.
\end{align}
%
Find $2a_1+a_2+a_3$.
\item For any two $3 \times 3$ matrices $A$ and $B$, let $A+B = 2B^T$ and $3A+2B=I_3$.  Which of the following 
is true?
\begin{enumerate}
\item $5A+10B=2I_3$.
\item $10A+5B=3I_3$.
\item $2A+B=3I_3$.
\item $3A+6B=2I_3$.
\end{enumerate}
%
\item If the line, 
\begin{equation}
L_1:\frac{x_1-3}{1}=
\frac{x_2+2}{-1} = 
\frac{x_3+\lambda}{-2}
\end{equation}
lies in the plane
\begin{align}
\myvec{2 & -4 & 3}\vec{x} &= 2,
\end{align}
find the shortest distance between $L_1$ and
\begin{equation}
L_2:\frac{x_1-1}{12}=
\frac{x_2}{9} = 
\frac{x_3}{4}
\end{equation}
\item Given
\begin{align}
\vec{A} = \myvec{1 & 1 & 0}^T
\\
\vec{B} = \myvec{0 & 3 & 4}^T
\end{align}
and $\vec{B}_2$ such that
\begin{align}
BB_2&\parallel OA
\\
\vec{B}_2^T &\vec{A} = 0
\end{align}
where $\vec{O}$ is the origin, find $\brak{\vec{B}-\vec{B}_2}\times \vec{B}_2$.
\item Find the distance between the point $\myvec{1 & -5 & 9}^T$ from the plane 
\begin{align}
\myvec{1 & -1 & 1}\vec{x} &= 5,
\end{align}
along the line $x_1=x_2=x_3$.
\item The line
\begin{equation}
L:\frac{x_1-3}{2}=
\frac{x_2+2}{-1} = 
\frac{x_3+4}{3}
\end{equation}
lies in the plane
\begin{align}
\myvec{l & m & -1}\vec{x} &= 9,
\end{align}
%
Find $l^2+m^2$.

\item Let $\vec{A}, \vec{B}, \vec{C}$ be three unit vectors such that
\begin{equation}
\vec{A}\times\brak{\vec{B}\times\vec{C}} = \frac{\sqrt{3}}{2}\brak{\vec{B}+\vec{C}}.
\end{equation}
If $\vec{B}$ is not parallel to $\vec{C}$, then find the angle between and
$\vec{A}$ and $\vec{B}$.
%
\item Find the range of the shortest distance between the lines
\begin{align}
L_1&:\frac{x_1}{2}=
\frac{x_2}{2} = 
\frac{x_3}{1}
\\
L_2&:\frac{x_1+2}{-1}=
\frac{x_2-4}{8} = 
\frac{x_3-5}{4}
\end{align}
%
\item Find the distance of the point $\myvec{1 & -2 & 4}^T$ from the plane passing through the point $\myvec{1 
& 2 & 2}^T$ and 
perpendicular to the planes 
\begin{align}
\myvec{1 & -1 & 2}\vec{x} &= 3
\\
\text{and }\myvec{2 & -2 & 1}\vec{x} &= -12.
\end{align}
\item In $\triangle ABC$, right angled at $\vec{A}$,
\begin{align}
\vec{A} = \myvec{3\\ 1 \\ -1},
\vec{B} = \myvec{-1\\ 3 \\ p}
\vec{C} = \myvec{5\\ q \\ -4}
\end{align}
sketch the point $\myvec{p \\ q}$
\item Find the distance of the point $\myvec{1 & 3 & -7}$ from the plane passing through the point $\myvec{1 & 
-1 & 
-1}$, having normal perpendicular to both the lines
\begin{align}
L_1&:\frac{x_1-1}{1}=
\frac{x_2+2}{-2} = 
\frac{x_3-4}{3}
\\
L_2&:\frac{x_1-2}{2}=
\frac{x_2+1}{-1} = 
\frac{x_3+7}{-1}
\end{align}
\item If the image of the point 
\begin{equation}
\vec{P}=\myvec{1 \\ -2 \\ 3}
\end{equation}
in the plane
\begin{align}
\myvec{2 & 3 & -4}\vec{x} &= -22.
\end{align}
%
measured parallel to the line
\begin{align}
L:\frac{x_1}{1}=
\frac{x_2}{4} = 
\frac{x_3}{5}
\end{align}
%
is $Q$, find $PQ$.
\item Let
\begin{equation}
A=\myvec{2 \\ 1 \\ -2} \text{ and }
B=\myvec{1 \\ 1 \\ 0}.
\end{equation}
%
If
\begin{align}
\abs{\vec{C}-\vec{A}} &= 3,
\\
\abs{\brak{\vec{A}\times\vec{B}}\times\vec{C}} &= 3,
\\
\frac{\vec{C}^{T}\brak{\vec{A}\times\vec{B}}}{\abs{\vec{C}}\abs{\vec{A}\times\vec{B}}} &= 
\frac{\sqrt{3}}{2},
\end{align}
then find $\vec{A}^{T}\vec{C}$.
\item Find $b$ such that the planes
\begin{align}
\myvec{1 & 1 & 1}\vec{x} &= 1
\\
\myvec{1 & a & 1}\vec{x} &= 1
\\
\myvec{a & b & 1}\vec{x} &= 0
\end{align}
%
do not intersect.
\item If the shortest distance between the lines
\begin{align}
x + 2\lambda = 2y=-12z
\\
x=y+4\lambda=6z-12\lambda
\end{align}
is $4\sqrt{2}$, find $\lambda$.
\item Find the perpendicular distance from the point 
\begin{equation}
\vec{A}=\myvec{3 \\ 1 \\ 1}
\end{equation}
%
on the plane passing through the point
\begin{equation}
\vec{B}=\myvec{1 \\ 2 \\ 3}
\end{equation}
%
and containing the line
\begin{equation}
\vec{x}=\myvec{1 \\ 1 \\ 0}+\lambda\myvec{2 \\ 1 \\ 4}
\end{equation}
%
\item If 
\begin{align}
\abs{\vec{a}}=1
\\
\abs{\vec{b}}=2
\\
\abs{\vec{c}}=4
\\
\vec{a}+\vec{b}+\vec{c}=0.
\end{align}
Find
\begin{equation}
4\vec{a}^T\vec{b}+3\vec{b}^T\vec{c}+3\vec{c}^T\vec{a}
\end{equation}
\item Find the coordinates of the foot of the perpendicular from the point
\begin{equation}
\vec{B}=\myvec{1 \\ -2 \\ 1}
\end{equation}
%
on the plane containing the lines
\begin{align}
L_1:\frac{x_1+1}{6}=
\frac{x_2-1}{7} = 
\frac{x_3-3}{8}
\\
L_2:\frac{x_1-1}{3}=
\frac{x_2-2}{5} = 
\frac{x_3-3}{7}
\end{align}
%
\item Find the intersection of the planes
\begin{align}
\myvec{3 & -1 & 1}\vec{x} &= 1
\\
\myvec{1 & 4 & -2}\vec{x} &= 2
\end{align}
%
\item If $\vec{A},\vec{B},\vec{C},\vec{D}$ are the vertices of a parallelogram such that
\begin{align}
\vec{A}-\vec{C}&=\myvec{8 \\ -6 \\ 0}
\\
\vec{B}-\vec{D}&=\myvec{3 \\ 4 \\ -12}
\end{align}
%
Find its area.
\item Find $\lambda$ for which the planes
\begin{align}
\myvec{1 & \lambda & -1
\\
\lambda & -1 & -1
\\
1 & 1 & -\lambda
}
\vec{x}
=0
\end{align}
%
do not intersect at the origin.
%
\item In $\triangle ABC$,
\begin{align}
\vec{A}=\myvec{2 \\ 3 \\ 5}
\vec{B}=\myvec{-1 \\ 3 \\ 2}
\vec{C}=\myvec{\lambda \\ 5 \\ \mu}
\end{align}
%
The median through $\vec{A}$ is equally inclined to the coordinate axes.  Find $\brak{\lambda^3+\mu^3+5}$
\item If
\begin{align}
L_1&:\frac{x_1-1}{1}=
\frac{x_2-2}{2} = 
\frac{x_3+3}{\lambda^2}
\\
L_2&:\frac{x_1-3}{1}=
\frac{x_2-2}{\lambda^2} = 
\frac{x_3-1}{2}
\end{align}
%
are coplanar, find the number of distinct real values of $\lambda$.
%
\item The circumcentre of $\triangle ABC$ is 
\begin{align}
\vec{P}=\frac{\vec{A}+\vec{B}+\vec{C}}{4}
\end{align}
%
Find its orthocentre.
\end{enumerate}
\end{document}
