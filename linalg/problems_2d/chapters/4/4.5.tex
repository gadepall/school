\renewcommand{\theequation}{\theenumi}
\begin{enumerate}[label=\arabic*.,ref=\thesubsection.\theenumi]
\item Find the locus of a point which moves so that the sum of the squares of its distances from the sides of an equilateral triangle
is constant.
\item Find the locus of a point which moves so that the sum of the squares of its distances from $n$ fixed points is constant.
\item Find the locus of a point at which two given circles
subtend equal angles.
\item A circle passes through the four points $\myvec{a\\0}$, $\myvec{b\\0}$, $\myvec{0\\c}$, $\myvec{0\\d}$.  By what relation 
are $a$, $b$, $c$, $d$ connected?  Find the equation of the 
circle and show that the tangent at the point $\myvec{a\\c+d}$ is
\begin{align}
\myvec{a-b & c+d}\vec{x}
-a\brak{a-b}-\brak{c+d}^2=0
\end{align}
\numberwithin{equation}{enumi}
\item Write down the equations of the tangents to the circles
\begin{align}
\vec{x}^T\vec{x}+\myvec{-2a & 0}\vec{x}-5 = 0
\\
\vec{x}^T\vec{x}+\myvec{0 & -2b}\vec{x}-5 = 0
\end{align}
at their points of intersection and verify that they cut at right angles.
\item Find the equation of the tangent to the circle 
\begin{align}
\norm{\vec{x}}=a
\end{align}
 at the point $\myvec{a\cos\theta\\a\sin\theta}$ and show that the length of the
tangent intercepted by the lines 
\begin{align}
\vec{x}^T\myvec{1 & 0\\0 & -1}\vec{x} = 0
\end{align}
is $\pm 2a\sec\theta$.
\item $\vec{A}$ and $\vec{B}$ are two fixed points $\myvec{c\\0}$, $\myvec{-c\\0}$, and $\vec{P}$ moves so that $PA=k.PB$.  Find the locus of $\vec{P}$ and prove that it is 
cut orthogonally by any circle through $\vec{A}$ and $\vec{B}$.
\item Show that the common chord of the circles
\begin{align}
\vec{x}^T\vec{x}-\myvec{6 & 4}\vec{x}+9 = 0
\\
\vec{x}^T\vec{x}-\myvec{8 & 6}\vec{x}+23 = 0
\end{align}
is a diameter of the latter circle and find the angle at which the circles cut.
\item Prove analytically that the tangents to a circle at the ends of a chord are equally inclined to the chord.
\item Prove that for different values of $a$ the equation
\begin{align}
\vec{x}^T\vec{x}+\myvec{-2a\text{cosec}\alpha & 0}\vec{x}+a^2\cot^2\alpha = 0
\end{align}
represents a family of circles touching the lines 
\begin{align}
\myvec{\pm\tan\alpha & 1}\vec{x} = 0
\end{align}
%
Prove also that the locus of the poles of the line 
\begin{align}
\myvec{l & m}\vec{x} = 0
\end{align}
 with regard to the circles is the line
\begin{align}
\myvec{m\sin^2\alpha & l\cos^2\alpha}\vec{x}=0
\end{align}
\item Find the coordinates of the middle point of the chord 
\begin{align}
\myvec{l & m}\vec{x} = 1
\end{align}
of the circle
\begin{align}
\vec{x}^T\vec{x}+2\myvec{g & f}\vec{x}+c = 0
\end{align}
\item Prove that the points of intersection of the line 
\begin{align}
\myvec{l & m}\vec{x} = 1
\end{align}
and the circle
\begin{align}
\vec{x}^T\vec{x}+2\myvec{g & f}\vec{x}+c = 0
\end{align}
subtend a right angle at the origin if
\begin{align}
c\brak{l^2+m^2}+2gl+2fm+2=0
\end{align}
\item Prove that the equation of the circle having for diameter the portion of the line 
\begin{align}
\myvec{\cos\alpha & \sin\alpha}\vec{x} = p
\end{align}
 intercepted by the circle 
\begin{align}
\norm{\vec{x}} = a
\end{align}
 is
\begin{align}
\vec{x}^T\vec{x}-2p\myvec{\cos\alpha & \sin\alpha}\vec{x}+2p^2-a^2 = 0
\end{align}
\item Prove that if a chord of the circle 
\begin{align}
\norm{\vec{x}} = a
\end{align}
subtends a right angle at a fixed point $\vec{x}_1$, the locus of the middle point
of the chord is
\begin{align}
2\vec{x}^T\vec{x}-2\vec{x}_1^T\vec{x}+\norm{\vec{x}_1}^2 - a^2  = 0
\end{align}
\item Prove that the equation of any tangent to the circle
\begin{align}
\norm{\vec{x}-\myvec{a\\b}} = r
\end{align}
may be written in the form
\begin{align}
\myvec{\cos\theta & \sin\theta}\brak{\vec{x}-\myvec{a\\b}} = r
\end{align}
Deduce that the equation of the tangents from $\vec{x}_1$ to the circle is
\begin{multline}
r^2\norm{\vec{x}-\vec{x}_1}^2
\\
 =\sbrak{\cbrak{\vec{x}-\myvec{a\\b}}\myvec{0 & -1\\1 & 0}\cbrak{\vec{x}_1-\myvec{a\\b}}}^2
\end{multline}
\item Prove that the distances of two points from the centre of a circle are proportional to the distance of each point from the polar of the
other.
\item Prove that the tangents to the circles of a coaxal system drawn from a limiting point are bisected by
the radical axis.
\item Show that a common tangent to the two circles is bisected by their radical axis and subtends a right angle at either limiting point.
\item Prove that if a point moves so that the difference of the squares of the tangents from it to two given circles is constant its locus
 is a straight line parallel to the radical axis of the circles.
 \item Prove that the polars of a fixed point with regard to a family of coaxal circles all pass through another fixed point.
 \item The circles
 \begin{align}
\vec{x}^T\vec{x}+\myvec{-2a\sec\alpha & 0}\vec{x}-a^2 = 0
 \\
\vec{x}^T\vec{x}+\myvec{0 & -2a\text{cosec}\alpha}\vec{x}-a^2 = 0
 \end{align}
 where $\alpha$ is a given angle, both cut orthogonally every member of a coaxal family of circles.  Find the radical axis and the limiting
 points of the family.
 \item Prove that, if two points $\vec{P}$, $\vec{Q}$ are conjugate with regard to a circle, the circle on $PQ$ as diameter cuts the first circle orthogonally. 
 \item Prove that if $\vec{P}$, $\vec{Q}$ are conjugate points with regard to a circle, the circles
 with $\vec{P}$, $\vec{Q}$ as centres which cut the given circle orthogonally are orthogonal to one another.
 \item Prove that, if $PQ$ is a diameter of a circle, then $\vec{P}$, $\vec{Q}$ are conjugate points with regard to
 any circle which cuts the given circle orthogonally.
 \item Prove that if $\vec{P}$, $\vec{Q}$ are conjugate points with regard to a circle, the square on $PQ$ is equal to the
 sum of the squares on the tangents from $\vec{P}$, $\vec{Q}$ to the circle.
\renewcommand{\theequation}{\theenumi}
 \item The equation 
 \begin{align}
\vec{x}^T\vec{x}+\myvec{-2g & 0}\vec{x}+2g-5 = 0
 \end{align}
 where $g$ is a variable parameter, represents a family of coaxial circles.  
 Show that the radius of the smallest circle of the family is 2.
 \item Prove that, if perpendiculars are drawn from a fixed point $\vec{P}$ to the polars of $\vec{P}$ with regard to a
 family of coaxial circles, then the locus of the feet of these perpendiculars is a circle whose centre
 lies on the radical axis of the family.
\numberwithin{equation}{enumi}
 \item Prove that, if the points in which the line 
 \begin{align}
\myvec{l & m}\vec{x}+n=0
 \end{align}
meets the circle, 
 \begin{align}
\vec{x}^T\vec{x}+2\myvec{g & f}\vec{x}+c = 0
 \end{align}
 and those in which the line 
 \begin{align}
\myvec{l_1 & m_1}\vec{x}+n_1=0
 \end{align}
 meets 
 \begin{align}
\vec{x}^T\vec{x}+2\myvec{g_1 & f_1}\vec{x}+c_1 = 0
 \end{align}
lie on a circle,
 then 
 \begin{multline}
 2\brak{g-g_1}\brak{mn_1-m_1n}+2\brak{f-f_1}
 \\
 \brak{nl_1-n_1l}+\brak{c-c_1}\brak{lm_1-l_1m}=0
 \end{multline}
 \item Show that, if a diameter of a circle is the portion of the line 
 \begin{align}
\myvec{l & m}\vec{x}=1
 \end{align}
intercepted by the lines 
 \begin{align}
\vec{x}^T\myvec{a & h\\h & b}\vec{x} = 0
 \end{align}
 then the 
 equation of the circle is
 \begin{multline}
\brak{am^2-2hlm+bl^2}\vec{x}^T\vec{x}
\\
+2\myvec{\brak{hm-bl} & \brak{hl-am}}\vec{x}+ a+b = 0
 \end{multline}
\item Prove that, as $k$ varies, the equation
 \begin{align}
\vec{x}^T\vec{x}+2\myvec{a & b}\vec{x}+c + 2k\cbrak{\myvec{a & -b}\vec{x}+1}= 0
 \end{align}
 represents a system of coaxial circles.  Also prove that the orthogonal system is given by
 \begin{align}
\vec{x}^T\vec{x}+\myvec{\frac{c+2}{2a} & \frac{c-2}{2b}}\vec{x}+h\cbrak{\myvec{\frac{1}{2a}&\frac{1}{2b}}\vec{x} + 1} = 0
 \end{align}
 where $h$ is a variable parameter.
\end{enumerate}
