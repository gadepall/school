\renewcommand{\theequation}{\theenumi}
The following problems refer to the ellipse whose 
equation is
\begin{align*}
\vec{x}^T\myvec{b^2 & 0\\0 & a^2}\vec{x} =a^2b^2
\end{align*}
and that $C$ is its centre and $S$, $S_1$ its foci.
\begin{enumerate}[label=\arabic*.,ref=\thesubsection.\theenumi]
\numberwithin{equation}{enumi}
\item Prove that, if the tangent and normal at a point $P$ on an ellipse meet the major axis
in $T$, $G$, then the tangent from either end of the minor axis to the circle $TPG$ is equal in length to half the major
axis.
\item Show that if $\brak{x_1,y_1}$ are the coordinates of a point of intersection of the ellipses $\frac{x^2}{a^2}+\frac{y^2}{b^2}=1$ and $\frac{x^2}{a_1^2}+\frac{y^2}{b_1^2}=1$, the equations of their common tangents are $\pm \dfrac{xx_1}{aa_1}\pm \dfrac{yy_1}{bb_1}=1$.
\item The normal at any point $P$ of the ellipse meets the axis in $G$; a point $Q$ is taken in the
tangent so that $PQ=\lambda.PG$, where $\lambda$ is constant; prove that the locus of $Q$ is the ellipse
\begin{align*}
\frac{x^2}{a^2}+\frac{y^2}{b^2}=\frac{a^2+\lambda b^2}{a^2}
\end{align*}
\item Prove that the line $\dfrac{ax}{k^2-1}+\dfrac{by}{2k}+\dfrac{a^2-b^2}{k^2+1} = 0$ is a normal to the ellipse 
$\frac{x^2}{a^2}+\frac{y^2}{b^2}=1$ for all values of $k$.
\item Prove that the foot of the focal perpendicular on the normal at any point of an ellipse is at
a distance from the centre equal to the difference between the semi-major axis and the focal
radius vector to the point at which the normal is drawn.
\item Prove that the locus of the poles of normal chords of the ellipse is the curve
\begin{align*}
\frac{a^6}{x^2}+\frac{b^6}{y^2}=\brak{a^2-b^2}^2.
\end{align*}
\item $P$ is a point $\brak{x_1,y_1}$ on the ellipse and $PS$, $PS_1$ meet the curve again in $Q$, $R$.  Prove that
the equation of $QR$ is 
\begin{align*}
\frac{xx_1}{a^2}+\frac{yy_1}{b^2}\frac{1+e^2}{1-e^2} + 1 = 0,
\end{align*}
where $e$ is the eccentricity.
\item Show that two parallel tangents to an ellipse are met by any other tangent in points
which lie on conjugate diameters.
\item Prove that if $CP$, $CD$ be any two conjugate semi-diameters of an ellipse and $PF$ be drawn
perpendicular to $CD$ and produced both ways to $E$, $E_1$ so that
$PE=PE_1=CD$, then $CE.CE_1=CS^2$, where $S$ is a focus.
\item Tangents are drawn from points on the ellipse to the circle $x^2+y^2=r^2$, show that the chords of contact touch
the ellipse
\begin{align*}
a^2x^2+b^2y^2 = r^4
\end{align*}
\item Tangents $TP$, $TQ_1$ are drawn to an ellipse so that $SP$, $S_1Q$ are parallel.  Prove that $CT$ is parallel to $SP$ or $S_1Q$.
\item the normal to an ellipse at a point $P$ passes through one end of the minor axis, and $CD$ is the semidiameter conjugate to $CP$.  The
perpendicular from $C$ to $CD$ meets the auxiliary circle in $E$.  Prove that $DE$ is equal to half the distance
betwen the directrices.
\item Prove that, if $\brak{x_1,y_1}$ be the middle point of a chord of the ellipse, the equation of the chord is
\begin{align*}
\frac{xx_1}{a^2}+\frac{yy_1}{b^2} = \frac{x_1^2}{a^2}+\frac{y_1^2}{b^2}
\end{align*}
\item  If $Q$ be the pole of the chord of the ellipse which is normal at a point $P$, and if
$CR$ drawn through the centre $C$ perpendicular to $CQ$ meet the normal at $P$ in $R$, prove that the locus of $R$ is
\begin{align*}
\frac{x^2}{b^6}+\frac{y^2}{a^6} = \frac{1}{a^2b^2}
\end{align*}
\item Prove that, if the point $P$ lies on the ellipse $\frac{x^2}{a_1^2}+\frac{y^2}{b_1^2}=1$, its polar with regard to the 
ellipse $\frac{x^2}{a^2}+\frac{y^2}{b^2}=1$ touches
the ellipse
\begin{align*}
\frac{a_1^2x^2}{a^4}+\frac{b_1^2y^2}{b^4}=1
\end{align*}
\item Show that the polar with regard to the ellipse $\frac{x^2}{a^2}+\frac{y^2}{b^2}=1$ of a point on the circle ${x^2}+{y^2}=c^2$
 touches the ellipse $\frac{x^2}{a^4}+\frac{y^2}{b^4}=\frac{1}{c^2}$.
\item Prove that, if the tangents to an ellipse at $\brak{x_1,y_1}$ and $\brak{x_2,y_2}$ meet at $\brak{x,y}$ and the normals at
$\brak{\xi,\eta}$, then $a^2\xi=e^2xx_1x_2$ and $b^2\eta = -e^2yy_1y_2$, where $e$ is the eccentricity.
\item Show that, if $\brak{x,y}$ is the middle point of a chord of the ellipse,
and the tangents at the ends of the chord intersect in $\brak{x_1,y_1}$ and the normals
in $\brak{x_2,y_2}$, then
\begin{align*}
\frac{a^2x_2}{x_1}+\frac{b^2y_2}{y_1}=\brak{a^2-b^2}\brak{\frac{xx_1}{a^2}-\frac{yy_1}{b^2}}.
\end{align*}
\item Prove that, if the normals at two points $P$, $Q$ on an ellipse intersect on the diameter
that bisects $PQ$, then the two normals are at right angles.
\item Prove that if a chord of the ellipse subtends a right angle at the centre
then it touches the circle
\begin{align*}
\brak{x^2+y^2}\brak{a^2+b^2} = a^2b^2
\end{align*}
\item The locus of middle points of chords of the ellipse which subtend a right angle at its centre is
\begin{align*}
\frac{x^2}{a^4}+\frac{y^2}{b^4}=\frac{a^2+b^2}{a^2b^2}\brak{\frac{x^2}{a^2}+\frac{y^2}{b^2}}^2
\end{align*}
\item Show that the tangents at the extremties of all chords of the
ellipse which subtend a right angle at the centre, intersect on the ellipse
\begin{align*}
\frac{x^2}{a^4}+\frac{y^2}{b^4}=\frac{1}{a^2}+\frac{1}{b^2}
\end{align*}
\item If $P$ be any point on the ellipse whose axes are $AA_1$, $BB_1$ and if the parallel lines
$AP$, $BQ$ be drawn, $Q$ being on the ellipse; $Q$ will be one
extremity of the diamter conjugate to that drawn from $P$.
\item From a any point $T$ on one of the equiconjugate diamters of a conic whose centre is $O$, tangents $TP$, $TQ$ are drawn to 
the conic.  Show that $O$, $P$, $Q$, $T$ are concyclic.
\item Prove that, if two conjugate radii of an ellipse cut the
director circle in $T$, $T_1$ then $TT_1$ touches the ellipse.
\item If $PSP_1$, $QSQ_1$ be two focal chords and if $PQ$ be parallel to the major axis, show that $P_1Q_1$ bisects the distance
between $S$ and the nearer directrix.
\item Tangents are drawn at the extremities of conjugate diameters of an ellipse, and meet in $O$.  Prove that the
perpendicular from $O$ on the focal radius to a point of contact is half the
minor axis.
\item Show that the area of the rectangle fromed by two parallel tangents and the corresponding
normals to an ellipse is never greater than half the square on the line joining the foci.
\item Prove that the angle btween the normal and the central radius at a point on an ellipse
is greatest when the point is the end of one of the equiconjugate diameters.
\item $P$, $Q$ are two points on the ellipse, and $PS$, $QS_1$ intersect on the curve, prove that the locus of the pole of $PQ$ is
\begin{align*}
\frac{x^2}{a^2}+\frac{b^2y^2}{\brak{2a^2-b^2}^2} = 1
\end{align*}
\item Two conjugate diameters of an ellipse meet a fixed straight line $ls+my=1$ in $P$, $Q$, and the straight lines
through $P$, $Q$ perpendicular to these diameters intersect in $R$;  prove that the locus of
$R$ is the straight line
\begin{align*}
a^2lx + b^2my = a^2+b^2
\end{align*}
\item Prove that if $\alpha$, $\beta$ are the eccentric angles of two points $P$, $P_1$ on an ellipse such that
the focal distances $SP$, $S_1P_1$ are parallel, then 
\begin{align*}
\tan\frac{\alpha}{2}:\tan\frac{\beta}{2} = 1\pm e: 1\mp e,
\end{align*}
and $PP_1$ touches the ellipse
\begin{align*}
\frac{x^2}{a^4}+\frac{y^2}{b^4} = \frac{1}{a^2}
\end{align*}
\item Prove that the square of the sum of two conjugate radii is greatest when the radii are equal.
\item Prove that, if on the inward normal to an ellipse at $P$ a length $PQ$ be taken
equal to the conjugate radius $CD$, the locus of $Q$ is a circle of radius $a-b$.
\item Prove that, if $P$, $Q$ are corresponding points on an ellipse and its auxiliary circle, and the
normal at $P$ to the ellipse meets the normal at $Q$ to the circle in $R$, then the locus of $R$ is a circle of radius $a+b$.
\item Prove that, if lines drawn from any poin on an ellipse to the ends of a diameter $PCP_1$ meet the conjugate
diameter $DCD_1$ in $M$, $M_1$, then $CM.CM_1=CD^2$.
\item Prove that, if an ellipse slides between two straight lines
at right angles to one another, the locus of its centre is a circle.
\end{enumerate}
