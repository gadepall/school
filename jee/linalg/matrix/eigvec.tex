\renewcommand{\theequation}{\theenumi}
\begin{enumerate}[label=\arabic*.,ref=\thesubsection.\theenumi]
\numberwithin{equation}{enumi}

\item Let 
\begin{align}
\vec{P} = \myvec{1 & 1 & 1 \\ 0 & 2 & 2 \\ 0 & 0 & 3}, 
\vec{Q} = \myvec{2 & x & x \\ 0 & 4 & 0 \\ x & x & 6}
\label{eq:2019_qp2_2_q}
\end{align}
%

\item Find $x$ such that $PQ=QP$.
%
\\
\solution 
\begin{align}
\because \vec{Q} &= \myvec{2 & 0 & 0 \\ 0 & 4 & 0 \\ 0 & 0 & 6} +x\myvec{0 & 1 & 1 \\ 0 & 0 & 0 \\ 1 & 1 & 0}, 
\end{align}
\begin{align}
\vec{P}\vec{Q} = 
 \myvec{2 & 4 & 6 \\ 0 & 8 & 12 \\ 0 & 0 & 18}+x\myvec{1 & 2 & 1 \\ 2 & 2 & 0 \\ 3 & 3 & 0}
\end{align}
and 
\begin{multline}
\vec{Q}\vec{P} = 
\myvec{2 & 2 & 2 \\ 0 & 8 & 8 \\ 0 & 0 & 18}+x\myvec{0 & 2 & 5 \\ 0 & 0 & 0 \\ 1 & 3 & 3} 
\end{multline}
%
Thus, 
\begin{align}
\vec{P}\vec{Q} = \vec{Q}\vec{P} 
\implies 
\myvec{0 & 2 & 4 \\ 0 & 0 & 4 \\ 0 & 0 & 0} &= x\myvec{-1 & 0 & 4 \\ -2 & -2 & 0 \\ -2 & 0 & 3}
\end{align}
which has no solution.
\item If 
\begin{align}
\vec{R} = \vec{P}\vec{Q}\vec{P}^{-1},
\end{align}
verify whether 
\begin{align}
\det{\vec{R}}= 
\det\myvec{2 & x & x \\ 0 & 4 & 0 \\ x & x & 5} + 8
\end{align}
for all $x$.
\\
\solution 
\begin{align}
\det(\vec{R}) &= \det(\vec{P})\det(\vec{Q})\det(\vec{P})^{-1}=\det(\vec{Q})
\nonumber \\
&= 4\brak{12-x^2}
\end{align}
%
Thus, 
\begin{multline}
\det(\vec{R}) - \det\myvec{2 & x & x \\ 0 & 4 & 0 \\ x & x & 5}
\\
= 4\cbrak{\brak{12-x^2}-\brak{10-x^2}}
\\
= 8
\end{multline}
%
which is true.
\item For $x= 0$, if 
\begin{align}
\label{eq:2019_qp2_2_ab}
\vec{R}\myvec{1 \\ a \\ b} = 6\myvec{1 \\ a \\ b}, 
\end{align}
%
then show that 
\begin{align}
a+b = 5.
\end{align}
\solution For $x=0$, 
\begin{align}
\vec{R} = \vec{P}\vec{Q}\vec{P}^{-1},
\end{align}
%
where $\vec{Q}$ is a diagonal matrix.  This is the eigenvalue decomposition of $\vec{R}$.  Thus, 
\begin{align}
\label{eq:2019_qp2_2_6eig}
\vec{R}\myvec{1 \\ 2 \\ 3} = 6\myvec{1 \\ 2 \\ 3}, 
\end{align}
%
where 
\begin{align}
\myvec{1 \\ 2 \\ 3}
\end{align}
%
is the eigenvector corresponding to the eigenvalue $6$. Comparing with \eqref{eq:2019_qp2_2_6eig},
\begin{align}
a=2,b=3 \implies a+b = 5.
\end{align}
%
\item For $x = 1$, verify if  there exists a  vector $\vec{y}$ for which $\vec{R}\vec{y} = \vec{0}$. 
\\
\solution 
\begin{align}
\vec{R}\vec{y} &= \vec{0} \implies \vec{P}\vec{Q}\vec{P}^{-1}\vec{y} = \vec{0}
\nonumber \\
\implies \vec{Q} \vec{z}&= \vec{0},
\label{eq:2019_qp2_2_null}
\end{align}
%
where 
\begin{align}
\label{eq:2019_qp2_2_yz}
\vec{z} = \vec{P}^{-1}\vec{y} 
\end{align}
For $x=1$, \eqref{eq:2019_qp2_2_q} and \eqref{eq:2019_qp2_2_null} yield
\begin{align}
\label{eq:2019_qp2_2_x1}
\myvec{2 & 1 & 1 \\ 0 & 4 & 0 \\ 1 & 1 & 6}\vec{z} &= \vec{0} 
\end{align}
Using row reduction,
\begin{align}
%\label{eq:2019_qp2_2_x1}
\myvec{2 & 1 & 1 \\ 0 & 4 & 0 \\ 1 & 1 & 6} &\leftrightarrow
\myvec{2 & 1 & 1 \\ 0 & 1 & 0 \\ 0 & 1 & 11} \leftrightarrow
\myvec{1 & 0 & -5 \\ 0 & 1 & 0 \\ 0 & 1 & 11} \leftrightarrow
\nonumber \\
\myvec{1 & 0 & -5 \\ 0 & 1 & 0 \\ 0 & 0 & 11} &
\end{align}
%
Thus, $\vec{Q}^{-1}$ exists and 
\begin{align}
\vec{z} = \vec{0} \implies \vec{y}= \vec{0}
\end{align}
upon substituting from \eqref{eq:2019_qp2_2_yz}.
This implies that the null space of $\vec{R}$ is empty. 
\end{enumerate}

