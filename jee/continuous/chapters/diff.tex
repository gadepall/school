\renewcommand{\theequation}{\theenumi}
\begin{enumerate}[label=\arabic*.,ref=\thesubsection.\theenumi]
\numberwithin{equation}{enumi}

	\item If y=f$\left(\dfrac{2x-1}{x^{2}+1}\right)$ and f'(x) = $\sin x^{2}$, then $\dfrac{dy}{dx}$ = ........
	
	\item $f_r(x)$,$g_r(x)$, $h_r(x)$, r = 1,2,3 are polynomials in x such that $f_r(a)$ = $g_r(a)$=$h_r(a)$ r = 1,2,3 and \\
	\begin{equation*}
	F(x) = 
   \begin{vmatrix} 
   f_{1}(x) & f_{2}(x) & f_{3}(x)  \\
   g_{1}(x) & g_{2}(x) & g_{3}(x)  \\
   h_{1}(x) & h_{2}(x) & h_{3}(x)  \\
   \end{vmatrix} 
\end{equation*}
 Then F'(x) at x = a is .............
	\item If f(x) = $\log_x$(ln x), then $f'(x)$ at x = e is ............\\
	
	\item The derevative of $sec^{-1}\left(\dfrac{1}{2x^2-1}\right)$ with respect to $\sqrt{1-x^2}$ at x = $\dfrac{1}{2}$ is ...........\\
	\item If f(x) = $|x-2|$ and g(x) = f[f(x)], then g'(x) = ............. for x$>$20
	\item If x$e^{xy}$ = y + $\sin^{2}$x, then at x=0, $\dfrac{dy}{dx}$ = ...........\\
	\item The derivavtive of an even function is always an odd function.\\
	\item If $y^{2}$ = P(x), a polynomial of degree 3, \\
	then 2$\dfrac{d}{dx}\left(y^3\dfrac{d^2y}{dx^2}\right)$, equals .....
	\begin{itemize}
	\begin{multicols}{2}
	\item [(a)]P'''(x)+P'(x)
	\item [(b)]P''(x)P'''(x)
	\item [(c)]P(x)P'''(x)
	\item [(d)]constant
	
	\end{multicols}
	\end{itemize}
	\item Let f(x) be a quadratic expression which is positive for all the real values of x. If g(x) = f(x)+f'(x)+f''(x), then for any real x,
	
	 \begin{itemize}
	\begin{multicols}{2}
	\item [(a)]g(x)$<$0
	\item [(b)]g(x)$>$0
	\item [(c)]g(x) = 0
	\item [(d)]g(x)$\geq$0
	
	\end{multicols}
	\end{itemize}
	\item If y = $\sin x^{\tan x}$ then $\dfrac{dy}{dx}$ is equals to\\
	(a) $\sin x^{\tan x}$(1+$\sec^2x\log\sin(x)$\\
	(b) $\tan x(\sin x)^{\tan x-1}$.$\cos x$\\
	(c) $\sin x^{\tan x}\sec^2x\log\sin x$\\
	(d) $\tan x(\sin x)^{\tan x-1}$\\
	\item $x^2 + y^2$ = 1
	\begin{itemize}
	\begin{multicols}{2}
	\item [(a)]yy"-2$y'^2$+1 = 0
	\item [(b)]yy"+$y'^2$+1 = 0
	\item [(c)]yy"+$y'^2$-1 = 0
	\item [(d)]yy"+2$y'^2$+1=0
	
	\end{multicols}
	\end{itemize}
	\item Let f:(0 $\infty$) $\to$ R and F(x) = $\int\limits_0^x$ f(t)dt. If F($x^2$) = $x^2$(1+x), then f(4) equals
	\begin{itemize}
	\begin{multicols}{2}
	\item [(a)]$\dfrac{5}{4}$\\
	\item [(b)]7\\
	\item [(c)]4\\
	\item [(d)]0\\
	
	\end{multicols}
	\end{itemize}
	\item If y is a function of x and $\log(x+y)$-2xy = 0, then the value of y'(0) is equals to
	\begin{itemize}
	\begin{multicols}{2}
	\item [(a)]1
	\item [(b)]-1
	\item [(c)]2
	\item [(d)]0
	
	\end{multicols}
	\end{itemize}
	\item If f(x) is a twice diffferentiable function and given that \\
	f(1) = 1;f(2) = 4,f(3) = 9, then\\
	(a) f"(x)=2 for $\forall$ x $\in$ (1,3)\\
	(b) f"(x)=f'(x) = 5 for some x $\in$ (2,3)\\
	(c) f"(x)=3 for $\forall$ x $\in$ (1,3)\\
	(d) f"(x) =2 for some x $\in$ (1,3)\\
	\\
	\item $\dfrac{d^2x}{dy^2}$ equals\\
	\begin{itemize}
	\begin{multicols}{2}
	\item [(a)]$\left(\dfrac{d^2y}{dx^2}\right)^{-1}$
	\item [(b)]-$\left(\dfrac{d^2y}{dx^2}\right)^{-1}$ $\left(\dfrac{dy}{dx}\right)^{-3}$
	\item [(c)]$\left(\dfrac{d^2y}{dx^2}\right)$ $\left(\dfrac{dy}{dx}\right)^{-2}$
	\item [(d)]-$\left(\dfrac{d^2y}{dx^2}\right)$  $\left(\dfrac{dy}{dx}\right)^{-3}$
	
	\end{multicols}
	\end{itemize}
	\item Let g(x)=$\log$f(x) where f(x) is twice differentiable positive function on (0, $\infty$) such that f(x+1) = xf(x). Then, for N=1,2,3,.......\\
	g"$\left(N+\dfrac{1}{2}\right)$-g"$\left(\dfrac{1}{2}\right)$ = \\
	g"$\left(N+\dfrac{1}{2}\right)$ - g"($\dfrac{1}{2}$ = \\
	\\
	(a) -4$\left\{1+\dfrac{1}{9}+\dfrac{1}{25}+......+\dfrac{1}{(2N-1)^2}\right\}$\\
	\\
	(b) 4$\left\{1+\dfrac{1}{9}+\dfrac{1}{25}+......+\dfrac{1}{(2N-1)^2}\right\}$\\
	\\
	(c) -4$\left\{1+\dfrac{1}{9}+\dfrac{1}{25}+......+\dfrac{1}{(2N+1)^2}\right\}$\\
	\\
	(d) 4$\left\{1+\dfrac{1}{9}+\dfrac{1}{25}+......+\dfrac{1}{(2N+1)^2}\right\}$\\
	\\
	\item Let f:[0, 2] $\to$ R be a function which is continuous on [0,2] and is differntiable on(0,2) with f(0) = 1. Let\\
	F(x) = $\int\limits_0^{x^2}$f($\sqrt{t}$)dt for x $\in$ [0,2]. If F'(x) = f'(x) for all x $\in$ (0,2), then F(2) equals
	\begin{itemize}
	\begin{multicols}{2}
	\item [{a}]$e^2$ - 1
	\item [(b)]$e^4$ - 1
	\item [(c)]e - 1
	\item [(d)]$e^4$
	
	\end{multicols}
	\end{itemize}
	\item Let f:R $\to$ R, g:R $\to$ R and h : R $\to$ R be differentiable functions such that f(x)=$x^2$+3x+2, g(f(x))=x and h(g(g(x)))=x for all x $\in$ R. Then
	\begin{itemize}
	\begin{multicols}{2}
	\item [(a)]g'(2) = $\dfrac{1}{15}$\\
	\item [(b)]h'(1) = 666\\
	\item [(c)]h(0) = 16\\
	\item [(d)]h(g(3)) = 36\\
	
	\end{multicols}
	\end{itemize}
	\item For every twice differentiable function\\
	 f:R $\to$ [-2,2] with $(f(0))^2$ + $(f(0))^2$ = 85, which of the following statement(s) is True?\\
	 \\
	(a) There exist r,s $\in$ R, where r$<$s, such that f is on the open interval (r,s)\\
	(b) There exists $x_0\to$(-4,0) such that $|f'(x_0)|\leq$ 1 \\
	(c) $\displaystyle{\lim_{x \to \infty}}$ = 1\\
	(d) There exists $\alpha$ $\to$ (-4, 4) such that f($\alpha$)+f'($\alpha$) = 0 and f'($\alpha$) $\neq$ 0\\
	\\
	\item For any positive integer n, define $f_n$:(0, $\infty$)$\to$R as $f_n$(x) = $\sum\limits_{j = 1}^n$ $\tan^{-1}$ $\left(\dfrac{1}{1+(x+j)(x+j-1)}\right)$ for all x$\in$ (0,$\infty$).\\
	Here, the inverse trignometric function $\tan^{-1}$(x) assumes values in $\left(-\dfrac{\pi}{2}, \dfrac{\pi}{2}\right)$\\
	Then, which of the following statements are True?\\
	(a) $\sum\limits_{j=1}^5 \tan^2 f_j(0))$ = 55\\
	(b) $\sum\limits_{j=1}^{10} (1+f'_j(0))\sec^2(f'_j(0))$ = 10\\
	(c) For any fixed positive integer n,$\displaystyle{\lim_{x \to \infty}}$ $\tan (f_n(x))$ = $\dfrac{1}{n}$\\
	(d)  For any fixed positive integer n,$\displaystyle{\lim_{x \to \infty}}$ $\sec^2 (f_n(x))$ = 1 
	\item Let f:(0,$\pi$)$\to$R be a twice differentiable \\
	\\
	function such that $\displaystyle{\lim_{t \to \infty}}$ $\dfrac{f(x)\sin t-f(t)\sin x}{t-x}$ = \\
	\\
	$\sin^2x$ for all x$\in(0,\pi)$. If $\dfrac{\pi}{6}$ = -$\dfrac{\pi}{12}$, then which of the following statement(s)  are True?\\
	\\
	\item [(a)] f$\left(\dfrac{\pi}{4}\right)$ = $\dfrac{\pi}{4\sqrt{2}}$\\
	
	\item [(b)]f(x)$<\dfrac{x^4}{6}-x^2$ for all x$\in(0,\pi)$\\
	\item [(c)]There exist $\alpha \in (0,\pi)$ such that f'($\alpha$) = 0\\
	\item [(d)]f"$\left(\dfrac{\pi}{2}\right)$+f$\left(\dfrac{\pi}{2}\right)$ = 0\\
	\item Find the derivative of $\sin(x^2+1)$ with respect to x from first principle.\\
	\item Find the derivative of\\
	$$
	F(x)=
	\begin{cases}
	\dfrac{x-1}{2x^2-7x+5},   & \text{when $x \neq 1$}\\
	\\
	-\dfrac{1}{3},   &\text{when $x = 1$}
	\end{cases}
	$$
	\\
	\item Given y = $\dfrac{5x}{3\sqrt{(1-x)^2}}$ + $\cos^2$(2x+1); Find $\dfrac{dy}{dx}$.
	\item y = $e^{x\sin x^3}$ + ($\tan x)^x$. Find  $\dfrac{dy}{dx}$\\
	\item Let f be a twice differentiable function such that\\
	f"(x) = -f(x) and f'(x) = g(x), h(x) = [$f(x)]^2$ + [$g(x)]^2$
	find h(10) if h(5) =11\\
	\item If $\alpha$ be a repeated root of a quadratic equation f(x) = 0 and A(x), B(x) and C(x) be polynomials of degree 3,4 and 5 respectively, then show that \begin{equation*}
   \begin{vmatrix} 
   A(x) & B(x) & C(x)  \\
   A(\alpha) & B(\alpha) & C(\alpha)  \\
   A'(\alpha) & B'(\alpha) & C'(\alpha)  \\
   \end{vmatrix} 
\end{equation*}\\
	is divisible by f(x), where prime denotes the derivatives.\\
	\item If x = $\sec \theta$ - $\cos \theta$ and y= $\sec^n \theta$ - $\cos^n \theta$, then show that $(x^2 + 4)\left(\dfrac{dy}{dx}\right)^2$ = $n^2(y^2 + 4)$.\\
	\item Find $\dfrac{dy}{dx}$ at x = -1, when\\
	\\
	$(\sin y)^{\sin\left(\frac{\pi}{2} x \right)}$ + $\dfrac{\sqrt{3}}{2}\sec^{-1}(2x)$ + $2^x \tan(ln(x+2))$ = 0\\
	\\
	\item  y = $\dfrac{ax^2}{(x-a)(x-b)(x-c)}$ + $\dfrac{bx}{(x-b)(x-c)}$ + $\dfrac{c}{(x-c)}$+1, prove that\\
	\\
	 $\dfrac{y'}{y}$ = $\dfrac{1}{x}\left(\dfrac{a}{a-x}+\dfrac{b}{b-x}+\dfrac{c}{c-x}\right)$.\\
	\item Let f(x) = 2+$\cos x$ for all real x.
	\\
	\\
	STATEMENT-1: For each real t, there exist a point c in [t,t+$\pi$] such that f'(c) = 0 because\\
	STATEMENT-2 f(t) = f(t+2$\pi$) for each real t.\\
	\\
	(a) Statement-1 is True, Statement-2 is True; Statement-2 is a correct explanation of Statement-1\\
	(b) Statement-1 is True, Statement-2 is True; Statement-2 is NOT a correct explanation of Statement-1\\
	(c) Statement-1 is True, Statement-2 is False\\
	(d) Statement-1 is False, Statement-2 is True.\\
	\item Let f and g be real valued functions defined on interval (-1,1) such that g"(x) is continuous, g(0) $\neq$ 0. g'(0) = 0, g" $\neq$ 0, and f(x) = g(x)$\sin x$\\
	\\
	STATEMENT-1: $\displaystyle{\lim_{x \to 0}}$ [g(x)cot x - g(0)cosec x] = f"(0) and\\
	STATEMENT-2: f'(0) = g(0)\\
	\\
	(a) Statement-1 is True, Statement-2 is True; Statement-2 is a correct explanation of Statement-1\\
	(b) Statement-1 is True, Statement-2 is True; Statement-2 is NOT a correct explanation of Statement-1\\
	(c) Statement-1 is True, Statement-2 is False\\
	(d) Statement-1 is False, Statement-2 is True.\\
	\\
	\item If the function f(x) =$x^3 + e^{\dfrac{x}{2}}$ and g(x) =$f^{-1}$(x), then the value of g'(1) is\\
	\\
	\item Let f($\theta$) = sin$\left(\tan^{-1}\left(\dfrac{\sin \theta}{\sqrt{\cos2\theta}}\right)\right)$, where\\
	 $-\dfrac{\pi}{4}<\theta<\dfrac{\pi}{4}$, Then the value of $\dfrac{d}{d(\tan \theta)}$(f($\theta$)) is 
	\item If y = (x+$\sqrt{1+x^2})^n$, then (1+$x^2$)$\dfrac{d^2y}{dx^2}$+x$\dfrac{dy}{dx}$ is\\
	(a) $n^2y$\\
	(b) -$n^2y$\\
	(c) -y\\
	(d) $2x^2y$\\
	\item If f(y) =$e^y$, g(y) = y; y$>$0 and\\
	 \\
	 F(t) = $\int\limits_0^t$ f(t-y)g(y)dy, then\\
	 (a) F(t) = t$e^{-t}$\\
	 (b) F(t) = 1-t$e^{-t}$(1+t)\\
	 (c) F(t) = $e^{t}$-(1+t)\\
	 (a) F(t) = t$e^{t}$\\
	\\
	\item f(x) = $x^n$, then the value of\\
	\\
	f(1)-$\dfrac{f'(1)}{1!} + \dfrac{f"(1)}{2!} + \dfrac{f"'}{3!}+.........\dfrac{(-1)^n f^n((1)}{n!}$ is\\
	\begin{itemize}
	\begin{multicols}{4}
	\item [(a)]1
	\item [(b)]$2^n$
	\item [(c)]$2^n$-1
	\item [(d)]0
	
	\end{multicols}
	\end{itemize}
	\item Let f(x) be a polynomial function of second degree. If f(1) = f(-1) and a,b,c are in A.P then f'(a),f'(b),f'(c) are in\\
	
	(a)Arthemetic-Geometric progression\\
	(b)A.P\\
	(c)G.P\\
	(d)H.P\\
	\\
	\item  If $e^{{y+e}^y+e^{y+...\infty}}$, x$>$0, then $\dfrac{dy}{dx}$ \\
	\begin{itemize}
	\begin{multicols}{4}
	\item [(a)]$\dfrac{1+x}{x}$
	\item [(b)]$\dfrac{1}{x}$
	\item [(c)]$\dfrac{1-x}{x}$
	\item [(d)]$\dfrac{x}{1+x}$
	
	\end{multicols}
	\end{itemize}
	\item The value of a for which sum of the squares of the roots of the equation $x^2$-(a-2)x-a-1 = 0 assume the least value is\\
	\begin{itemize}
	\begin{multicols}{4}
	\item [(a)]1
	\item [(b)]0
	\item [(c)]3
	\item [(d)]2
	
	\end{multicols}
	\end{itemize}
	\item If the roots of the equation $x^2$-bx+c = 0 be two consecutive integers, then $b^2$-4ac equals\\
	\begin{itemize}
	\begin{multicols}{4}
	\item [(a)]-2
	\item [(b)]3
	\item [(c)]2
	\item [(d)]1
	
	\end{multicols}
	\end{itemize}
	
	\item let f:R$\to$R be a differentiable function having f(2) = 6, f'(2) = $\dfrac{1}{48}$ Then $\displaystyle{\lim_{x \to f(x)}}$ $\int_6^{f(x)}\dfrac{4t^3}{x-2}$dt equals \begin{itemize}
	\begin{multicols}{4}
	\item [(a)]24
	\item [(b)]36
	\item [(c)]12
	\item [(d)]18
	
	\end{multicols}
	\end{itemize}
	\item The set of points where f(x) =$\dfrac{x}{1+|x|}$ is differentiable is\begin{itemize}
	\begin{multicols}{2}
	\item [(a)]$(-\infty,0) \cup (0,\infty)$
	\item [(b)]$(-\infty,-1) \cup (-1,\infty)$
	\item [(c)]$(-\infty,\infty)$
	\item [(d)]$(0,\infty)$
	
	\end{multicols}
	\end{itemize}
	\item If $x^m. y^n$ = $(x+y)^{m+n}$, then $\dfrac{dy}{dx}$ is
	\begin{itemize}
	\begin{multicols}{4}
	\item [(a)]$\dfrac{y}{x}$
	\item [(b)]$\dfrac{x+y}{xy}$
	\item [(c)]xy
	\item [(d)]$\dfrac{x}{y}$
	
	\end{multicols}
	\end{itemize}
	\item Let y be an implicit function of x defined by $x^{2x}-2x^xcot y-1$ = 0. Then y'(1) equals
	 \begin{itemize}
	\begin{multicols}{4}
	\item [(a)]1
	\item [(b)]log 2
	\item [(c)]-log 2
	\item [(d)]-1
	
	\end{multicols}
	\end{itemize}
	\item Let f:(-1,1)$\to$R be a differentiable function with f(0) = -1 and f'(0) = 1. Let g(x)=$[f(2f(x)+2))]^2$ Then g'(0) = 
	 \begin{itemize}
	\begin{multicols}{4}
	\item [(a)]-4
	\item [(b)]0
	\item [(c)]-2
	\item [(d)]4
	
	\end{multicols}
	\end{itemize}
	\item $\dfrac{d^2x}{dx^y}$ equals:
	 \begin{itemize}
	\begin{multicols}{2}
	\item [(a)]-$\left(\dfrac{d^2y}{dx^2}\right)^{-1} $ $\left(\dfrac{dy}{dx}\right)^{-3}$\\
	\item [(b)]$\left(\dfrac{d^2y}{dx^2}\right) $ $\left(\dfrac{dy}{dx}\right)^{-2}$\\
	\item [(c)]-$\left(\dfrac{d^2y}{dx^2}\right) $  $\left(\dfrac{dy}{dx}\right)^{-3}$\\
	\item [(d)]$\left(\dfrac{d^2y}{dx^2}\right)^{-1} $	\\
	\end{multicols}
	\end{itemize}
	\item If y= sec($\tan^{-1}x)$, then $\dfrac{dy}{dx}$ at x = 1 is equals to:
	\begin{itemize}
	\begin{multicols}{4}
	\item [(a)]$\dfrac{1}{\sqrt{2}}$
	\item [(b)]$\dfrac{1}{2}$
	\item [(c)]1
	\item [(d)]$\sqrt{2}$
	
	\end{multicols}
	\end{itemize}
	\item If g is the inverse of a function f and\\
	 $f^{-1}(x)$ = $\dfrac{1}{1+x^5}$ then g'(x) is equals to:
	\begin{itemize}
	\begin{multicols}{2}
	\item [(a)]$\dfrac{1}{1+\left(g(x)\right)^5}$\\
	\item [(b)]1+$\left(g(x)\right)^5$\\
	\item [(c)]1+$x^5$\\
	\item [(d)]$5x^4$\\
	
	\end{multicols}
	\end{itemize}
	\item If x = -1 and x = 2 are extreme points of f(x) =$\alpha log|x|+\beta x^2 +x$ then
	\begin{itemize}
	\begin{multicols}{2}
	\item [(a)]$\alpha = 2$, $\beta = -\dfrac{1}{2}$\\
	\item [(b)]$\alpha = 2$, $\beta = \dfrac{1}{2}$\\
	\item [(c)]$\alpha = -6$, $\beta = \dfrac{1}{2}$\\
	\item [(d)]$\alpha = -6$, $\beta = -\dfrac{1}{2}$\\
	
	\end{multicols}
	\end{itemize}
	\item If for x$\in$ $\left(0,\dfrac{1}{4}\right)$, the derivative of $tan^{-1}\left(\dfrac{6x\sqrt{x}}{1-9x^3}\right)$ is $\sqrt{x}$.g(x), then g(x) equals:
	\begin{itemize}
	\begin{multicols}{2}
	\item [(a)]$\dfrac{3}{1+9x^3}$\\
	\item [(b)]$\dfrac{9}{1+9x^3}$\\
	\item [(c)]$\dfrac{3x\sqrt{x}}{1-9x^3}$\\
	\item [(d)]$\dfrac{3x}{1-9x^3}$\\
	
	\end{multicols}
	\end{itemize}
	
\end{enumerate}
