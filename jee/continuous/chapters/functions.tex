\renewcommand{\theequation}{\theenumi}
\begin{enumerate}[label=\arabic*.,ref=\thesubsection.\theenumi]
\numberwithin{equation}{enumi}

\item The values of $f(x)$ = 3 $\sin\Bigg(\sqrt{\frac{\pi^2}{16}-x^2}\Bigg)$ lie in the interval.......

\item For the function f(x) = $\frac{x}{1+e^{1/x}}$, x $\neq$ 0 and f(x) = 0, x = 0
the derivative from the right, $f^{'}(0+)$= ........, and the derivative from the left, $f^{'}(0-)$= ..........

\item The domain of the funtion f(x)=$\sin^{-1}\Big(log_2\frac{x^2}{2}\Big)$ is given by .......

\item Let A be a set of n distinct elements. Then the total number of distinct functions from A to A is.......and out of these.......are onto functions.

\item If 
\begin{align*}
f(x) = \sin ln\Big(\frac{\sqrt{4 - x^{2}}}{1-x}\Big),
\end{align*}
then domain of f(x) is.... and its range is......

\item There are exactly two distinct linear functions.......,and......which map [-1,1] onto [0,2].

\item If f is an even function defined on the interval (-5,5), then four real values of x satisfying the equation f(x)=$f(\frac{x+1}{x+2})$ are.........., ..........., ........., and......

\item If
\begin{align*}
 f(x)= \sin^{2}x + \sin^{2} \Big(x+\frac{\pi}{3}\Big) + cosxcos\Big(x+\frac{\pi}{3}\Big)
\end{align*}
and g$\big(\frac{5}{4}\big)=1$, then (gof)(x)= ........

\item If f(x)=$(a-x^{n})^{1/n}$ where $a > 0$ and n is a positive integer, then f[f(x)]=x.

\item The function f(x)=$\frac{x^{2}+4x+30}{x^{2}-8x+18}$ is not one-to-one.
                                          
\item If $f_{1}(x)$ and $f_{2}(x)$ are defined on domains $D_{1}$ and $D_{2}$ respectively, then $f_{1}(x)+ f_{2}(x)$  is defined on $D_{1} \cup D_{2}$.

\item Let R be the set of real numbers. If f:R $\to$ R is a function defined by f(x)=$x^{2}$, then f is :
\begin{enumerate}
\item Injective but not surjective
\item Surjective but not injective
\item Bijective
\item None of these.
\end{enumerate}

\item The entire graphs of the equation y = $x^{2}+kx-x+9$ is stirctly above the x-axis if and only if
\begin{enumerate}
\item $k < 7$
\item $-5 < k < 7$
\item $k > -5$
\item None of these.
\end{enumerate}

\item Let f(x)= $\begin{vmatrix} x-1 \end{vmatrix}$. Then
\begin{enumerate}
\item $f(x^{2})=(f(x))^{2}$
\item f(x+y)=f(x)+f(y)
\item $f(\begin{vmatrix} x \end{vmatrix}) = \begin{vmatrix} f(x) \end{vmatrix}$
\item None of these
\end{enumerate}

\item If x satisfies $\begin{vmatrix} x-1 \end{vmatrix} + \begin{vmatrix} x-2 \end {vmatrix} + \begin{vmatrix} x-3 \end{vmatrix} \geq 6$,then
\begin{enumerate}
\item $0 \leq x \leq 4$
\item $x \leq -2$ or $x \geq 4$
\item $x \leq 0$ or $x \geq 4$
\item None of these
\end{enumerate}

\item If f(x)=$\cos(ln x)$, then f(x)f(y) - $\frac{1}{2}\Big[f\big(\frac{x}{y}\big)+f(xy)\Big]$  has the value
\begin{enumerate}
\item -1
\item $\frac{1}{2}$
\item -2
\item none of these
\end{enumerate}

\item The domain of definition of the function y=$\frac{1}{log_{10}(1-x)} + \sqrt{x+2}$ is
\begin{enumerate}
\item (-3,2) excluding -2.5
\item $[0,1]$ excluding 0.5
\item $[-2,1)$ excluding 0
\item none of these
\end{enumerate}

\item Which of the following functions is periodic?
\begin{enumerate}
\item f(x)=x-[x] where [x] denotes the largest integer less than or equal to the real number x
\item f(x)=$\sin\frac{1}{x}$ for $x \neq 0$, f(0)=0
\item f(x)=x$\cos x$
\item none of these
\end{enumerate}

\item Let f(x)=$\sin x$ and g(x) = ln $\begin{vmatrix} x \end{vmatrix}$. If the ranges of the composition functions $fog$ and $gof$ are $R_{1}$ and $R_{2}$ respectively, then
\begin{enumerate}
\item $R_1 = \{u: -1 \leq u < 1\}, R_2 = \{v: -\infty < v < 0\}$
\item $R_1 = \{u: -\infty < u < 0\}, R_2 = \{v: -1 \leq v \leq 0\}$
\item $R_1 = \{u: -1 < u < 1\}, R_2 = \{v: -\infty < v < 0\}$
\item $R_1 = \{u: -1 \leq u \leq 1\}, R_2 = \{v: -\infty < v \leq 0\}$
\end{enumerate}

\item Let $f(x)=(x+1)^{2}-1$, x $\geq -1$. Then the set $\{x: f(x)= f^{-1}(x)\}$ is
\begin{enumerate}
\item $\{0,-1, \frac{-3+i\sqrt{3}}{2}, \frac{-3-i\sqrt{3}}{2}\}$
\item $\{0,1,-1\}$
\item $\{0,-1\}$
\item empty
\end{enumerate}

\item The function f(x) = $\begin{vmatrix} px-q \end{vmatrix}$ + r$\begin{vmatrix} x \end{vmatrix}$,
 $x \in (-\infty,\infty)$ where p $>$ 0, q $>$ 0, r $>$ 0 assumes its minimum value only on one point if
\begin{enumerate}
\item p $\neq$ q
\item r $\neq$ q
\item r $\neq$ p
\item p = q = r
\end{enumerate}

\item Let $f(x)$ be defined for all $x > 0$ and be continuos. Let $f(x)$ satisfy $f(\frac{x}{y})$ = $f(x)-f(y)$ for all x,y and $f(e)=1$. Then
\begin{enumerate}
\item $f(x)$ is bounded
\item $f(\frac{1}{x}) \to 0$ as $x \to 0$
\item $xf(x)\to 1$ as $x \to 0$
\item $f(x)=ln x$
\end{enumerate}

\item If the function $f:[1,\infty) \rightarrow [1,\infty)$ is defined by $f(x)=2^{x(x-1)}$, then $f^{-1}(x)$ is
\begin{enumerate}
\item $\frac{1}{2}^{x(x-1)}$
\item $\frac{1}{2}(1+\sqrt{1+4log_2x})$
\item $\frac{1}{2}(1-\sqrt{1+4log_2x})$
\item not defined
\end{enumerate}

\item Let $f: R\rightarrow$ R be any function. Define g: R $\rightarrow$ R by $g(x)=\begin{vmatrix} f(x) \end{vmatrix}$ for all x. Then g is
\begin{enumerate}
\item onto if f is onto
\item one-one if f is one-one
\item continuos if f is continuos
\item differentiable if f is differentiable
\end{enumerate} 

\item The domain of definition of the function $f(x)$ given by the equation $2^{x} + 2^{y} = 2$ is 
\begin{enumerate}
\item $0< x \leq 1$
\item $0 \leq x \leq 1$
\item $-\infty < x \leq 0$
\item $-\infty < x < 1$
\end{enumerate}

\item Let $g(x) = 1+x-[x]$ and \[f(x)=\begin{cases} 
      -1, & x < 0\\
       0, & x = 0.\\
       1, & x > 0 
   \end{cases}\] then for all x, $f(g(x))$ is equal to
\begin{enumerate}
\item x
\item 1
\item $f(x)$
\item $g(x)$
\end{enumerate} 

\item If $f:[1,\infty) \rightarrow [2,\infty)$ is given by $f(x) = x+\frac{1}{x}$ then $f^{-1}(x)$ equals
\begin{enumerate}
\item $(x+\sqrt{x^2-4})/2$
\item $x/(1+x^2)$
\item $(x-\sqrt{x^2-4})/2$
\item $1+\sqrt{x^2-4}$
\end{enumerate}

\item The domain of definition of $f(x)=\frac{log_2(x+3)}{x^2+3x+2}$ is
\begin{enumerate}
\item $R\backslash\{-1,-2 \}$
\item $(-2,\infty)$
\item R$\backslash\{-1,-2,-3 \}$
\item $(-3,\infty)\backslash\{-1,-2 \}$
\end{enumerate} 

\item Let E=$\{1,2,3,4 \}$ and F=$\{1,2 \}$. Then the number of onto functions from E to F is
\begin{enumerate}
\item 14
\item 16
\item 12
\item 8
\end{enumerate}

\item Let $f(x)=\frac{\alpha x}{x+1}$,x $\neq$ -1. Then, for what value of $\alpha$ is $f(f(x))=x$?
\begin{enumerate}
\item $\sqrt{2}$
\item $-\sqrt{2}$
\item 1
\item -1
\end{enumerate} 

\item Suppose $f(x)=(x+1)^2$ for x $\geq$ -1. If $g(x)$ is the function whose graph is the reflection of the  graph of $f(x)$ with respect to the line y=x then $g(x)$ equals
\begin{enumerate}
\item $-\sqrt{x}-1, x \geq 0$
\item $\frac{1}{(x+1)^{2}}, x > -1$
\item $\sqrt{x+1}, x\geq-1$
\item $\sqrt{x}-1, x\geq0$
\end{enumerate}

\item Let function $f: R \rightarrow$ R be defined by $f(x) = 2x+\sin x$ for x $\in$ R, then $f$ is
\begin{enumerate}
\item one-to-one and onto
\item one-to-one but NOT onto
\item onto but NOT one-to-one
\item neither one-to-one nor onto
\end{enumerate}

\item If $f: [0,\infty) \rightarrow [0,\infty)$, and $f(x)=\frac{x}{1+x}$ then $f$ is
\begin{enumerate}
\item one-one and onto
\item one-one but not onto
\item onto but not one-one
\item neither one-one nor onto
\end{enumerate}

\item Domain of the definition of the function $f(x)=\sqrt{\sin^{-1}(2x)+\frac{\pi}{6}}$ for real valued $x$ ,is
\begin{enumerate}
\item $[-\frac{1}{4},\frac{1}{2}]$
\item $[-\frac{1}{2},\frac{1}{2}]$
\item $[-\frac{1}{2},\frac{1}{9}]$
\item $[-\frac{1}{4},\frac{1}{4}]$
\end{enumerate}

\item Range of the function $f(x)=\frac{x^2+x+2}{x^2+x+1}$;$x\epsilon$R is
\begin{enumerate}
\item $(1,\infty)$
\item $(1,\frac{11}{7}]$
\item $(1,\frac{7}{3}]$
\item $(1,\frac{7}{5}]$
\end{enumerate}

\item If $f(x)=x^{2}+2bx+2c^{2}$ and $g(x)=-x^{2}-2cx+b^{2}$ such that min $f(x) > max g(x)$, then the relation between b and c, is
\begin{enumerate}
\item no real value of b \& c
\item $0<c<b\sqrt{2}$
\item $\begin{vmatrix} c \end{vmatrix} < \begin{vmatrix} b\end{vmatrix} \sqrt{2}$
\item $\begin{vmatrix} c \end{vmatrix} > \begin{vmatrix} b\end{vmatrix} \sqrt{2}$
\end{enumerate}

\item If $f(x)=\sin x+\cos x$, $g(x)=x^{2}-1$, then $g(f(x))$ is invertible in the domain
\begin{enumerate}
\item $[0,\frac{\pi}{2}]$
\item $[-\frac{\pi}{4},\frac{\pi}{4}]$
\item $[-\frac{\pi}{2},\frac{\pi}{2}]$
\item $[0,\pi]$
\end{enumerate}

\item If the functions $f(x)$ and $g(x)$ are defined on R$\rightarrow$R such that \[f(x)=\begin{cases} 
       0, & x \in $rational$\\
       x, & x \in $irrational$\\
   \end{cases}\]; \[g(x)=\begin{cases} 
       0, & x \in $irrational$\\
       x, & x \in $rational$\\
   \end{cases}\] then $(f-g)(x)$ is
\begin{enumerate}
\item one-one \& onto
\item neither one-one nor onto
\item one-one but not onto
\item onto but not one-one
\end{enumerate}

\item X and Y are two sets and $f: X \rightarrow Y$. If $\{ f(c)=y; c \subset X, y \subset Y \}$ and  
$\{ f^{-1}(d) = x; d \subset Y, x \subset X \}$, then the true statement is
\begin{enumerate}
\item $f(f^{-1}(b))$ = b
\item $f^{-1}(f(a))$ = a
\item $f(f^{-1}(b))$ = b,b $\subset$ y
\item $f(f^{-1}(a))$ = a,a $\subset$ x
\end{enumerate} 

\item If $F(x) = \Big(f\Big(\frac{x}{2}\Big)\Big)^2+\Big(g\Big(\frac{x}{2}\Big)\Big)^2$ where $f^{"}(x)= -f(x)$ and $g(x)=f^{'}(x)$ and given that F(5)=5, then F(10) is equal to
\begin{enumerate}
\item 5
\item 10
\item 0
\item 15
\end{enumerate}

\item Let $f(x)=\frac{x}{(1+x^n)^{1/n}}$ for n $\geq$ 2 and $g(x)=
\underbrace{(fofo......of)}_{\text{f occurs n times}}(x)$. Then $\int x^{n-2}g(x)dx$ equals.
\begin{enumerate}
\item $\frac{1}{n(n-1)}(1+nx^n)^{1-\frac{1}{n}}+K$
\item $\frac{1}{(n-1)}(1+nx^n)^{1-\frac{1}{n}}+K$
\item $\frac{1}{n(n+1)}(1+nx^n)^{1+\frac{1}{n}}+K$
\item $\frac{1}{(n+1)}(1+nx^n)^{1+\frac{1}{n}}+K$
\end{enumerate}

\item Let f, g and h be real-valued functions defined on the interval $[0,1]$ by $f(x)=e^{x{^2}}+e^{{-x}^2}$, $g(x)=xe^{x{^2}}+e^{{-x}^2}$ and $h(x)=x^2e^{x{^2}}+e^{{-x}^2}$. If a, b and c denote, respectively, the absolute maximum of f,g and h on $[0,1]$, then
\begin{enumerate}
\item a = b and c $\neq$ b
\item a = c and a $\neq$ b
\item a $\neq$ b and c $\neq$ b
\item a = b = c
\end{enumerate}

\item Let $f(x)=x^2$ and $g(x)=\sin x$ for all $x \in$ R. Then the set of all x satisfying $(fogogof)(x)=(gogof)(x)$, where $(fog)(x)=f(g(x))$, is
\begin{enumerate}
\item $\pm \sqrt{n\pi}$, n $\in \{0,1,2.....\}$
\item $\pm \sqrt{n\pi}$, n $\in \{1,2.....\}$
\item $\frac{\pi}{2}+2n\pi$, n $\in \{.....-2,-1,0,1,2.....\}$
\item $2n \pi$, n $\in \{.....-2,-1,0,1,2.....\}$
\end{enumerate}

\item The function $f:[0,3] \rightarrow [1,29]$, defined by $f(x)=2x^{3}-15x^{2}+36x+1$, is
\begin{enumerate}
\item one-one and onto
\item onto but not one-one
\item one-one but not onto
\item neither one-one nor onto
\end{enumerate}

\item If y=$f(x)=\frac{x+2}{x-1}$ then
\begin{enumerate}
\item $x=f(y)$
\item $f(1)$=3
\item y increases with $x$ for $x<1$
\item $f$ is a rational function of $x$
\end{enumerate}

\item Let $g(x)$ be a function defined on $[-1,1]$. If the area of the equilateral triangle with two of its vertices at (0,0) and $[x,g(x)]$ is $\frac{\sqrt{3}}{4}$, then the function $g(x)$ is
\begin{enumerate}
\item $g(x)=\pm\sqrt{1-x^2}$
\item $g(x)=\sqrt{1-x^2}$
\item $g(x)=-\sqrt{1-x^2}$
\item $g(x)=\sqrt{1+x^2}$
\end{enumerate}

\item If f(x)=$\cos [\pi^2]x + \cos [-\pi^2]x$, where [x] stands for the greatest integer function, then
\begin{enumerate}
\item $f(\frac{\pi}{2})=-1$
\item $f(\pi)=1$
\item $f(-\pi)=0$
\item $f(\frac{\pi}{4})=1$
\end{enumerate}

\item If $f(x)$=3x-5, then $f^{-1}(x)$
\begin{enumerate}
\item is given by $\frac{1}{3x-5}$
\item is given by $\frac{x+5}{3}$
\item does not exist because $f$ is not one-one
\item does not exist because $f$ is not onto.
\end{enumerate}

\item If $g(f(x))=|\sin x|$ and $f(g(x))=(\sin \sqrt{x})^2$, then
\begin{enumerate}
\item $f(x)= \sin^2x$,$g(x)=\sqrt{x}$
\item $f(x)=\sin x$,$g(x)= \begin{vmatrix} x \end{vmatrix}$
\item $f(x)=x^2$, $g(x)=\sin \sqrt{x}$
\item $f$ and g cannot be determined.
\end{enumerate}

\item Let $f$:(0,1) $\rightarrow$ R be defined by $f(x)=\frac{b-x}{1-bx}$, where b is a constant such that $0<b<1$. Then 
\begin{enumerate}
\item $f$ is not invertible on (0,1)
\item $f\neq f^{-1}$ on (0,1) and $f^{'}(b)=\frac{1}{f^{'}(0)}$
\item $f=f^{-1}$ on (0,1) and $f^{'}(b)=\frac{1}{f^{'}(0)}$
\item $f^{-1}$ is differentiable (0,1)
\end{enumerate}

\item Let $f$:(-1,1) $\rightarrow$ IR be such that $f(\cos 4\Theta)=\frac{2}{2-\sec^2\Theta}$ for $\Theta\in\Big(0,\frac{\pi}{4}\Big) \cup \Big(\frac{\pi}{4},\frac{\pi}{2}\Big)$. Then the values of $f(\frac{1}{3})$ is 
\begin{enumerate}
\item 1-$\sqrt{\frac{3}{2}}$
\item 1+$\sqrt{\frac{3}{2}}$
\item 1-$\sqrt{\frac{2}{3}}$
\item 1+$\sqrt{\frac{2}{3}}$
\end{enumerate}

\item The funciton 
$f(x)$ = 2$\begin{vmatrix} x \end{vmatrix}$
 + $\begin{vmatrix} x+2 \end{vmatrix}$
 - $\begin{vmatrix}\begin{vmatrix} x+2 \end{vmatrix} 
 - 2\begin{vmatrix} x \end{vmatrix}\end{vmatrix}$ has a local minimum or a local maximum at x=
\begin{enumerate}
\item -2
\item $\frac{-2}{3}$
\item 2
\item $\frac{2}{3}$
\end{enumerate}

\item Let $f: (-\frac{\pi}{2},\frac{\pi}{2})\rightarrow R$ be given by $f(x)=(log(\sec x+\tan x))^3$. Then
\begin{enumerate}
\item $f(x)$ is an odd function 
\item $f(x)$ is one-one function
\item $f(x)$ is an onto function
\item $f(x)$ is an even function
\end{enumerate}

\item Let a$\in$R and let $f$: R $\rightarrow$ R be given by $f(x)=x^5-5x+a$. Then
\begin{enumerate}
\item $f(x)$ has three real roots if a$>$4
\item $f(x)$ has only real root if a$>$4
\item $f(x)$ has three real roots if a$<$-4
\item $f(x)$ has three real roots if -4$<$a$<$4
\end{enumerate}

\item Let $f(x)=\sin\Big(\frac{\pi}{6}\sin\Big(\frac{\pi}{2}\sin x\Big)\Big)$ for all $x\in$R and $g(x)=\frac{\pi}{2}\sin x$ for all x$\in$R. Let $(fog)(x)$ denote $f(g(x))$ and $(gof)(x)$ denote $g(f(x)).$ Then which of the following is true?
\begin{enumerate}
\item Range of $f$ is $[-\frac{1}{2},\frac{1}{2}]$
\item Range of fog is $[-\frac{1}{2},\frac{1}{2}]$
\item $\lim_{x \to 0}\frac{f(x)}{g(x)}=\frac{\pi}{6}$
\item There is an $x\in$R such that $(gof)(x)=1$
\end{enumerate}

\item Find the domain and range of the function $f(x)=\frac{x^2}{1+x^2}.$ Is the function one-to-one?

\item Draw the graph of $y = \begin{vmatrix} x \end{vmatrix}^{1/2}$ for $-1 \leq x \leq 1$.

\item If $f(x)=x^{9}-6x^{8}-2x^{7}+12x^{6}+x^{4}-7x^{3}+6x^{2}+x-3$, find $f(6)$.

\item Consider the following relations in the set of real numbers R. $R = \{(x,y);x \in R, y\in R, x^2+y^2 \leq 25\}$\\
$R^{'} = \{(x,y):x \in R, y \in R, y \geq \frac{4}{9}x^2\}$. Find the domain and the range of $R \cap R^{'}$. Is the relation $R \cap R^{'}$ a function?

\item Let A and B be two sets each with a finite number of elements. Assume that there is an injective mapping from A to B and that there is an injective mapping from B to A. Prove that there is a bijective mapping from A to B.

\item Let $f$ be a one-one function with domain $\{x,y,z\}$ and range $\{1,2,3\}$. It is given that exactly one of the following statements is true and the remaining two are false $f(x)=1$, $f(y) \neq 1$,$f(z) \neq 2$ determine $f^{-1}(1)$. 
 
\item Let R be the set of real numbers and $f$: R $\rightarrow$ R be such that for all x and y in R$\begin{vmatrix} f(x)-f(y) \end{vmatrix} \leq \begin{vmatrix} x-y \end{vmatrix}^3$. Prove that $f(x)$ is a constant.

\item Find the natural number $'a'$ for which $\sum_{k=1}^{n} f(a+k) = 16(2^n-1)$, where the function $'f'$ satisfies the relation $f(x+y)=f(x)f(y)$ for all natural numbers x,y and further $f(1)=2$.

\item Let $\{x\}$ and $[x]$ denotes the fractional and integral part of a real number $x$ respectively. Solve 4$\{x\}$ = $x+[x]$.

\item A function $f: IR \rightarrow IR$, where IR is the set of real numbers, defined by $f(x)=\frac{\alpha x^{2}+6x-8}{\alpha+6x-8x^{2}}$. Find the interval of values of $\alpha$ for which f is onto. Is the function one-to-one for $\alpha = 3$? Justify your answer.

\item Let $f(x) = Ax^{2}+Bx+c$ where A,B,C are real numbers. Prove that if $f(x)$ is an integer whenever x is an integer, then the numbers 2A,A+B and C are all integers. Conversely, prove that if the numbers 2A,A+B and C are all integers then $f(x)$ is an integer whenever $x$ is an integer.

\item Let $f: [0,4\pi] \rightarrow [0,\pi]$ be defined by $f(x)=\cos^{-1}(\cos x)$. The number of points $x \in [0,4\pi]$ satisfying the equation $f(x)=\frac{10-x}{10}$ is

\item The value of $((log_29)^{2})^{\frac{1}{log_2(log_29)}} \times (\sqrt{7})^{\frac{1}{log_47}}$ is ......

\item Let X be a set with exactly 5 elements and Y be a set with exactly 7 elements. If $\alpha$ is the number of one-one functions from X to Y and $\beta$ is the number of onto functions from Y to X, then the value of $\frac{1}{5!}(\beta-\alpha)$ is ......

\item The domain of $\sin^{-1}[log_3(x/3)]$ is
\begin{enumerate}
\item $[1,9]$
\item $[-1,9]$
\item $[-9,1]$
\item $[-9,-1]$
\end{enumerate}

\item The function $f(x)=log(x+\sqrt{x^{2}+1})$, is
\begin{enumerate}
\item neither an even nor an odd function
\item an even function
\item an odd function
\item a periodic function.
\end{enumerate}

\item Domain of definition of the function $f(x)=\frac{3}{4-x^{2}}+log_{10}(x^{3}-x)$, is
\begin{enumerate}
\item $(-1,0) \cup (1,2) \cup (2,\infty)$
\item (a,2)
\item $(-1,0) \cup (a,2)$
\item $(1,2) \cup (2,\infty)$.
\end{enumerate}

\item If $f: R \rightarrow R$ satisfies $f(x+y)=f(x)+f(y)$, for all $x$, y $\in$ R and f(1)=7, then 
$\sum_{r=1}^{n} f(r)$ is
\begin{enumerate}
\item $\frac{7n(n+1)}{2}$
\item $\frac{7n}{2}$
\item $\frac{7(n+1)}{2}$
\item $7n+(n+1)$
\end{enumerate}

\item A function $f$ from the set of natural numbers to integers defined by \[f(n)=\begin{cases} 
       \frac{n-1}{2}, & $when n is odd$\\
       -\frac{n}{2}, & $when n is even$\\
   \end{cases}\] is
\begin{enumerate}
\item neither one-one nor onto
\item one-one but not onto
\item onto but not one-one
\item one-one and onto both.
\end{enumerate}
    
\item The range of the function $f(x)= ^{7-x}P_{x-3}$ is
\begin{enumerate}
\item $\{1,2,3,4,5\}$
\item $\{1,2,3,4,5,6\}$
\item $\{1,2,3,4\}$
\item $\{1,2,3,\}$
\end{enumerate}

\item Let $f: R \rightarrow S$, defined by $f(x)=\sin x-\sqrt{3}\cos x+1$, is onto, then the interval of S is
\begin{enumerate}
\item $[-1,3]$
\item $[-1,1]$
\item $[0,1]$
\item $[0,3]$
\end{enumerate}

\item The graph of the function $y=f(x)$ is symmetrical about the line x=2, then
\begin{enumerate}
\item $f(x)=-f(-x)$
\item $f(2+x)=f(2-x)$
\item $f(x)=f(-x)$
\item $f(x+2)=f(x-2)$
\end{enumerate}

\item The domain of the function $f(x)=\frac{\sin^{-1}(x-3)}{\sqrt{9-x^{2}}}$ is
\begin{enumerate}
\item $[1,2]$
\item $[2,3)$
\item $[1,2]$
\item $[2,3]$
\end{enumerate}

\item Let $f: (-1,1) \rightarrow B$, be a function defined by $f(x)=tan^{-1}\frac{2x}{1-x^2}$, then $f$ is both one-one and onto when B is the interval
\begin{enumerate}
\item $(0,\frac{\pi}{2})$
\item $[0,\frac{\pi}{2})$
\item $[-\frac{\pi}{2},\frac{\pi}{2})$
\item $(-\frac{\pi}{2},\frac{\pi}{2})$
\end{enumerate}

\item A function is matched below against an interval where it is supposed to be increasing. Which of the following pairs is incorrectly matched?
\begin{table}[h!]
\centering
\begin{tabular}{c c} 
 Interval & Function\\ [0.5ex] 
 (a). $(-\infty,\infty)$ & $x^3-3x^2+3x+3$\\ 
 (b). $[2,\infty)$ & $2x^3-3x^2-12x+6$\\
 (c). $(-\infty,\frac{1}{3}]$ & $3x^2-2x+1$\\
 (d). $(-\infty,-4)$ & $x^3+6x^2+6$\\[1ex] 
\end{tabular}
\end{table}

\item A real valued function $f(x)$ satisfies the functional equation 
\begin{align*}
f(x-y)=f(x)f(y)-f(a-x)f(a+y)
\end{align*} 
where a is a given constant and $f(0)=1, f(2a-x)$ is equal to 
\begin{enumerate}
\item $-f(x)$
\item $f(x)$
\item $f(a)+f(a-x)$
\item $f(-x)$
\end{enumerate}

\item The Largest interval lying in $(\frac{-\pi}{2},\frac{\pi}{2})$ for which the function, 
\begin{align*}
f(x)=4^{-x^2}+cos^{-1}(\frac{x}{2}-1)+log(\cos x))
\end{align*},
is defined, is
\begin{enumerate}
\item $[-\frac{\pi}{4},\frac{\pi}{2})$
\item $[0,\frac{\pi}{2})$
\item $[0,\pi]$
\item $(-\frac{\pi}{2},\frac{\pi}{2})$
\end{enumerate} 

\item Let $f: N \rightarrow Y$ be a function defined as $f(x)=4x+3$ where
\begin{align*}
Y=\{ y \in N:y=4x+3 for some x \in N\}
\end{align*}
\begin{enumerate}
\item $g(y)=\frac{3y+4}{3}$
\item $g(y)=4+\frac{y+3}{4}$
\item $g(y)=\frac{y+3}{4}$
\item $g(y)=\frac{y-3}{4}$
\end{enumerate}

\item Let $f(x)=(x+1)^2-1$,x $\geq$ -1\\
Statement-1 : The set $\lbrace x:f(x)=f^{-1}(x)=\lbrace0,-1\rbrace\rbrace$\\
Statement-2 : $f$ is a bijection
\begin{enumerate}
\item Statement-1 is true,Statement-2 is true. Statement-2 is not a correct explanation for Statement-1.
\item Statement-1 is true, Statement-2 is false.
\item Statement-1 is false,Statement-2 is true.
\item Statement-1 is true, Statement-2 is true. Statement-2 is a correct explanation for Statement-1.
\end{enumerate}

\item For real x, let $f(x)=x^{3}+5x+1$, then 
\begin{enumerate}
\item $f$ is onto R but not one-one
\item $f$ is one-one and onto R
\item $f$ is neither one-one nor onto R
\item $f$ is one-one but not onto R
\end{enumerate}

\item The domain of the function $f(x)=\frac{1}{\sqrt{\begin{vmatrix} x \end{vmatrix} - x}}$ is
\begin{enumerate}
\item $(0,\infty)$
\item $(-\infty,0)$
\item $(-\infty,\infty)-\{0\}$
\item $(-\infty,\infty)$
\end{enumerate}

\item For $x \in R-\{0,1\}$, let $f_{1}(x)=\frac{1}{x}, f_{2}(x)=1-x$ and $f_{3}(x)=\frac{1}{1-x}$ be three given functions. If a function, J(x) satisfies $(f_{2}oJof_{1})(x)=f_{3}(x)$ then J(x) is equal to:
\begin{enumerate}
\item $f_3(x)$
\item $f_3(x)$
\item $f_2(x)$
\item $f_1(x)$
\end{enumerate} 

\item If the fractional part of the number $\frac{2^{403}}{15}$ is $\frac{k}{15}$, then k is equal to:
\begin{enumerate}
\item 6
\item 8
\item 4
\item 14
\end{enumerate}

\item If the function $f:R-\{1,-1\}$. A defined by f(x)=$\frac{x^2}{1-x^2}$, is surjective, then A is equal to:
\begin{enumerate}
\item R-$\{-1\}$
\item $[0,\infty)$
\item R-$[-1,0)$
\item R-$(-1,0)$
\end{enumerate}

\item Let
\begin{align*}
\sum_{k=1}^{10} f(a+k)=16(2^{10}-1),
\end{align*} 
where the function f satisifies f(x+y)=f(x)f(y) for all natural numbers x,y and f(a) is = 2. Then the natural number $'a'$ is:
\begin{enumerate}
\item 2
\item 16
\item 4
\item 3
\end{enumerate}

\textbf{Match the following}

\item Let the function defined in Colum 1 have domain $(-\frac{\pi}{2},\frac{\pi}{2})$ and range $(-\infty,\infty)$
\begin{table}[h!]
\centering
\begin{tabular}{c c} 
 Column I & Column II\\ [0.5ex] 
 (A) 1+2x & (p) onto but not one-one\\ 
 (B) $\tan x$ & (q) one-one but not onto\\
     & (r) one-one and onto\\
     & (s) neither one-one nor onto\\[1ex] 
\end{tabular}
\end{table}

\item Let $f(x)=\frac{x^2-6x+5}{x^2-5x+6}$\\
Match of expressions/statements in Column I with expressions/statements in Column II and indicate your answer by darkening the appropriate bubbles in the 4$\times$4 matrix given in the ORS.
\begin{table}[h!]
\centering
\begin{tabular}{c c} 
 Column I & Column II\\ [0.5ex] 
 (A) If $-1 < x < 1$, then $f(x)$ satisfies & (p) $0 < f(x) < 1$\\ 
 (B) If $1 < x < 2$, then $f(x)$ satisfies & (q) $f(x) < 0$\\
 (C) If $3 < x < 5$, then $f(x)$ satisfies   & (r) $f(x) > 0$\\
 (D) If $x > 5$, then $f(x)$ satisfies  & (s) $f(x) < 1$\\[1ex] 
\end{tabular}
\end{table}
 

\item Let $E_1 = \{ x \in R: x \neq 1 and \frac{x}{x-1} > 0\}$ and $E_2 = \{ x \in E_1:\sin^{-1}\Big(log_e(\frac{x}{x-1})\Big) is a real number\}$. (Here, the inverse trigonometric function $\sin^{-1}x$ assumes values in $[-\frac{\pi}{2},\frac{\pi}{2}]$).\\
Let $f: E_1 \rightarrow R$ be the function defined by $f(x)=log_e(\frac{x}{x-1})$ and g: $E_2 \rightarrow$ R be the function defined by $g(x)=sin^{-1}(log_e(\frac{x}{x-1}))$.
\begin{table}[h!]
\centering
\begin{tabular}{c c} 
 LIST-I & LIST-II\\ [0.5ex] 
  P. The range of $f$ is &         1. $(-\infty,\frac{1}{1-e}]\cup[\frac{e}{e-1},\infty)$\\ 
  Q. The range of g contains &     2. (0,1)\\
  R. The domain of $f$ contains &  3. $[-\frac{1}{2},\frac{1}{2}]$\\
  S. The domain of g is &          4. $(-\infty,0)\cup(0,\infty)$\\
                                 & 5. $(-\infty,\frac{e}{e-1}]$\\
                                 & 6. $(-\infty,0)\cup(\frac{1}{2},\frac{e}{e-1}]$\\[1ex] 
\end{tabular}
\end{table}\\
The correct option is:
\begin{enumerate}
\item $P \rightarrow 4; Q \rightarrow 2; R \rightarrow 1; S \rightarrow 1$
\item $P \rightarrow 4; Q \rightarrow 2; R \rightarrow 1; S \rightarrow 6$ 
\item $P \rightarrow 3; Q \rightarrow 3; R \rightarrow 6; S \rightarrow 5$
\item $P \rightarrow 4; Q \rightarrow 3; R \rightarrow 6; S \rightarrow 5$
\end{enumerate} 
\end{enumerate} 

 

