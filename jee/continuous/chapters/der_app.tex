\renewcommand{\theequation}{\theenumi}
\begin{enumerate}[label=\arabic*.,ref=\thesubsection.\theenumi]
\numberwithin{equation}{enumi}

\item The largest of $\cos(ln \theta)$ and $\ln(\cos \theta)$ if\\ $e^{\frac{\pi}{2}} < \theta < \frac{\pi}{2}$ is ..........

\item The function 
\begin{align*} 
y = 2x^2 - ln|x|
\end{align*}
is monotonically increasing for values of $x( \neq 0)$ satisfying the inequalities ...... and monotonically decreasing for values of x satisfying the inequalities .....

\item The set of all x for which ln(1 + x) $ \leq $ x is equal to ......

\item Let P be a variable point on the ellipse
\begin{align} 
\frac{x^2}{a^2} + \frac{y^2}{b^2} = 1
\end{align} 
with the foci $F_1$ and $F_2$. If A is the area of the triangle P$F_1$ $F_2$ then the maximum value of A is......

\item Let C be the curve 
\begin{align}
y^2 - 3xy + 2 = 0
\end{align}
If H is the set of points on the curve C where the tangent is horizontal and V is the set of the point on the curve C where the tangent is vertical then H = ....... and V = ....... 

\textbf{True/False:}

\item If x - r is the factor of the polynomial 
\begin{align*}
f(x)= a_nx^4 + .....+ a_0,
\end{align*} 
repeated m times ($1 < m < n$), then r is a root of $f'(x) = 0$ repeated m times.

\item For $0 < a < x$, the minimum value of the function $log_a x + log_x$ a is 2.

\textbf{MCQs with One Correct Answer:}

\item If a + b + c = 0,Then the quadratic equation 
\begin{align}
3ax^2 + 2bx + c = 0
\end{align} 
has 
\begin{enumerate}
\item at least one root in [0 , 1]
\item one root in [2,3] and the other in [-2,-1]
\item imaginary roots
\item none of these
\end{enumerate}

\item AB is a diameter of a circle and C is any point on the circumference of the circle.Then 
\begin{enumerate}
\item The area of $\Delta$ABC is maximum when it is isosceles
\item The area of $\Delta$ABC is maximum when it is isosceles
\item The perimeter of $\Delta$ABC is maximum when it is isosceles
\item none of these 
\end{enumerate}

\item The normal to the curve 
\begin{align*} 
x = a(\cos \theta + \theta \sin \theta)
\end{align*}
\begin{align*} 
y = a(\sin \theta - \theta \cos \theta) 
\end{align*} at any point $'\theta'$ is such that.
\begin{enumerate}
\item it makes a constant angle with x-axis
\item it passes through the origin 
\item it is at a constant distance from the origin 
\item none of these
\end{enumerate}

\item If 
\begin{align}
y = a lnx + bx^2 + x
\end{align} 
has its extremum values at x = -1 and x = 2 then 
\begin{enumerate}
\item a = 2, b = -1
\item a = 2, b = $\frac{-1}{2}$
\item a = -2, b = $\frac{1}{2}$
\item none of these
\end{enumerate}

\item Which one of the following curves cut the parabola 
\begin{align} 
y^2 = 4ax
\end{align} 
at right angles?
\begin{enumerate}
\item $x^2+ y^2 = a^2$
\item $y = e^{\frac{x}{2a}}$
\item y = ax
\item $x^2 = 4ay$
\end{enumerate}

\item The function defined by 
\begin{align*} 
f(x) = (x + 2)e^{-x} 
\end{align*} 
is
\begin{enumerate}
\item decreasing for all x
\item decreasing in ($-\infty$, -1) and increasing in (-1, $\infty$) 
\item increasing for all x
\item decreasing in (-1, $\infty$) and increasing in ($-\infty$, -1) 
\end{enumerate}

\item The function 
\begin{align*}
f(x) = \frac{ln(\pi + x)}{ln(e + x)}
\end{align*}
\begin{enumerate}
\item increasing on $(0, \infty,)$ 
\item decreasing on $(0, \infty)$
\item increasing on $(0, \frac{\pi}{e})$, decreasing on $(\frac{\pi}{e}, \infty)$
\item decreasing on $(0, \frac{\pi}{e}$), increasing on $(\frac{\pi}{e}, \infty)$
\end{enumerate}

\item On the interval [0, 1] the function $x^{25}(1 - x)^{75}$ takes its maximum value at the point
\begin{enumerate}
\item 0
\item $\frac{1}{4}$
\item $\frac{1}{2}$
\item $\frac{1}{3}$
\end{enumerate}

\item The slope of the tangent to a curve  
$y = f(x)$ at $[x, f(x)]$ is 2x + 1. If the curve passes through the point (1, 2), then the area bounded by the curve, the x-axis and line x = 1 is
\begin{enumerate}
\item $\frac{5}{6}$
\item $\frac{6}{5}$
\item $\frac{1}{6}$
\item 6
\end{enumerate}

\item If 
\begin{align*} 
f(x) = \frac{x}{\sin x} 
and 
\end{align*} 
\begin{align*}
g(x) = \frac{x}{\tan x}
\end{align*} 
where $0 < x \leq 1$, then in this interval
\begin{enumerate}
\item both $f(x)$ and $g(x)$ are increasing functions
\item both $f(x)$ and $g(x)$ are decreasing functions
\item $f(x)$ is an increasing function
\item $g(x)$ is an increasing function
\end{enumerate}

\item The function 
\begin{align*} 
f(x) = sin^4 x + cos^4x 
\end{align*} 
increasing if
\begin{enumerate}
\item $0 < x < \frac{\pi}{8}$
\item $\frac{\pi}{4} < x < \frac{3\pi}{8}$
\item $\frac{3\pi}{8} < x < \frac{5\pi}{8}$
\item $\frac{5\pi}{8} < x < \frac{3\pi}{4}$
\end{enumerate}

\item Consider the following statements in S and R

\textbf{S:} Both sin x and cos x are decreasing functions in the interval $(\frac{\pi}{2}, \pi)$

\textbf{R:} The differentiable function decrease in an interval (a, b), then its derivative also decreases in (a,b).

Which of the following is true
\begin{enumerate}
\item Both S and R are wrong
\item Both S and R are correct but R is not the correct explanation for S
\item S is correct and R is correct explanation for S 
\item S is Correct and R is wrong
\end{enumerate}

\item Let 
\begin{align*}
f(x) = \int e^x (x - 1)(x - 2)dx
\end{align*}
Then f decrease in the interval
\begin{enumerate}
\item $(-\infty, -2)$
\item $(-2, -1)$
\item (1, 2)
\item $(2, \infty)$
\end{enumerate}

\item If the normal ti the curve $y = f(x)$ at the point (3, 4) makes an angle $\frac{3\pi}{4}$ with the positive 
x-axis, then $f'(3)$ =
\begin{enumerate}
\item -1
\item $\frac{-3}{4}$
\item $\frac{4}{3}$
\item 1
\end{enumerate}

\item Let
\begin{align*} 
f(x) =
\left\lbrace\begin{cases} 
      |x|  &  0 < |x| \leq 2 \\
      1  & x = 0  \\
\end{cases}\right\rbrace 
\end{align*}
then at x = 0, f has
\begin{enumerate}
\item a local maximum
\item no local maximum
\item a local minimum
\item no extremum
\end{enumerate}

\item For all $x \in (0, 1)$
\begin{enumerate}
\item $e^x < 1 + x$
\item $log_e(1 + x) < x$
\item $\sin x > x$
\item $log_ex > x$
\end{enumerate}

\item If $f(x) = xe^{x(1 - x)}$ then f(x) is
\begin{enumerate}
\item increasing on [$\frac{-1}{2}, 1$]
\item decreasing on R
\item increasing on R
\item decreasing on [$\frac{-1}{2}, 1$]
\end{enumerate}

\item the triangle formed by the tangents of the curve 
\begin{align*} 
f(x) = x^2 + bx - b
\end{align*}
at the point (1, 1) and the coordinate axes, lies in the first quadrant. If its area is 2, Then the value of b is
\begin{enumerate}
\item -1
\item 3
\item -3
\item 1
\end{enumerate}

\item Let 
\begin{align} 
f(x) = (1 + b^2)x^2 + 2bx + 1
\end{align} 
and let m(b) be the minimum value of f(x). As b varies, the range of m(b) is
\begin{enumerate}
\item $[0, 1]$
\item $(0, \frac{1}{2}]$
\item $[\frac{1}{2} ,1]$
\item $(0, 1]$
\end{enumerate}

\item The length of the longest interval in which the function $3\sin x - 4\sin^3x$ is increasing is
\begin{enumerate}
\item $\frac{\pi}{3}$
\item $\frac{\pi}{2}$
\item $\frac{3\pi}{2}$
\item ${\pi}$
\end{enumerate}

\item The points on the curve 
\begin{align} 
y^3 + 3x^2 = 12y
\end{align} 
where the tangent is vertical, is 
\begin{enumerate}
\item $(\pm\frac{4}{\sqrt{3}}, -2)$
\item $(\pm\sqrt{\frac{11}{3}}, 1)$
\item (0, 0)
\item $(\pm\frac{4}{\sqrt{3}}, 2)$
\end{enumerate}

\item In [0, 1] Lagranges Mean Value theorem is NOT applicable to
\begin{enumerate}
\item 
f(x) =
\[\begin{cases} 
      \frac{1}{2} - x &  x < \frac{1}{2} \\
      (\frac{1}{2} - x)^2 & x \geq \frac{1}{2} \\
   \end{cases}\] 
\item 
f(x) =
\[\begin{cases} 
      \frac{\sin x}{x} &  x \neq 0 \\
      1 & x = 0\\
   \end{cases}\] 
\item $f(x) = x|x|$
\item $f(x) = |x|$
\end{enumerate}

\item Tangent is drawn to ellipse
\begin{align*}
\frac{x^2}{27} + y^2 = 1 at (3\sqrt{3}\cos \theta, \sin \theta)(where \theta \in (0, \frac{\pi}{2}))
\end{align*}
Then the value of $\theta$ such that sum of intercepts on axes made by this tangent is minimum is
\begin{enumerate}
\item $\frac{\pi}{3}$
\item $\frac{\pi}{6}$
\item $\frac{\pi}{8}$
\item $\frac{\pi}{4}$
\end{enumerate}

\item If 
\begin{align*}
f(x) = x^3 + bx^2 + cx + d 
\end{align*} 
and $0 < b^2 < c$, then in ($-\infty, \infty$)
\begin{enumerate}
\item $f(x)$ is strictly increasing function
\item $f(x)$ has local maxima
\item $f(x)$ is a strictly decreasing function 
\item $f(x)$ is bounded
\end{enumerate}

\item If 
\begin{align*}
f(x) = x^{\alpha} log x
\end{align*}
and $f(0) = 0$ then the value of $\alpha$ for which Rolle's theorem can be applied in [0, 1] is
\begin{enumerate}
\item -2
\item -1
\item 0
\item $\frac{1}{2}$
\end{enumerate}

\item If $P(x)$ is a polynomial of degree less than or equal to 2 then S is the set of all such polynomials so that p(0) = 0, P(1) = 1 and  $P'(x) > 0 \forall x \in [0,1]$ then 
\begin{enumerate}
\item S = $\Phi$
\item $S = ax + (1 - a)x^2 \forall a\in(0, 2)$
\item $S = ax + (1 - a)x^2 \forall a\in(0, \infty)$
\item $S = ax + (1 - a)x^2 \forall a\in(0, 1)$
\end{enumerate}
 
\item The tangent to the curve $y = e^x$ drawn at the point $(c, e^c)$ intersects the line joining the points 
$(c - 1, e^{c - 1})$ and $(c + 1, e^{c + 1})$ 
\begin{enumerate}
\item on the left of x = c
\item on the right of x = c
\item at no point 
\item at all points
\end{enumerate}

\item Consider the two curves
\begin{align*}
C_1: y^2 = 4x
\end{align*}
\begin{align*}
C_2: x^2 + y^2 - 6x + 1 = 0 
\end{align*}
then,
\begin{enumerate}
\item $C_1$ and $C_2$ touch only each other at one point
\item $C_1$ and $C_2$ touch each other exactly at two point
\item $C_1$ and $C_2$ intersect at exactly two points
\item $C_1$ and $C_2$ neither intersect nor touch each other
\end{enumerate}

\item The total number of local minima and local maxima of the function 
f(x)=
\[ \begin{cases} 
      (2 + x)^3 &  -3 < x \leq -1 \\
      x^{\frac{2}{3}}& -1 < x < 2\\
   \end{cases}
\]
is 
\begin{enumerate}
\item 0
\item 1
\item 2
\item 3
\end{enumerate}

\item Let the function g: $(-\infty, \infty) \to (\frac{-\pi}{2}, \frac{\pi}{2})$ be given by 
\begin{align*}
g(u) = 2\tan^{-1}(e^u) - \frac{\pi}{2}
\end{align*}
 Then g is
\begin{enumerate}
\item even and it is strictly increasing in (0,$\infty$)
\item odd and is strictly decreasing in ($-\infty,\infty$)
\item odd and is strictly increasing in ($-\infty,\infty$)
\item neither even nor odd but it is strictly increasing in ($-\infty,\infty$)  
\end{enumerate}

\item The least value of $a \in R$ for which $4\alpha x^2 + \frac{1}{x} \leq 1$, for all $x > 0$ is
\begin{enumerate}
\item $\frac{1}{64}$
\item $\frac{1}{32}$
\item $\frac{1}{27}$
\item $\frac{1}{25}$
\end{enumerate}

\item If $f: R \to R$ is a twice differentiable function such that $f''(x) > 0$ for all $x \in R$, and 
$f(\frac{1}{2}) = \frac{1}{2}$, $f(1) = 1$, then
\begin{enumerate}
\item $f'(1) \leq 0$
\item $0 < f'(1) \leq \frac{1}{2}$
\item $\frac{1}{2} < f'(1) \leq 1$
\item $f'(1) > 1$
\end{enumerate}

\textbf{MCQ's with One or More than One Correct Answer:}

\item Let 
\begin{align*}
P(x) = a_0 + a_1x^2 + a_2x^4.....a_nx^{2n}
\end{align*}
 be a polynomial equation in real variable x with 
$0 < a_0 < a_1 < a_2 < ..... < a_n$.The function $P(x)$ has
\begin{enumerate}
\item neither a maximum nor a minimum 
\item only one maximum 
\item only one minimum 
\item only one minimum and one maximum
\item none of these
\end{enumerate}

\item If the line ax + by + c = 0 is a normal to the curve xy = 1 then 
\begin{enumerate}
\item $a > 0, b > 0$
\item $a > 0, b < 0$
\item $a < 0, b > 0$
\item $a < 0, b < 0$
\item none of these
\end{enumerate}

\item The smallest positive root of the equation, $\tan x - x = 0$ lies in
\begin{enumerate}
\item $(0, \frac{\pi}{2})$
\item $(\frac{\pi}{2}, \pi)$
\item $(\pi, \frac{3\pi}{2})$
\item $(\frac{3\pi}{2}, 2\pi)$
\end{enumerate}

\item Let $f$ and $g$ be the increasing and decreasing functions respectively from $[0, \infty)$ to $[0, \infty)$. Let $h(x) = f(g(x))$. If $h(0) = 0$, then $h(x) - h(1)$ is
\begin{enumerate}
\item always zero
\item always negative
\item always positive
\item strictly increasing
\item None of these
\end{enumerate}

\item If 
\begin{align*}
f(x)=
\left\lbrace
\begin{array}{ll} 
      3x^2 + 12x - 1 &  -1 \leq x \leq 2\\
      37 - x & 2 < x \leq 3\\
\end{array}
\right\rbrace
\end{align*}
then:
\begin{enumerate}
\item $f(x)$ is increasing on [-1, 2]
\item $f(x)$ is continues on [-1, 3]
\item $f(2)$ does not exist
\item $f(x)$ has the maximum value at x = 2
\end{enumerate}

\item If 
\begin{align*}
h(x) = f(x) - (f(x))^2 + (f(x))^3
\end{align*}
for every real number x. Then 
\begin{enumerate}
\item h is increasing whenever f is increasing
\item h is increasing whenever f is decreasing
\item h is decreasing whenever f is decreasing
\item nothing can be said in general
\end{enumerate}

\item If 
\begin{align*}
f(x) = \frac{x^2 - 1}{x^2 + 1}
\end{align*}
for every real number then x then the minimum value of f
\begin{enumerate}
\item does not exist because f is unbounded 
\item is not attained even though f is bounded 
\item is equal to 1
\item is equal to -1
\end{enumerate}

\item The number of values of x where function 
\begin{align*}
f(x) = \cos x + \cos(\sqrt{2}x)    
\end{align*}
attains its maximum is 
\begin{enumerate}
\item 0
\item 1
\item 2
\item infinite
\end{enumerate}

\item The function 
\begin{align*}
f(x) = \int_{-1}^{x} t(e^t - 1)(t - 1)(t - 2)^{3}(t - 3)^{5}dt
\end{align*}
has a local minimum at x =
\begin{enumerate}
\item 0
\item 1
\item 2
\item 3
\end{enumerate}

\item $f(x)$ is a cubic polynomial with $f(2) = 18$ and $f(1) = -1$. Also $f(x)$ has local maxima at x = -1 and $f'(x)$ has local minima at x = 0, then
\begin{enumerate}
\item the distance between (-1, 2) and $(af(a))$, where x = a is the point of local minima is $2\sqrt{2}$
\item $f(x)$ is increasing for $x \in [1, 2\sqrt{5}]$
\item $f(x)$ has local minima at x = 1
\item the value of $f(0) = 15$
\end{enumerate}

\item Let 
f(x)=
\[ \begin{cases} 
      e^x &  1 < x \leq 1\\
      2 - e^{x - 1} & 1 < x \leq 2\\
      x - e & 2 < x \leq 3
   \end{cases}
\] 
and 
$g(x) = \int_{0}^{x}f(t)dt$, $x \in [1, 3]$ then $g(x)$ has 
\begin{enumerate}
\item local maxima at x = 1 + ln2 and local minima at x = e
\item local maxima at x = 1 and local minima at x = 2
\item no local maxima
\item no local minima
\end{enumerate}

\item For the function 
\begin{align*} 
f(x) = x \cos \frac{1}{x}, x \geq 1,
\end{align*}
\begin{enumerate}
\item for atleast one x in the interval 
\begin{align*}
[1, \infty), f(x + 2) - f(x) < 2
\end{align*}
\item $\lim_{x \to \infty} f'(x) = 1$
\item for all x in the interval 
\begin{align*}
[1, \infty), f(x + 2) - f(x) > 2
\end{align*}
\item $f'(x)$ is strictly decreasing for the interval $[1, \infty)$
\end{enumerate}

\item If 
\begin{align*}
f(x) = \int_{0}^{x} e^{x^2}(t - 2)(t - 3)dt
\end{align*} 
for all $x \in  (0, \infty)$, then 
\begin{enumerate}
\item f has local maxima at x = 2
\item f is decreasing on (2, 3)
\item there exist some $c \in (0, \infty)$, such that $f'(c) = 0$
\item f has a local minimum at x = 3
\end{enumerate}

\item A rectangular sheet of fixed perimeter with sides having length in the ratio 8 : 15 is converted into an open rectangular box by folding after removing squares of equal area from all four corners. If the total area of removed squares is 100, the resulting box has maximum volume. Then the lengths of the sides of the rectangular sheet are
\begin{enumerate}
\item 24
\item 32
\item 45
\item 60
\end{enumerate}

\item Let $f: (0, \infty) \to R$ be given by 
\begin{align*}
f(x)\int_{\frac{1}{x}}^{x} e^{-(t + \frac{1}{t})\frac{dt}{t}}
\end{align*}
then 
\begin{enumerate}
\item $f(x)$ is monotonically increasing on $[1, \infty)$
\item $f(x)$ is monotonically decreasing on(0, 1)
\item $f(x) + f(\frac{1}{x}) = 0$ for all $x \in (0, \infty)$
\item $f(2^x)$ is an odd function of x on R
\end{enumerate}

\item Let $f, g: [-1, 2] \to R$ be continuous functions which are twice differentiable on the interval (-1, 2). Let the values of f and g at the point -1, 0 and 2 be as given in the following table

\begin{tabular}{|c| c| c| c|} 
 \hline
  & x = -1 & x = 0 & x = 2 \\ 
 \hline
 f(x) & 3 & 6 & 0 \\ 
 \hline
 g(x) & 0 & 1 & -1 \\
 \hline
\end{tabular}\\

in each of the intervals (-1, 0) and (0, 2) the function $(f-3g)''$ never vanishes.Then the correct statement (s) is 
(are )
\begin{enumerate}
\item $f'(x) - 3g'(x) = 0$ has exactly three solutions in (-1, 0) $\cup$ (0, 2)
\item $f'(x) - 3g'(x) = 0$ has exactly one solutions in (-1, 0)
\item $f'(x) - 3g'(x) = 0$ has exactly one solutions in (0, 2)
\item $f'(x) - 3g'(x) = 0$ has exactly two solutions in (-1 ,0), exactly two solutions in (0, 2)
\end{enumerate}

\item Let $f: R \to R$ is a differentiable functions such that $f'(x) > 2f(x)$ for all $x \in R$ and $f(0) = 1$, then 
\begin{enumerate}
\item $f(x)$ is increasing in (0, $\infty$)
\item $f(x)$ is decreasing in (0, $\infty$)
\item $f(x) >  e^{2x}$ in (0, $\infty$)
\item $f'(x) > e^{2x}$ in (0, $\infty$)
\end{enumerate}

\item If  
$f(x) = 
\begin{vmatrix}
\cos(2x) & \cos(2x) & \sin (2x) \\ 
-\cos x & \cos x & -\sin x  \\
\sin x & \sin x& \cos x 
\end{vmatrix}$
Then 
\begin{enumerate}
\item $f'(x) = 0$ at exactly three points in ($-\pi, \pi$)
\item $f'(x) = 0$ at more than three points in ($-\pi, \pi$)
\item $f(x)$ attains its maximum at x = 0
\item $f(x)$ attains its minimum at x = 0
\end{enumerate}

\item Defines collections $\{E_1, E_2, E_3,.....\}$ of ellipses and $\{R_1, R_2, R_3,....\}$ of rectangles as follows:
\begin{align}
E_1: 
\frac{x^2}{9} + \frac{y^2}{4} = 1
\end{align}
$R_1$: Rectangle of largest area, with parallel sides to the axes inscribed in $E_1$\\
\begin{align}
E_n: Ellipse \frac{x^2}{a_n^2} + \frac{y^2}{b_n^2} = 1
\end{align}
of largest area inscribed in $R_{n - 1}$, $n > 1$;
$R_n$: Rectangle of largest area with sides parallel to the axes inscribed in $E_n$, $n > 1$.\\
Then which of the following options are correct?
\begin{enumerate}
\item The eccentricities of $E_18$ and $E_19$ are NOT equal
\item The length of latus rectum of $E_9$ is $\frac{1}{6}$
\item $\sum_{n = 1}^N$ (area of $R_n$) $<$ 24 for each positive integer N
\item The distance of a focus from the centre in $E_9$ is $\frac{\sqrt{5}}{32}$
\end{enumerate}

\item Let $f: R \to R$ be given by 
\begin{align*}
f(x) = (x - 1)(x - 2)(x - 5)
\end{align*}
 Define 
\begin{align*}
F(x) = \int_{0}^{x} f(t)dt, x > 0
\end{align*} 
Then when of the following options is/are correct?
\begin{enumerate}
\item F has a local maximum at x = 2
\item F has a local minimum at x = 1
\item F has two local maximum and one local minimum (0, $\infty$)
\item F(x) 0 for all x $\in$ (0, 5) 
\end{enumerate}

\item Let 
\begin{align*}
f(x) = \frac{\sin \pi x}{x^2}, x > 0
\end{align*}
Let $x_1 < x_2 < x_3...... < x_n < $.... be all the points of local maximum of f and $y_1 < y_2 < y_3 < $....$ < y_n <$....be all the points of local minimum of f. Then which of the following options is/are correct?
\begin{enumerate}
\item $x_{n + 1} - x_n > 2$
\item $x_n \in (2n, 2n + \frac{1}{2})$ for every n
\item $|x_n - y_n| > 1$ for every n
\item $x_1 < y_1$
\end{enumerate}

\textbf{E.Subjective Problems}

\item Prove that minimum value of
\begin{align*} 
\frac{(a + x)(b + x)}{(c + x)}, a, b > c, x > -c
\end{align*} 
is
\begin{align*} 
(\sqrt{a - c} + \sqrt{b - c})^2
\end{align*}
 
\item Let x and y be two real variables such that $x > 0$ and xy = 1. Find minimum value of x + y.

\item For all x in [0, 1], Let the second derivative $f"(x)$ of a function $f(x)$ exist and satisfy $|f"(x)| < 1$. If f(0) = f(1) Then show that $|f'(x)|<0$ for all x in [0, 1].

\item Use the function $f(x) = x^\frac{1}{x}$, $x > 0$ to determine the biggest of the two numbers $e^{\pi}$ and $\pi^e$.

\item If $f(x)$ and $g(x)$ are differentiable function for $0 \leq x \leq 1$ such that $f(0)=2$, $g(0) = 0$, $f(1) = 6$, $g(1) = 2$, then show that there exist c satisfying $0 < c < 1$ and $f'(c) = 2g'(c)$.

\item Find the shortest distance of the points(0, c) from the parabola $y = x^2$ where $0 \leq c \leq 5$.

\item If $ax^2 + \frac{b}{x} \geq c$ for all positive x where $a > 0$ and $b > 0$ show that $27ab^2 \geq 4c^3$.

\item Show that $1 + xln(x + \sqrt{x^2 + 1}) \geq \sqrt{1 + x^2}$ for all $x \geq 0$.

\item Find the coordinates of the points on the curve $y = \frac{x}{1 + x^2}$ where the tangent to the curve has the greatest slope.

\item Find all the tangents to the curve $y = \cos(x + y)$, $-2\pi \leq x \leq 2\pi$ that are parallel to the line 
x + 2y = 0.

\item Let $f(x) = \sin^3 x + \lambda \sin^2 x$, $\frac{-\pi}{2} < x < \frac{\pi}{2}$ find the intervals in which 
$\lambda$ should lie in order that $f(x)$ has exactly one minumum and one maximum.

\item Find the point on the curve 
\begin{align*}
4x^2 + a^2y^2 = 4a^2, 4 < a^2 < 8 
\end{align*}
that is farthest from the point (0, -2).

\item Investigate maxima and minima the function
\begin{align*}
f(x) = \int_{1}^{x}[2(t - 1)(t - 2)^2 + 3(t - 1)^2(t - 2)^2]dt
\end{align*}

\item Find all maxima and minima of the function $y = x(x - 1)^2, 0 \leq x \leq 2$ also determine the area bounded by the curve $y = x(x - 1)^2$ the y-axis and the line y = 2.

\item Show that $2\sin x + \tan x \geq 3x$ where $0 \leq x < \frac{\pi}{2}$.

\item A point P is given on the circumference of a circle of radius r. Chord QR is parallel to the tangent at P. Determine the maximum possible area of the triangle PQR.

\item A window of perimeter P is in the form of rectangle surrounded by a semicircle. The semi-cicular portion is fitted with coloured glass while the rectangular portion is fitted with the clear glass transmits three times as much light per square meter as the colour glass does. What is the ratio for the sides of rectangle so that the window transmits the maximum light?

\item A cubic $f(x)$ vanishes at x = 2 and has relative minimum and maximum at x = -1 and $x = \frac{1}{3}$ if 
\begin{align*}
\int_{-1}^{1}fdx = \frac{14}{3}
\end{align*}
find the cubic f(x).

\item What normal to the curve $y = x^2$ forms the shortest chord?

\item Find the equation of normal to the curve 
\begin{align*}
y = (1 + x)^y + \sin^{-1}(\sin^2 x)
\end{align*} 
at x=0. 

\item Let 
f(x) = 
\[ \begin{cases} 
      -x^3 + \frac{(b^3 - b^2 + b - 1)}{(b^2 + 3b + 2)} &  0 \leq x < 1\\
      2x - 3 & 1 \leq x \leq 3\\
      \end{cases}
\]
Find all possible real values of b such that $f(x)$ has the smallest value at x = 1.

\item The curve 
\begin{align}
y = ax^3 + bx^2 + cx + 5
\end{align} 
touches the x-axis at P(-2, 0) and cuts the y axis at a point Q where its gradient is 3 . Find a, b, c

\item The circle 
\begin{align}
x^2 + y^2 = 1
\end{align} 
cuts the x-axis at Pand Q another circle with centre at Q and variable radius intersects the first circle at R above the x-axis and the line segment PQ at S Find the maximum area of the triangle QSR.

\item Let (h, k) be a fixed point where $h > 0, k > 0$. A stright line passing through this point cuts the positive direction of the coordinate axis at points P and Q. Find the minimum area of triangle OPQ, O being the origin.

\item A curve $y = f(x)$ passes through the point p(1, 1). The normal to the curve at P is $a(y - 1) + (x - 1) = 0$. If the slope of the tangent at any point on the curve is proportional to the ordinate of the point, determine th equation of the curve also obtain the area bounded by the y-axis the curve and the normal to the curve at P.

\item Determine the points of maxima and minima of the function 
\begin{align*} 
f(x) = \frac{1}{8} ln x - bx + x^2
\end{align*}
x $>$ 0 where $b \geq 0$ is a constant.

\item Let 
f(x)=
\[ \begin{cases} 
      xe^{ax} &  x \leq 0 \\
      x + ax^2 - x^3& x > 0\\
      \end{cases}
\] 
where a is positive constant. Find  the interval in which $f'(x)$ is increasing.



\clearpage
\onecolumn

\textbf{Match the Following Questions:}

\item In this questions there are entries in column I and column II. Each entry in column I ia related to exactly one entry in column II.Write the correct letter from column II againest the entry number in column I  in your answer book.Let the functions defined in column I have domain $(-\frac{\pi}{2},\frac{\pi}{2})$
\begin{table}[ht!]
\centering
\begin{tabular}{c c} 
 \textbf{Column I} & \textbf{Column II}\\ [0.5ex] 
 (A) X + $\sin X$                                            &(p) increasing\\
 (B) $\sec x$                                                &(q) decreasing\\
                                                            &(r) neither increasing nor decreasing\\
\end{tabular}
\end{table}\\
By appropriately matching the matching the information given in the three columns of the following table.
Let f(x)=x+$log_ex-xlog_ex$ $x \in (0,\infty)$
\begin{enumerate}
    \item Column 1 contains information about zeros of f(x),f'(x) and f"(x).
    \item Column 2 contains information about the limiting behaviour of f(x),f'(x) and f"(x) at infity
    \item Column 3 contains information about the increasing/decreasing nature of f(x) and f'(x).
\end{enumerate}

\begin{center}
\begin{tabular}{llll}
Column-1 & Column-2 & Column-3\\
(I) f(x)=0 for some $x \in (1, e^2)$   &(i) $lim_{x \to \infty}f(x)=0$         &(P) f is increasing on (0, 1)\\
&&&\\
(II)f'(x) for $x \in (1, e)$           &(ii)$lim_{x \to \infty}f(x)=-\infty$   &(Q) f is increasing in $(e, e^2)$\\
&&&\\
(III)f'(x) for $x \in(0, 1)$           &(iii)$lim_{x \to \infty}f'(x)=-\infty$ &(R) f' is increasing in (0, 1)\\
&&&\\
(IV)f"(x) for $x \in (1, e)$           &(iv)$lim_{x \to \infty}f"(x)=0$        &(S) f' is decreasing in $(e, e^2)$\\
&&&\\
\end{tabular}
\end{center}

\item Which of the following option is the only correct combination
\begin{enumerate}
    \item (I)(i)(P)                \item (II)(ii)(Q)
    \item (III)(iii)(R)            \item (IV)(iv)(S)
\end{enumerate}
\item Which of the following option is the only correct combination
\begin{enumerate}
    \item (I)(ii)(R)
    \item (II)(iii)(S)
    \item (III)(iv)(P)
    \item (IV)(i)(S)
\end{enumerate}
\item Which of the following option is the only incorrect combination
\begin{enumerate}
    \item (I)(iii)(P)
    \item (II)(iv)(Q)
    \item (III)(i)(R)
    \item (II)(iii)(P)
\end{enumerate}

\clearpage
\twocolumn

\textbf{Comprehension Based Questions:}

\textbf{PASSAGE-1}

If a continuous function f defined on the real line R,assume positive and negative values in R then the equation f(x)=0 has a root in R.For example ,if it is known that a continuous function f on R is positive at some point and its minimum value is negative then the equation f(x)=0 has a root in R.consider f(x)=$ke^x-x$ for all real x where k is a real constant.

\item The line y = x meets $y = ke^x$ for $k \leq 0$ at
\begin{enumerate}
\item no point
\item one point
\item two points
\item more than two points
\end{enumerate}

\item The positive value of k for $ke^x - x = 0$ has only one root is
\begin{enumerate}
\item $\frac{1}{e}$
\item 1
\item e
\item $log_e2$
\end{enumerate}

\item for $k > 0$ the set of all values of k for which $ke^x - x = 0$ has two distinct roots
\begin{enumerate}
\item (0, $\frac{1}{e}$)
\item ($\frac{1}{e}$, 1)
\item ($\frac{1}{e}$, $\infty$)
\item (0, 1)
\end{enumerate}

\textbf{PASSAGE-2}

Let f(x) = $(1 + x)^2 \sin^2 x + x^2$ for all $x\ in IR$ and let g(x) = $\int_{1}^{x} (\frac{2(t - 1)}{t + 1} - lnt)$ f(t)dt for all $x \in (1,\infty)$

\item Consider the following statements:

\textbf{P:} There exist some x $\in$ R such that f(x) + 2x = $2(1 + x^2)$

\textbf{Q:} There exist some $x \in R$ such that 2f(x) + 1 = 2x(1 + x)
Then 
\begin{enumerate}
\item Both P and Q are true
\item P is true and Q is false
\item P is false and Q is true
\item Both P and Q are false
\end{enumerate}

\item Which of the following is true?
\begin{enumerate}
\item g is increasing on (0, $\infty$)
\item g is decreasing on (1, $\infty$)
\item g is increasing on (1, 2) and decreasing on(2, $\infty$)
\item g is decreasing on (1, 2) and increasing on (2, $\infty$)
\end{enumerate}

\textbf{PASSAGE-3}

Let f:[0, 1] $\to$ R be a function suppose the function f is twice differenciable, f(0) = f(1) = 0 and satisfies $f''(x) - 2f'(x) + f(x) \geq e^x$, x $\in$ [0, 1].

\item Which of the following is true for $0 < x < 1$?
\begin{enumerate}
\item $0 < f(x) < \infty$
\item $-\frac{1}{2} < f(x) < \frac{1}{2}$
\item $-\frac{1}{4} < f(x) < 1$
\item $-\infty < f(x) < 0$
\end{enumerate}

\item If function $e^{-x}$ f(x) assumes its minimum in the interval [0, 1] at x = $\frac{1}{4}$ which of the following is true?
\begin{enumerate}
\item $f'(x) < f(x), \frac{1}{4} < x < \frac{3}{4}$
\item $f'(x) > f(x),  0 < x < \frac{1}{4}$
\item $f'(x) < f(x),  0 < x < \frac{1}{4}$
\item $f'(x) < f(x),  \frac{3}{4} < x < 1$
\end{enumerate}

\textbf{Integer Value Correct Type:}

\item The maximum value of the function $f(x) = 2x^3 - 15x^2 + 36x - 48$ on the set A = \{$|x|x^2 + 20 \leq 9x$\}

\item Let p(x) be the polynomial of degree 4 having extremum at x = 1, 2 and $\lim_{x \to 0}(1 + \frac{p(x)}{x^2}) = 2$. then the value of p(2) is

\item Let f be a real valued differential function on R such that f(1) = 1. If the y-intercept of the tangent at any point P(x, y) on the curve y = f(x) is equal to the cube of the abscissa of P then find the value of f(-3).

\item Let f be a function defined on R such that $f'(x) = 2010(x - 2009)(x - 2010)^2(x - 2011)^3(x - 2012)^4$ forall 
$x \in R$. If g is a function defined on R with values in the interval (0, $\infty$) such that f(x) = ln(g(x)),for all 
$x \in R$. Then the number of points at which g has a local maximum is

\item Let $f: IR \to IR$ be defined as f(x) = $|x| + |x^2 - 1|$. The total number of points at which f attains either local maximum or local minimum is 

\item Let p(x) be a real polynomials of least degree which has a local maximum at x = 1 and local minimum at x = 3. If p(1) = 6 and p(3) = 2, Then $p'(0)$ is
 
\item A vertical line passing through the point(h, 0) intersects the ellipse $\frac{x^2}{4} + \frac{y^2}{3} = 1$ at the points P and Q. Let the tangent to the ellipse at P and Q meet at the point R. If $\Delta$(h) = area of the triangle PQR,$\Delta_1$ = $\max_{\frac{1}{2}\leq h \leq 1}$ $\Delta$(h) and $\Delta_2$ = $\min_{\frac{1}{2}\leq h \leq 1}$ $\Delta$(h), then $\frac{8}{\sqrt{5}}\Delta_1$ - $8\Delta_2$ =

\item The slope of the tangent to the curve $(y - x^5)^2 = x(1 + x^2)^2$ at the point (1, 3) is

\item A cylindrical container is to be made from certain solid material with the following constraints. It has a fixed inner volume of V $mm^3$ has a 2mm thick solid wall and is open at the top. The bottom of the container is a solid circular disc of thickness 2mm and is of radius equal to the outer radius of the container. If the volume of the material used to make the container is minimum when the inner radius of the container is 10mm, then the values of 
$\frac{v}{250\pi}$ is

\item Let $-1 \leq p \leq 1$. Show that the equation $4x^3 - 3x - p = 0$ has a unique root in the interval 
$[\frac{1}{2}, 1]$ and identify it.

\item Find a point on the curve $x^2 + 2y^2 = 6$ whose distance from the line x + y = 7, is minimum.

\item Using the relation $2(1 - \cos x) < x^2$, $x \neq  0$ or otherwise prove that $\sin(\tan x)\geq x \forall x \in [0, \frac{\pi}{4}]$.

\item If the function $f: [0, 4] \to R$ is differentiable then show that 
\begin{enumerate}
\item for $a, b \in(0, 4)$, $(f(4))^2 - (f(0))^2 = 8 f'(a)f(b)$
\item $\int_{0}^{4} f(t)dt = 2[\alpha f(\alpha^2) + \beta f(\beta^2)] \forall 0< \alpha, \beta < 2$
\end{enumerate}

\item If $p(1) = 0$ and $\frac{dP(x)}{dx} > P(x)$ for all x $\geq$ 1 then prove that $P(x) > 0$, for all x $>$ 1.

\item Using Rolle's theorem prove that there is at least one root for $(45^\frac{1}{100},46)$ of polynomial
\begin{align*}
P(x) = 51x^{101} - 2323(x)^{100} - 45x + 1035
\end{align*}

\item Prove that for $x \in [0, \frac{\pi}{2}]$, $\sin x + 2x \geq \frac{3x(x + 1)}{\pi}$ explain the identity if any used in the proof.

\item $|f(x_1) - f(x_2)| < (x_1 - x_2)^2$ for $x_1, x_2 \in R$. Find the equation of the tangent to the curve $y = f(x)$ at the point (1, 2).

\item If p(x) be the polynomial of degree 3 satisfying p(-1) = 10, p(1) = -6 and p(x) has maxima at x = -1
and $p'(x)$ has minima at x = 1. Find the distance between the local maxima and local minima of the curve.

\item For a twice differenciable function f(x), g(x) is defined as g(x) = $(f'(x)^2 + f''(x))$ f(x) on [a, e] If for $a < b < c < d < e$, f(a) = 0, f(b) = 2, f(c) = -1, f(d) = 2, f(e) = 0 then find the minimum number of zeros of g(x).

\item Let a + b = 4, where a $<$ 2 and let $g(x)$ be a differentiable function. If $\frac{dg}{dx} > 0$ for all x Prove that $\int_{0}^{a} g(x) dx + \int_{0}^{b}$ g(x) dx increases as (b - a) increasing.

\item Suppose $f(x)$ is a function statisfying th following conditions 
\begin{enumerate}
\item f(0) = 2, f(1) = 1
\item f has a minimum value at x = $\frac{5}{2}$ and 
\item for all x
\end{enumerate}
\[
f'(X) =
\begin{vmatrix}
2ax & 2ax - 1 & 2ax + b + 1  \\ 
b & b + 1 & -1  \\
2(ax + b) & 2ax + 2b + 1 & 2ax + b 
\end{vmatrix}
\]
where a, b be are some constants. Determine the constants a, b and the function f(x).

\item A  curve C has the property that if the tangent drawn at any point P on C the co-ordinate axes at A and B then P is the mid point of AB. The curve passes through the point (1, 1). Determine the equation of the curve.

\item Suppose
\begin{align*} 
p(x) = a_0 + a_1x + ..... + a_nx^n
\end{align*} 
If $|p(x)|\leq |e^{x - 1} - 1|$ for all x $\geq$ 0. Prove that 
\begin{align*}
|a_1 + 2a_2 + ....+na_n| \leq 1.
\end{align*}

\textbf{Section-B:}


\item The maximum distance from origin of a point on the curve 
\begin{align*}
x = a\sin t - b\sin\frac{at}{b}
\end{align*}
\begin{align*}
y = a\cos t - b\cos \frac{at}{b}, a, b > 0
\end{align*}
\begin{enumerate}
\item a - b
\item a + b
\item $\sqrt{a^2 + b^2}$
\item $\sqrt{a^2 - b^2}$
\end{enumerate}

\item If 2a + 3b + 6c = 0,($a, b, c \in R$) then the quadratic equation $ax^2 + bx + c = 0$ has 
\begin{enumerate}
\item at least one root in [0, 1]
\item at least one root in [2, 3]
\item at least one root in [4, 5]
\item none of these
\end{enumerate}

\item If the function $f(x) = 2x^3 - 9ax^2 + 12a^2 + 1$, where $a > 0$ attains its maximum and  minimum at p and q respectively such that $p^2 = q$ then a equals
\begin{enumerate}
\item $\frac{1}{2}$
\item 3
\item 1
\item 2
\end{enumerate}

\item A point on the parabola $y^2 = 18x$ at which the ordinate increase at twice the rate of the abscissa is \begin{enumerate}
\item ($\frac{9}{8}, \frac{9}{2}$)
\item (2, -4)
\item ($-\frac{9}{8}, \frac{9}{2}$)
\item (2, 4)
\end{enumerate}
    
\item A function y = f(x) has a second order derivative $f''(x) = 6(1 - x)$. If its graph passes through the point (2, 1) and at that point the tangent to the graph is y = 3x - 5 then the function is
\begin{enumerate}
\item $(x + 1)^2$
\item $(x - 1)^3$
\item $(x + 1)^3$
\item $(x - 1)^2$
\end{enumerate}

\item The normal to the curve x = $(1 + \cos \theta)$, y = $a\sin \theta$ at $\theta$ always passes through the fixed point 
\begin{enumerate}
\item (a, a)
\item (0, a)
\item (0, 0)
\item (a, 0)
\end{enumerate}

\item If 2a + 3b + 6c = 0 then at least one root of the equation $ax^2 + bx + c = 0$ lies in the interval
\begin{enumerate}
\item (1, 3)
\item (1, 2)
\item (2, 3)
\item (0, 1)
\end{enumerate}

\item Area of the greatest rectangle that can be inscribed in the ellipse $\frac{x^2}{a^2} + \frac{y^2}{b^2} = 1$ is 
\begin{enumerate}
\item 2ab
\item ab
\item $\sqrt{ab}$
\item $\frac{a}{b}$
\end{enumerate}

\item The normal to the curve x = a$(\cos \theta + \theta \sin \theta)$, y = a($\sin \theta - \theta \cos \theta$) at any point $'\theta'$ is such that
\begin{enumerate}
\item it passes through the origin 
\item it makes an angle $\frac{\pi}{2} + \theta$ with the x-axis
\item it passes through (a$\frac{\pi}{2}$, -a)
\item it is at constant distance from the origin
\end{enumerate}

\item A spherical iron ball 10 cm in radius is coated with a layer of ice of uniform thickness of ice is 5cm then the rate at which the thickness of ice decrease is 
\begin{enumerate}
\item $\frac{1}{36\pi} cm/min$
\item $\frac{1}{18\pi} cm/min$
\item $\frac{1}{54\pi} cm/min$
\item $\frac{5}{6\pi} cm/min$
\end{enumerate}

\item if the equation $a_nx^n + a_{n - 1}x^{n - 1}.......+a_1x$ = 0, $a_1 \neq 0,n\geq 2$, has a positive root 
x = $\alpha$ then the equation $na_n x^{n - 1} + (n - 1)a_{n - 1}x^{n - 2}+.......+ a_1$ = 0 has a positive root which is 
\begin{enumerate}
\item greater than $\alpha$
\item smaller than $\alpha$
\item greater than or equal to $\alpha$
\item equal to $\alpha$
\end{enumerate}

\item The function f(x) = $\frac{x}{2} + \frac{2}{x}$ has a local minimum at
\begin{enumerate}
\item x = 2
\item x = -2
\item x = 0
\item x = 1
\end{enumerate}

\item A triangular park is enclosed on two sides by a fence and on the third side by a straight river bank.The two sides having fence are of same length x,the maximum area enclosed by the park is 
\begin{enumerate}
\item $\frac{3}{2}x^2$
\item $\sqrt{\frac{x^3}{8}}$
\item $\frac{1}{2}$
\item $\pi x^2$
\end{enumerate}

\item A value of C for which conclusion of Mean Value Theorem holds for the function f(x) = $\log_e^x$ on the interval [1, 3] is 
\begin{enumerate}
\item $\log_3 e$
\item $\log_e 3$
\item $2\log_3 e$
\item $\frac{1}{2}\log_3 e$
\end{enumerate}

\item The function f(x) = $\tan^{-1}(\sin x + \cos x)$ is an increasing function in 
\begin{enumerate}
\item (0, $\frac{\pi}{2}$)
\item ($-\frac{\pi}{2}, \frac{\pi}{2}$)
\item ($\frac{\pi}{4}, \frac{\pi}{2}$)
\item ($-\frac{\pi}{2}, \frac{\pi}{4}$)
\end{enumerate}

\item If p and q are positive real numbers such that $p^2 + q^2 = 1$ then the maximum value of (p + q) is
\begin{enumerate}
\item $\frac{1}{2}$
\item $\frac{1}{\sqrt{2}}$
\item $\sqrt{2}$
\item 2
\end{enumerate}

\item Suppose the cubic $x^3 - px + q$ has three distinct real roots where $p > 0$ and $q > 0$ Then which one of the following holds?
\begin{enumerate}
\item the cubic has minimum at $\sqrt{\frac{p}{3}}$ and maximum at $-\sqrt{\frac{p}{3}}$
\item the cubic has minimum at $-\sqrt{\frac{p}{3}}$ and maximum at $\sqrt{\frac{p}{3}}$
\item the cubic has minimum at both  $\sqrt{\frac{p}{3}}$ and $-\sqrt{\frac{p}{3}}$
\item the cubic has maximum at $\sqrt{\frac{p}{3}}$ and  $-\sqrt{\frac{p}{3}}$
\end{enumerate}

\item How many real solutions does the equation $x^7 + 14x^5 + 16x^3 + 30x - 560 = 0$ have?
\begin{enumerate}
\item 7
\item 1
\item 3
\item 5
\end{enumerate}

\item Let f(x) = x$|x|$ and g(x) = $\sin x$.
\begin{enumerate}
\item Statement 1: gof is differentiable at x = 0 and its derivative is continuous at that point.
\item Statement 2: gof is twice differentiable at x = 0.
\end{enumerate}
\begin{enumerate}
\item statement-1 is true,statement-2 is true statement-2 is not correct explination for statement-1
\item statement-1 is true,statement-2 is false 
\item statement-1 is false and statement-2 is true 
\item statement-1 is true,statement-2 is true statement-2 is correct explination of statement-1
\end{enumerate}

\item Given P(x) = $x^4 + ax^3 + bx^2 + cx + d$ such that x = 0 is the only real root of $P'(x)=0$. If P(-1) $<$ P(1), Then in the interval [-1, 1]:
\begin{enumerate}
\item P(-1) is not a minimum but P(1) is the maximum of P
\item P(-1) is the minimum but P(1) is  not the maximum of P
\item Neither P(-1) is a minimum nor P(1) is the maximum of P
\item P(-1) is a minimum but P(1) is the maximum of P
\end{enumerate}

\item The equation of the tangent to the curve $y = x + \frac{4}{x^2}$ that is parallel to the x-axis is 
\begin{enumerate}
\item y = 1
\item y = 2
\item y = 3
\item y = 0
\end{enumerate}

\item Let $f: R \to R$ be defined by f(x) = 
\[ \begin{cases} 
      k - 2x &  if x \leq -1 \\
      2x + 3 & if x > -1\\
      \end{cases}
\] 
if f has a local minimum at x = -1 then a possible value of k is 
\begin{enumerate}
\item 0
\item $\frac{-1}{2}$
\item -1
\item 1
\end{enumerate}

\item Let $f: R \to R$ be a continuous function defined by f(x) = $\frac{1}{e^x + 2e^{-x}}$
\begin{enumerate}
\item Statement-1: f(c) = $\frac{1}{3}$ for some $c \in R$
\item Statement -2: $0 < f(x) \leq \frac{1}{2\sqrt{2}}$ for all $x \in R$
\end{enumerate}
\begin{enumerate}
\item statement-1 is true,statement-2 is true statement-2 is not correct explination for statement-1
\item statement-1 is true,statement-2 is false 
\item statement-1 is false and statement-2 is true 
\item statement-1 is true,statement-2 is true statement-2 is correct explination of statement-1
\end{enumerate}

\item The shortest distance between line y - x = 1 and curve x = $y^2$ is 
\begin{enumerate}
\item $\frac{3\sqrt{2}}{8}$
\item $\frac{8}{3\sqrt{2}}$
\item $\frac{4}{\sqrt{3}}$
\item $\frac{\sqrt{3}}{4}$
\end{enumerate}

\item For $x \in (0, \frac{5\pi}{2})$ define f(x) = $\int_{0}^{x}\sqrt{t} \sin t$dt then f has 
\begin{enumerate}
\item local minimum at $\pi and 2\pi$
\item local minimum at $\pi$ and local maximum at $2\pi$
\item local maximum at $\pi$ and local minimum at $2\pi$
\item local maximum at $\pi$ and $2\pi$
\end{enumerate}

\item A spherical balloon filled with 4500$\pi$ cubic meters of helium gas. If a leak in balloon causes the gas to escape at the rate of $72\pi$ cubic meters per minute then then rate at which the radius of balloon  decreases 49 minutes after the leakage began is:
\begin{enumerate}
\item $\frac{9}{7}$
\item $\frac{7}{9}$
\item $\frac{2}{9}$
\item $\frac{9}{2}$
\end{enumerate}

\item Let a, b $\in$ R be such that the function f given by f(x) = ln$|x|$ + b$x^2$ + ax, $x \neq 0$ has extreme values at x = -1 and at x = 2.
\begin{enumerate}
\item Statement-1: f has local maximum at x = -1 and at x = 2
\item Statement-2: a = $\frac{1}{2}$ and $b = -\frac{1}{4}$
\end{enumerate}
\begin{enumerate}
\item Statement-1 is false,Statement-2 is true
\item Statement-1 is true,statement-2 is true statement-2 is a correct explanation of statement -1
\item Statement-1 is true,statement-2 is true statement-2 is not a correct explanation of statement -1
\item Statement-1 is true and statement-2 is false
\end{enumerate}

\item A line is drawn through the point [1, 2] to meet the coordinates axes at P and Q such that it forms a triangle OPQ where O is the origin. If the area of the triangle OPQ is least then the slope of the line PQ is:
\begin{enumerate}
\item $\frac{-1}{4}$
\item -4
\item -2
\item $\frac{-1}{2}$
\end{enumerate}

\item The intercepts on the axis made by tangents to the curve y = $\int_0^x |t|$ dt, $x \in R$ which are parallel to the line y = 2x are equal to 
\begin{enumerate}
\item $\pm$1
\item $\pm$2
\item $\pm$3
\item $\pm$4
\end{enumerate}

\item If f and g are differentiable functions in [0, 1] satisfying f(0) = 2 = g(1), g(0) = 0 and f(1) = 6, Then for some c $\in$ [0, 1]
\begin{enumerate}
\item $f'(c) = g'(c)$
\item $f'(c) = 2g'(c)$
\item $2f'(c) = g'(c)$
\item $2f'(c) = 3g'(c)$
\end{enumerate}

\item Let f(x) be the polynomial of degree four having extreme values at x = 1 and x = 2. If $\lim_{x \to 0}$
[1 + $\frac{f(x)}{x^2}$] = 3, then f(2) is equal to:
\begin{enumerate}
\item 0
\item 4
\item -8
\item -4
\end{enumerate}

\item Consider:
\begin{align*}
f(x) = \tan^{-1}(\sqrt{\frac{1 + \sin x}{1 - \sin x}}) x \in (0, \frac{\pi}{2})
\end{align*} 
A normal to y = f(x) at x = $\frac{p}{6}$ also passes through the point:
\begin{enumerate}
\item ($\frac{\pi}{6}$, 0)
\item ($\frac{\pi}{4}$, 0)
\item (0, 0)
\item (0, $\frac{2\pi}{3}$)
\end{enumerate}

\item A wire of length 2 units is cut into two parts which are bent respectively to form a square of side = x units and a circle os radius = r units. If sum of the areas of the squares and the circle so formed is minimum then,
\begin{enumerate}
\item x = 2r
\item 2x = r
\item 2x = $(\pi + 4)$r
\item $(4 - \pi)$x = $\pi r$
\end{enumerate}

\item The function $f: R \to [\frac{-1}{2}, \frac{1}{2}]$ defined as f(x) = $\frac{x}{1 + x^2}$ is
\begin{enumerate}
\item neither injective nor surjective 
\item invertible
\item injective but not surjective 
\item surjective but not injective
\end{enumerate}

\item The Normal to the curve y(x - 2)(x - 3) = x + 6 at the point where the curve intersects the y-axis passes through the point:
\begin{enumerate}
\item ($\frac{1}{2}, \frac{1}{3}$)
\item ($-\frac{1}{2}, -\frac{1}{2}$)
\item ($\frac{1}{2}, \frac{1}{2}$)
\item ($\frac{1}{2}, -\frac{1}{3}$)
\end{enumerate}

\item Twenty meter of wire is available for fencing off a flower bed in the form of circular sector Then the maximum area of flower bed is:
\begin{enumerate}
\item 30
\item 12.5
\item 10
\item 25
\end{enumerate}

\item The eccentricity of an ellipse whose centre is at the origin is $\frac{1}{2}$ If one of its directices is x = -4 then the equation of normal to it at (1, $\frac{3}{2}$) is ;
\begin{enumerate}
\item x + 2y = 4
\item 2y - x = 2
\item 4x - 2y = 1
\item 4x + 2y = 7
\end{enumerate}

\item Let f(x) = $x^2 + \frac{1}{x^2}$ and g(x) = x - $\frac{1}{x}$, $x \in R-\{-1, 0, 1\}$. If h(x) = $\frac{f(x)}{g(x)}$ then local minimum value of h(x) is:
\begin{enumerate}
\item -3
\item -2$\sqrt{2}$
\item 2$\sqrt{2}$
\item 3
\end{enumerate}

\item If the curves $y^2 = 6x$, $9x^2 + by^2$ = 16 intersects each other at right angles then the value of b is:
\begin{enumerate}
\item $\frac{1}{2}$
\item 4
\item $\frac{9}{2}$
\item 6
\end{enumerate}

\item The maximum volume of the right circular cone having slant height 3m is
\begin{enumerate}
\item $6\pi$
\item $3\sqrt{3}\pi$
\item $\frac{4}{3}\pi$
\item $2\sqrt{3}\pi$
\end{enumerate}

\item If q denotes the acute angle between the curves, y = $10 - x^2$ and y = 2 + $x^2$ at a point of their intersection then $|\tan \theta|$ is equal to 
\begin{enumerate}
\item $\frac{4}{9}$
\item $\frac{8}{15}$
\item $\frac{7}{17}$
\item $\frac{8}{17}$
\end{enumerate}

\item If f(x) is a non-zero polynomial of degree four having local extreme points at x = -1, 0, 1 then the set 
S = \{x R: f(x) = f(0)\} contains exactly
\begin{enumerate}
\item four irrational numbers.
\item four rational numbers. 
\item two irrational and two rational number.
\item two irrational and one rational number.
\end{enumerate}

\item If the tangent to the curve y = x3 + ax - b at the point (1, -5) is perpendicular to the line -x + y + 4 = 0, then which one of the following points lie on the curve?
\begin{enumerate}
\item (-2, 1)
\item (-2, 2)
\item (2, -1)
\item (2, -2)
\end{enumerate}

\item Let S be the set of all values of x for which the tangent to the curves y = f(x) = $x^3 - x^2 - 2x$ at (x, y) is parallel to the line segment joining the points (1, f(a)) and (-1, f(-1)) then S is equal to: 
\begin{enumerate}
\item \{$\frac{1}{3}$, 1\}
\item \{$-\frac{1}{3}$, -1\}
\item \{$\frac{1}{3}$, -1\}
\item \{$-\frac{1}{3}$, 1\}
\end{enumerate}

\end{enumerate}

