% Copyright 2004 by Till Tantau <tantau@users.sourceforge.net>.
%
% In principle, this file can be redistributed and/or modified under
% the terms of the GNU Public License, version 2.
%
% However, this file is supposed to be a template to be modified
% for your own needs. For this reason, if you use this file as a
% template and not specifically distribute it as part of a another
% package/program, I grant the extra permission to freely copy and
% modify this file as you see fit and even to delete this copyright
% notice. 

\documentclass{beamer}
\usepackage{amsmath}
\usepackage{listings}
% There are many different themes available for Beamer. A comprehensive
% list with examples is given here:
% http://deic.uab.es/~iblanes/beamer_gallery/index_by_theme.html
% You can uncomment the themes below if you would like to use a different
% one:
%\usetheme{AnnArbor}
%\usetheme{Antibes}
%\usetheme{Bergen}
%\usetheme{Berkeley}
%\usetheme{Berlin}
%\usetheme{Boadilla}
%\usetheme{boxes}
%\usetheme{CambridgeUS}
%\usetheme{Copenhagen}
%\usetheme{Darmstadt}
\usetheme{default}
%\usetheme{Frankfurt}
%\usetheme{Goettingen}
%\usetheme{Hannover}
%\usetheme{Ilmenau}
%\usetheme{JuanLesPins}
%\usetheme{Luebeck}
%\usetheme{Madrid}
%\usetheme{Malmoe}
%\usetheme{Marburg}
%\usetheme{Montpellier}
%\usetheme{PaloAlto}
%\usetheme{Pittsburgh}
%\usetheme{Rochester}
%\usetheme{Singapore}
%\usetheme{Szeged}
%\usetheme{Warsaw}

\title{Optimization}

% A subtitle is optional and this may be deleted
\subtitle{Problem}

\author{Sai Ashish Somayajula\inst{1}}
% - Give the names in the same order as the appear in the paper.
% - Use the \inst{?} command only if the authors have different
%   affiliation.

\institute[Universities of Somewhere and Elsewhere] % (optional, but mostly needed)
{
  \inst{1}%
  EE16BTECH11043
  }
% - Use the \inst command only if there are several affiliations.
% - Keep it simple, no one is interested in your street address.


% - Either use conference name or its abbreviation.
% - Not really informative to the audience, more for people (including
%   yourself) who are reading the slides online


% This is only inserted into the PDF information catalog. Can be left
% out. 

% If you have a file called "university-logo-filename.xxx", where xxx
% is a graphic format that can be processed by latex or pdflatex,
% resp., then you can add a logo as follows:

% \pgfdeclareimage[height=0.5cm]{university-logo}{university-logo-filename}
% \logo{\pgfuseimage{university-logo}}

% Delete this, if you do not want the table of contents to pop up at
% the beginning of each subsection:


% Let's get started
\begin{document}

\begin{frame}
  \titlepage
\end{frame}



% Section and subsections will appear in the presentation overview
% and table of contents.

\begin{frame}{Problem}
  \begin{itemize}
  \item {
Find w such that,  \begin{align}
  L = {||y-Xw||}^2
 \end{align} is minimized.
\\where,
 \begin{align}
 X = 
\begin{pmatrix}
		1 & 0 \\ 1 & 1 \\ 1 & 2 
		\end{pmatrix}
 \end{align}
 \begin{align}
  y = \begin{pmatrix}
		6\\0\\0
		\end{pmatrix}
 \end{align}

  }
  \end{itemize}
  
\end{frame}

\begin{frame}{Least Squares Method}
    \begin{itemize}
  \item {
\begin{align}
 \widehat{w} =\min_w {||y-Xw||}^{2}
\\= {(X^{T}X)}^{-1}X^{T}y
 \end{align}  
  

  }
  \end{itemize}
 
\end{frame}

\begin{frame}{Gradient Descent}

\begin{itemize}
  \item {
 \begin{align}
w_{n+1} = w_{n} - \mu \frac{\partial L}{\partial w}
 \end{align}
 Till,
 \begin{align}
  w_{n+1} \approx w_{n}
 \end{align}

  }
 \end{itemize}
  
\end{frame}


\begin{frame}[fragile]{Python code}

\begin{lstlisting}[language=Python]
#Gradient Descent
#Least Squares
import numpy as np 
c = 2
cur_w = np.array([[1.0],[1.0]]) 
# The algorithm starts at w = [1,1]
gamma = 0.01 # step size multiplier
precision = 1e-6
previous_step_size = 1 
max_iters = 10000 
# maximum number of iterations
iters = 0 #iteration counter

X = np.array([[1.0,0.0],[1.0,1.0],[1.0,2.0]])
y = np.array([[6.0],[0.0],[0.0]])

\end{lstlisting}

\end{frame}

\begin{frame}[fragile]{Python code}

\begin{lstlisting}[language=Python]

df = lambda x, y, w:
	(-2.0*np.matmul(x.T,(y-np.matmul(x,w))))
		 
while (previous_step_size > precision)
		 & (iters < max_iters):
    prev_w = cur_w
    cur_w -= gamma * df(X,y,prev_w)
    previous_step_size = 
    		np.linalg.norm(cur_w - prev_w)
    iters+=1

\end{lstlisting}

\end{frame}


\begin{frame}[fragile]{Python code}

\begin{lstlisting}[language=Python]
# solution from Gradient Descent
Gradient_descent = curr_w

# equation for least squares
least_squares = np.matmul(
np.linalg.inv(np.matmul(X.T,X)),(np.matmul(X.T,y)))


\end{lstlisting}

\end{frame}

\begin{frame}[fragile]{Answer}

Answer:

\begin{itemize}
  \item {
  The minimum occurs at(gradient Descent)
\begin{align}
  w = \begin{pmatrix}
		1.0 \\ 0.84
		\end{pmatrix}
 \end{align}
The minimum occurs at(least squares)
 \begin{align}
  w = \begin{pmatrix}
		1.0 \\ 0.84
		\end{pmatrix}
 \end{align}

  }
 \end{itemize}

\end{frame}

\end{document}


