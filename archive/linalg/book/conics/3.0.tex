\renewcommand{\theequation}{\theenumi}
\begin{enumerate}[label=\arabic*.,ref=\thesubsection.\theenumi]
\numberwithin{equation}{enumi}
\item From \eqref{eq:quad_form}, the equation of a conic section is 
\begin{align}
%\label{eq:quad_form}
\vec{x}^T\vec{V}\vec{x}+2\vec{u}^T\vec{x}+f=0
\end{align}
%
for
%\begin{equation}
%Ax_1^2+Bx_1x_2+Cx_2^2+Dx_1+Ex_2+F = 0
%\label{eq:quadratic}
%\end{equation}
%
\begin{align}
\begin{vmatrix}
\vec{V}&\vec{u}
\\
\vec{u}^T&f
\end{vmatrix}
\ne 0
\end{align}
%Show that  \eqref{eq:quadratic} can be expressed as
%\begin{equation}
%\vec{x}^T\vec{V}\vec{x}+2\vec{u}^T\vec{x}+ f = 0
%\label{eq:quad_form}
%\end{equation}
%
%Find the matrix $\vec{V}$ and vector $\vec{u}$.
\item Show that 
\begin{align}
\frac{d\brak{\vec{u}^T\vec{x}}}{d\vec{x}} = \vec{u}
%+ \frac{d\vec{x}}{dx_1}V^T
%= 0
\end{align}
\item Show that 
\begin{align}
\frac{d\brak{\vec{x}^T\vec{V}\vec{x}}}{d\vec{x}} = 2\vec{V}^T\vec{x}
\end{align}
\item Show that 
\begin{align}
\frac{d\vec{x}}{dx_1} = \vec{m}
\end{align}
%
\numberwithin{equation}{enumi}
\item Find the {\em normal} vector to the curve in \eqref{eq:quad_form} at 
point $\vec{p}$.
\\
\solution Differentiating \eqref{eq:quad_form} with respect to 
$x_1$,
\begin{align}
\frac{d\brak{\vec{x}^T\vec{V}\vec{x}}}{d\vec{x}}\frac{d\vec{x}}{dx_1}+\frac{d\brak{\vec{u}^T\vec{x}}}{d\vec{x}}\frac{d\vec{x}}{dx_1}
&= 0
\\
\implies 2\vec{x}^T\vec{V}\vec{m}+2\vec{u}^T\vec{m}
& = 0  \because \brak{\frac{d\vec{x}}{dx_1} = \vec{m}}
\end{align}
Substituting  $\vec{x} = \vec{p}$ and simplifying
\begin{align}
\brak{ \vec{V}\vec{p}+\vec{u}}^T\vec{m} & = 0 
\\
\implies \vec{n} &= \vec{V}\vec{p}+\vec{u}
\end{align}
%
\renewcommand{\theequation}{\theenumi}
\item The {\em tangent} to the curve at $\vec{p}$ is given by 
\begin{align}
\vec{n}^T\brak{\vec{x}-\vec{p}} = 0
\end{align}
%\item Show that the tangent to \eqref{eq:quadratic} at a point $\vec{p}$ on 
%the curve is given by
%\begin{equation}
%\myvec{\vec{p}^T & 1}\myvec{V & \vec{u} \\ \vec{u}^T & F} \myvec{\vec{x} \\ 1} = 0
%\label{eq:tangent_one}
%\end{equation}
%
%Show that \eqref{eq:tangent_one} can be expressed as
This results in
\begin{equation}
\brak{\vec{p}^T\vec{V}+\vec{u}^T}\vec{x} + \vec{p}^T\vec{u} +f = 0
\label{eq:tangent}
\end{equation}

\item Let $\vec{P}$ be a rotation matrix and  $\vec{c}$ be a vector. Then 
\begin{align}
\vec{x} = \vec{P}\vec{y}+\vec{c}.
\label{eq:affine}
\end{align}
 is known as an {\em affine} transformation.
\item Classify the various conic sections based on $\eqref{eq:quad_form}$.
\\
\solution 
\begin{table}[!ht]
\begin{center}
\input{./conics/figs/conics.tex}
\end{center}
\caption{}
\label{table:conics}
\end{table}

%\item Show that the tangent to \eqref{eq:quadratic} at a point $\vec{p}$ on 
%the curve is given by
%\begin{equation}
%\myvec{\vec{p}^T & 1}\myvec{V & \vec{u} \\ \vec{u}^T & F} \myvec{\vec{x} \\ 1} = 0
%\label{eq:tangent_one}
%\end{equation}
%%
%\item Show that \eqref{eq:tangent_one} can be expressed as
%\begin{equation}
%\brak{\vec{p}^T\vec{V}+\vec{u}^T}\vec{x} + \vec{p}^T\vec{u} +f = 0
%\label{eq:tangent}
%\end{equation}
\end{enumerate}
