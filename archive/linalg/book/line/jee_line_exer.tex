\renewcommand{\theequation}{\theenumi}
\begin{enumerate}[label=\arabic*.,ref=\thesubsection.\theenumi]
\numberwithin{equation}{enumi}

    \item Find the area enclosed within the curve $\abs{x} + \abs{y} = 1 $.
    \item Find the equation of the line about which y=$10^x$ is the reflection of y=$\log_{10}x$.

    \item If $3a+2b+4c=0$, find the intersection of the  set of lines 
  \begin{align} 
    \myvec{a & b}\vec{x} + c &= 0.
    \end{align}
    \item Given the points $A=\myvec{0 \\ 4}$ and $B=\myvec{0 \\ -4}$, find the equation of the locus of the point $P=\myvec{x \\ y}$ such that $\abs{AP-BP} $=6.
    \item If a,b and c are in A.P, show that  the straight line 
\begin{align} 
    \myvec{a & b}\vec{x} + c &= 0
    \end{align}
will always pass through a fixed point and find its coordinates.
    \item Find the quadrant in which the orthocentre of the triangle formed by the lines 
\begin{align} 
    \myvec{1 & 1}\vec{x}  &= 1
\\
    \myvec{2 & 3}\vec{x}  &= 6
\\
    \myvec{4 & -1}\vec{x} + 4 &= 0
    \end{align} lies.
    \item Let the algebraic sum of the perpendicular distances from the points $\myvec{2 \\ 0}$,  $\myvec{0 \\ 2}$ and $\myvec{1 \\ 1}$ to a variable straight line be zero.  Show that  the line passes through a fixed point and find its  coordinates. 
    \item The vertices of a triangle are $\vec{A}=\myvec{-1 \\ -7},\vec{B}=\myvec{5 \\ 1}$ and $\vec{C}=\myvec{1 \\ 4}$. Find the equation of the bisector of $\angle ABC$.
    \item Verify if the straight line 
\begin{align} 
    \myvec{5 & 4}\vec{x} &= 0
    \end{align} passes through the point of intersection of the straight lines \begin{align} 
    \myvec{1 & 2}\vec{x} - 10 &= 0
    \end{align} and \begin{align} 
    \myvec{2 & 1}\vec{x} + 5 &= 0
    \end{align}
    \item Do the lines \begin{align} 
    \myvec{2 & 3}\vec{x} + 19 &= 0
    \end{align} and \begin{align} 
    \myvec{9 & 6}\vec{x} - 17 &= 0
    \end{align} cut the coordinate axes in concyclic points?
    \item The points $\myvec{-a\\b}$,\myvec{0\\0},\myvec{a\\b} and \myvec{a^2\\ab} are:
    \begin{enumerate}
    \item  Collinear
    \item  Vertices of a parallelogram
    \item  Vertices of a rectangle
    \item  None of these
    \end{enumerate}
    \item The point $\myvec{4 \\ 1}$ undergoes the following three transformations successively
    (i) Reflection about the line \begin{align} 
    \myvec{-1 & 1}\vec{x} &= 0
    \end{align}
    (ii)Translation through a distance 2 units along the positive direction of x-axis
    (iii) Rotation through an angle $\frac{\pi}{4}$ about the origin in counter clockwise direction.
    Find  the final position of the point.
\item The straight lines \begin{align} 
    \myvec{1 & 1}\vec{x} &= 0
\\
    \myvec{3 & 1}\vec{x} - 4 &= 0
\\
    \myvec{1 & 3}\vec{x} - 4 &= 0
    \end{align}form a triangle which is
    \begin{enumerate}
     \item isosceles
     \item equilateral
     \item right angled
     \item none of these
     \end{enumerate}
    \item If $\vec{P}=\myvec{1\\0}$, $\vec{Q}=\myvec{-1\\0}$ and $\vec{R}=\myvec{2\\0}$ are three given points, then the locus of the point $\vec{S}$ satisfying the relation $SQ^2+SR^2=2SP^2$, is
    \begin{enumerate}
     \item  a straight line parallel to X-axis
     \item a circle passing through the origin
     \item a circle with centre at the origin
     \item  a straight line parallel to Y-axis.
     \end{enumerate}
    \item Line L has intercepts a and b on the coordinate axes.When the axes are rotated through a given angle, keeping the origin fixed, the same line L has intercepts p and q, then
    \begin{enumerate}
     \item  $a^2+b^2=p^2+q^2$
     \item  $\frac{1}{a^2}+\frac{1}{b^2}=\frac{1}{p^2}+\frac{1}{q^2}$
     \item  $a^2+p^2=b^2+q^2$
     \item  $\frac{1}{a^2}+\frac{1}{p^2}=\frac{1}{b^2}+\frac{1}{q^2}$
     \end{enumerate}
    \item If the sum of the distances of a point from two perpendicular lines in a plane is 1, then its locus is
    \begin{enumerate}
     \item  Square
     \item  Circle
     \item  Straight line
     \item  Two intersecting lines
     \end{enumerate}
    \item The locus of a variable point whose distances from $\myvec{-2\\0}$ is $\frac{2}{3}$ times its distance from the line $x=-\frac{9}{2}$ is
    \begin{enumerate}
     \item  Ellipse
     \item  Parabola
     \item  Hyperbola
     \item  None of these
    \end{enumerate}
    \item The equations of a pair of opposite sides of a parallelogram are  
\begin{align} 
x^2-5x+6 = 0
\\
y^2-6y+5 = 0
\end{align}
Find  the equations of its diagonals. 
    \item Find the orthocentre of the triangle formed by the lines 
\begin{align}
    \vec {x}^T\myvec{0 & 1 \\ 1 & 0} \vec{x} &=0
    \end{align} 
and 
\begin{align}
\myvec{1 & 1} \vec x =1
    \end{align} 
    \item Let PQR be an isosceles triangle, right angled at  $\vec{P}=\myvec{2\\1}$. If the equation of the line QR is \begin{align}\myvec{2 & 1}\vec{x} = 3, \end{align} then find the equation representing the pair of lines PQ and PR is
    \item If $x_1,x_2,x_3$ as well as $y_1,y_2,y_3$ are in G.P with the same common ratio, then the points\myvec{x_1\\y_1},\myvec{x_2\\y_2} and \myvec{x_3\\y_3}
    \begin{enumerate}
     \item  lie on a straight line
     \item  lie on a ellipse
    \item  lie on a circle
    \item  are the vertices of a triangle
    \end{enumerate}
    \item Let PS be the median of the triangle with vertices $\vec{P}=\myvec{2\\2}, \vec{Q}=\myvec{6\\-1}$ and $\vec{R}=\myvec{7\\3}$. Find the equation of the line passing through $\myvec{1\\-1}$ and parallel to PS.
    \item Find the incentre of the triangle with vertices $\myvec{1\\\sqrt{3}},\myvec{0\\0}$ and $\myvec{2\\0}$. 
    \item The number of integer values of $m$, for which the x coordinate of the point of intersection of the lines \myvec{3\\4}$\vec {x}=9$ and \myvec{-m\\1}$\vec {x} -1=0$  is also an integer, is
    \begin{enumerate}
     \item  2
     \item  0
     \item  4
     \item  1
     \end{enumerate}
    \item Find the area of the parallelogram formed by the lines \myvec{-m\\1}$\vec {x}$=0, \myvec{-m\\1}$\vec {x}$+1=0 ,\myvec{-n\\1}$\vec {x}$=0 and \myvec{-n\\1}$\vec {x}$+ 1 =0.
    \item Let $0 <\alpha< \frac{\pi}{2}$  be a fixed angle. If
    $\vec{P}=\myvec{\cos\theta \\ \sin \theta}$ and $\vec{Q}=\myvec{\cos\brak{\alpha-\theta}\\ \sin\brak{\alpha-\theta}}$
    then the $\vec{Q}$ is obtained from $\vec{ P}$ by
    \begin{enumerate}
     \item  clockwise rotation around origin through an angle$\alpha$
     \item  anticlockwise rotation around origin through an angle$\alpha$
     \item  reflection in the line through origin with slope $\tan \alpha$
     \item  reflection in the line through origin with slope $\tan \frac{\alpha}{2}$
     \end{enumerate}
    \item Let $\vec{P}=\myvec{-1\\0} \vec{Q}=\myvec{0\\0}$ and $\vec{R}=\myvec{3\\3\sqrt{3}}$ be three points. Then find the equation of the bisector of the angle PQR.
    \item A straight line through the origin $\vec{O}$ meets the parallel lines \begin{align} \myvec{4 & 2} \vec {x} &= 9\end{align} and \begin{align} \myvec{2 & 1} \vec {x} + 6 &= 0\end{align} at points $\vec{P}$ and $\vec{Q}$ respectively. Find the ratio in which $\vec{O}$ divides the segment PQ. 
    \item The number of integral points (integral points means both the coordinate should be integer) exactly in the interior of the triangle with the vertices \myvec{0\\0} , \myvec{0\\21} and  \myvec{21\\0} is 
    \begin{enumerate}
     \item 133
     \item 190
     \item 233
     \item 105
     \end{enumerate}
    \item Find the orthocentre of a triangle with vertices   \myvec{0\\0},\myvec{3\\4},\myvec{4\\0} 
    \item Find the area of the triangle formed by the line \begin{align} \myvec{1 & 1} \vec {x} &= 3\end{align} and angle bisectors of the pair of straight lines \begin{align}\vec{x}^T \myvec{1 & 0 \\ 0 & -1} \vec {x} + \myvec{0 & 2} \vec {x} &= 1.\end{align}
    \item Let $\vec{O}= \myvec{0\\0}, \vec{P}=\myvec{3\\4}, \vec{Q}= \myvec{6\\0}$  be the vertices of the triangle OPQ. The point $\vec{R}$ inside the triangle OPQ is such that the triangles OPR,PQR, OQR are of equal area. Find the coordinates of $\vec{R}$.
    \item A straight line L through the point \myvec{3\\-2} is inclined at an angle of $60\degree$ to the line\begin{align} \myvec{\sqrt{3} & 1} \vec {x} &= 1.\end{align} If L also intersects the x-axis, then find the equation of L.
%

    \item Three lines \begin{align}\myvec{p & q} \vec {x} + r &= 0,\end{align} \begin{align}\myvec{q & r} \vec {x} + p &= 0\end{align} and \begin{align}\myvec{r & p} \vec {x} + q &= 0\end{align} are concurrent if
    \begin{enumerate}
     \item  $p+q+r=0$
     \item  $p^2+q^2+r^2=qr+rp+pq$
     \item  $p^3+q^3+r^3=3pqr$
     \item  none of these
     \end{enumerate}
    \item The points \myvec{0\\\frac{8}{3}},\myvec{1\\3} and \myvec{82\\30} are vertices of 
    \begin{enumerate}
     \item  an obtuse angled triangle
     \item  an acute angled triangle
     \item  a right angled triangle
     \item  none of these
     \end{enumerate}
    \item All points lying inside the triangle formed by the points \myvec{1\\3}, \myvec{5\\0} and \myvec{-1\\2} satisfy
    \begin{enumerate}
     \item  $\myvec{3 & 2} \vec {x} \geq 0$
     \item  $\myvec{2 & 1} \vec {x}-13\geq 0$
     \item  $\myvec{2 & -3} \vec {x}-12\leq 0$
     \item  $\myvec{-2 & 1} \vec {x} \geq 0$
     \item none of these
     \end{enumerate}
    \item A vector $\vec{a} = \myvec{2p \\ 1}$  with respect to a rectangular cartesian system. The system is rotated through a certain angle about the origin in the counter clockwise sense. If, with respect to the new system,  $\vec {a} = \myvec{p+1\\1}$, then
    \begin{enumerate}
     \item  p=0
     \item  p=1 or p=$-\frac{1}{3}$
     \item  p=-1 or p=$\frac{1}{3}$
     \item  p=1 or p=-1
     \item none of these
     \end{enumerate}
    \item If $\vec{P} = \myvec{1\\2}, \vec{Q} = \myvec{4\\6}, \vec{R} = \myvec{5\\7}$ and $\vec{S} = \myvec{a\\b}$ are the vertices of a parallelogram PQRS, find $a$ and $b$.
    \item The diagonals of a parallelogram PQRS are along the lines \begin{align}\myvec{1 & 3} \vec {x} &= 4\end{align} and \begin{align}\myvec{6 & -2} \vec {x} &= 7\end{align}. Then PQRS must be a
    \begin{enumerate}
     \item  rectangle
     \item  square
     \item  cyclic quadrilateral
     \item  rhombus
     \end{enumerate}
    \item If the vertices $\vec{P}, \vec{Q} , \vec{R}$ of a triangle PQR are rational points, which of the following points of the triangle PQR is (are) always rational point(s)?
    \begin{enumerate}
     \item  centroid
     \item  incentre
     \item  circumcentre
     \item  orthocentre
     \end{enumerate}
    \item Let $L_1$ be a straight line passing through the origin and $L_2$ be the straight line \begin{align}\myvec{1 & 1} \vec {x} &= 1.\end{align} If the intercepts made by the circle  \begin{align}\vec{x}^T\vec{x} + \myvec{-1 & 3} \vec {x} &= 0\end{align} on $L_1$ and $L_2$ are equal, then which of the following equations can represent $L_1$?
    \begin{enumerate}
     \item  $\myvec{1 & 1} \vec {x} = 0$
     \item  $\myvec{1 & -1} \vec {x} = 0$
     \item  $\myvec{1 & 7} \vec {x} = 0$
     \item  $\myvec{1 & -7} \vec {x} = 0$
     \end{enumerate}
    \item For $a > b > c > 0$, the distance between $\myvec{1\\1}$ and the point of intersection of the lines $\myvec{a & b}\vec {x}+c=0$ and $\myvec{b & a} \vec {x}+c=0$ is less than 2$\sqrt2$. Then 
    \begin{enumerate}
     \item  a+b-c\textgreater 0
     \item  a-b+c\textless 0
     \item  a-b+c\textgreater 0
     \item  a+b-c\textless 0
    \end{enumerate}
    \item  A straight line segment of length $l$, moves with its ends on two mutually perpendicular lines. Find the locus of the points which divides the line segment in the ratio 1:2.
    \item The area of triangle is 5. Two of its vertices are $\vec{A} = \myvec{2\\1}, \vec{B} = \myvec{3\\-2}$. The third vertex $\vec{C}$ lies on $\myvec{-1 & 1}\vec {x} = 3$. Find $\vec{C}$.
    \item One side of a rectangle lies along the line \begin{align}\myvec{4 & 7} \vec {x} + 5 &= 0.\end{align} Two of its vertices are \myvec{-3\\1} and \myvec{1\\1}. Find the equations of the other three sides.
    \item  Two vertices of a triangle are \myvec{5\\-1} and \myvec{2\\-3}. If  the orthocentre of the triangle is the origin, find the coordinates of the third point.
     \item  Find the equation of the line which bisects the obtuse angle between the lines \begin{align}\myvec{1 & -2} \vec {x} + 4 &= 0\end{align} and \begin{align}\myvec{4 & -3} \vec {x} -2 &= 0.\end{align}
    \item A straight line L is perpendicular to the line \begin{align}\myvec{5 & -1} \vec {x} &= 1.\end{align} The area of the triangle formed by the line L and the coordinate axes is 5. Find the equation of the line L.
    \item The end $\vec{A},\vec{B}$ of a straight line segment of constant length c slide upon the fixed rectangular axis OX,OY respectively. If the rectangle OAPB be completed, then show that the locus of the foot of the perpendicular drawn from $\vec{P}$ to AB is $x^{2/3}+y^{2/3}=c^{2/3}$.
    \item The vertices of a triangle are $\myvec{a t_1t_2 \\ a(t_1+t_2)},\myvec{a t_2t_3 \\ a(t_2+t_3)},\myvec{a t_3t_1\\ a(t_3+t_1)}$. Find the orthocentre of the triangle.
    \item The coordinates of $\vec{A}, \vec{B}, \vec{C}$ are $\myvec{6\\3},\myvec{-3\\5},\myvec{4\\-2}$ respectively, and $\vec{P}$ is any point $\vec{x}$. Show that the ratio of the area of the triangles $\triangle PBC$ and $\triangle ABC$ is $ \frac{\abs{\myvec{1 & 1}\vec {x}-2}}{7}$
    \item Two equal sides of an isosceles triangles are given by the equations \begin{align}\myvec{7 & -1} \vec {x} + 3 &= 0 \end{align} and \begin{align}\myvec{1 & 1} \vec {x} - 3 &= 0 \end{align} and its third side passes through the point \myvec{1\\10}. Determine the equation of third side.
    \item One of the diameters of the circle circumscribing the rectangle ABCD is \begin{align}\myvec{-1 & 4} \vec {x}  &= 7. \end{align} If A and B are the points \myvec{-3\\4} and \myvec{5\\4} respectively, then find the area of the rectangle.
    \item Two sides of a rhombus ABCD are parallel to the lines \begin{align}\myvec{-1 & 1} \vec {x}  &= 2 \end{align} and \begin{align}\myvec{-7 & 1} \vec {x}  &= 3 \end{align}. If the diagonals of the rhombus intersect at the point \myvec{1 \\ 2} and  vertex $\vec{A}$ is on the y axis. Find possible coordinates of $\vec{A}$.
    \item Lines \begin{align}L_1\equiv\myvec{a & b} \vec {x} + c &= 0 \end{align} \begin{align}L_2\equiv\myvec{l & m} \vec {x} + n &= 0 \end{align} intersect at the point $\vec{P}$ and make an angle $\theta$ with each other. Find the equation of a line L different from $L_2$ which passes through $\vec{P}$ and makes the same angle $\theta$ with $L_1$.
    \item Let ABC be a traingle with AB=AC. If $\vec{D}$ is the mid point of BC, $\vec{E}$ is the foot of the perpendicular drawn from $\vec{D}$ to AC and $\vec{F}$ the mid-point of DE, Prove that AF perpendicular to BE.
    \item Straight lines\begin{align}\myvec{3 & 4} \vec {x}  &= 5 \end{align} and \begin{align}\myvec{4 & -3} \vec {x}  &= 15 \end{align} intersect at the point $\vec{A}$. Points $\vec{B}$ and $\vec{C}$ are chosen on these two lines such that AB=AC. Determine the possible equations of the lines BC passing through the point\myvec{1\\2}
    \item A line cuts the x-axis at $\vec{A}= \myvec{7\\0}$ and the y-axis at $\vec{B}=\myvec{0\\-5}$. A variable line PQ is draw perpendicular to AB cutting the x-axis in $\vec{P}$ and the y-axis in $\vec{Q}$. If AQ and BP intersect at $\vec{R}$, find the locus of $\vec{R}$.
    \item Find the equation of the line passing through the point \myvec{2\\3} and making an intercept of  length 2 units between the lines $\myvec{2 & 1}\vec {x}$=3 and $\myvec{2 & 1} \vec {x}$=5.
%\includegraphics[scale=1.5]{sample}
    \item Show that all chords of the curve  \begin{align}\vec{x}^T\myvec{3 & 0 \\ 0 & -1} \vec {x} + \myvec{-2 & 4} \vec {x} &= 0 \end{align} which subtend a right angle at the origin, pass through a fixed point. Find the coordinates of the point.
    \item Determine all values of $\alpha$ for which the point $\myvec{\alpha\\ \alpha^2}$ lies inside the triangle formed by the lines 
    \begin{align}\myvec{2 & 3} \vec {x} -1 &= 0  \\ \myvec{1 & 2} \vec {x} -3 &= 0 \\ \myvec{5 & -6} \vec {x} -1 &= 0 \end{align}
    \item The tangent at a point $\vec{P}_1$ (other than the origin) on the curve 
    \begin{align}y = x^3 \end{align} meets the curve again at $\vec{P}_2$. The tangent at $\vec{P}_2$ meets the curve at $\vec{P}_3$ and so on. Show that the abscissae of $\vec{P}_1,\vec{P}_2,\vec{P}_3 \dots \vec{P}_n$, form a G.P. Also find the ratio $\frac{\text{area}(\triangle P_1,P_2,P_3)}{\text{area}(\triangle P_2,P_3,P_4)}$.
    \item A line through $\vec{A}=\myvec{-5\\-4}$ meets the line $\myvec{1 & 3}\vec {x}+2=0,\myvec{2 &1} \vec {x}$ +4=0 and $\myvec{1 & -1}\vec {x}$ -5=0 at the points $\vec{B},\vec{C}$ and $\vec{D}$ respectively. If
\begin{align}
\brak{\frac{15}{AB}}^2+\brak{\frac{10}{AC}}^2 = \brak{\frac{6}{AD}}^2,
\end{align}
find the equation of the line.
    \item A rectangle PQRS has its side PQ parallel to the line \begin{align}\myvec{-m & 1} \vec {x}  &= 0 \end{align} and vertices $\vec{P},\vec{Q}$ and $\vec{S}$ on the lines 
\begin{align}
\myvec{0 & 1} \vec {x}  &= a 
\\
\myvec{1 & 0} \vec {x}  &= b 
\\
\myvec{1 & 0} \vec {x}  &= -b 
\end{align} respectively. Find the locus of vertex $\vec{R}$.
    \item Using coordinate geometry, prove that the three altitudes of any triangle are concurrent.
    \item For points $\vec{P}=\myvec{x_1\\y_1}$ and $\vec{Q}=\myvec{x_2\\y_2}$ of the coordinate plane, a new distance $d(\vec{P},\vec{Q})=\vert x_1-x_2\vert+ \vert y_1-y_2\vert$. Let $\vec{O}=\myvec{0\\0}$ and $\vec{A}=\myvec{3\\2}$. Prove that the set of points in the first quadrant which are equidistant (with respect to the new distance) from $\vec{O}$ and $\vec{A}$ consists of the union of a line segment of finite length and an infinite ray. Sketch this set in a labelled diagram.
    \item Let ABC and PQR be any two triangles in the same plane. Assume that the perpendiculars from the points $\vec{A},\vec{B},\vec{C}$ to the sides QR,RP,PQ respectively are concurrent. Using vector methods or otherwise, prove that the perpendiculars from $\vec{P},\vec{Q},\vec{R}$ to BC,CA,AB respectively are also concurrent.
    \item Let a,b,c be real numbers with $a^2+b^2+c^2=1$. show that the equation
    \begin{align}\mydet{\myvec{a & -b}\vec {x} - c & \myvec{b & a}\vec {x} & \myvec{c & 0}\vec {x} + a \\ \myvec{b & a}\vec {x} &  \myvec{-a & b}\vec {x} - c & \myvec{0 & c}\vec {x} + b \\ \myvec{c & 0}\vec {x} + a & \myvec{0 & c}\vec {x} + b & \myvec{-a & -b}\vec {x} + c} = 0 \end{align} represents a straight line.
    \item A straight line L through the origin meets the line and \begin{align}\myvec{1 & 1} \vec {x} &= 3 \end{align} at $\vec{P}$ and $\vec{Q}$ respectively. Through $\vec{P}$,  straight lines $L_1$ and $L_2$ are drawn parallel to \begin{align}\myvec{2 & -1} \vec {x} &= 5 \\ \myvec{3 & 1} \vec {x}  &= 5 \end{align} respectively. Lines $L_1$ and $L_2$ intersect at that the locus of $\vec{R}$, as L varies, is a straight line.
    \item A straight line L with negative slope passes through point \myvec{8\\2} and cuts the positive coordinates are $\vec{P}$ and $\vec{Q}$. Find the absolute minimum value of OP varies, where $\vec{O}$ is origin.
    \item The area of the triangle formed by the intersection of a line parallel to the x-axis and passing through $\vec{P}= \myvec{h\\k}$ with the lines 
\begin{align}
\myvec{1 & -1} \vec {x} &= 0 
\\
\myvec{1 & 1} \vec {x} &= 2 
\end{align} is 4$h^2$. Find the locus of the point.
    \item Lines 
\begin{align}L_1: \myvec{-1 & 1} \vec {x} &=0
\\
L_2: \myvec{2 & 1} \vec {x} &=0
\end{align} 
intersect the line 
\begin{align}
L_3: \myvec{0 & 1} \vec {x} + 2 &=0
\end{align} at $\vec{P}$ and $\vec{Q}$ respectively. 
The bisector of the acute angle between $L_1$ and $L_2$ intersects $L_3$ at $\vec{R}$.
\begin{enumerate}[label=Statement-\arabic*.,ref=\thesubsection.\theenumi]
\item The ratio PR:RQ equals 2$\sqrt2:\sqrt5$
\item In any triangle, the bisector of an angle divides the triangle into two similar triangles.
\end{enumerate}
\begin{enumerate}
    
     \item  Statement-1 is true, Statement-2 is true ; Statement-2 is not a correct explanation for Statement-1
     \item  Statement-1 is true, Statement-2 is true ;  Statement-2 is not a correct explanation for Statement-1
     \item  Statement-1 is True,Statement False
     \item  Statement-1 is False,Statement True\\
\end{enumerate}


\item For a point $\vec{P}$ in the plane, let $d_1(\vec{P})$ and $d_2(\vec{P})$ be the distance of the point $\vec{P}$ from the lines 
\begin{align} 
\myvec{1 & -1} \vec {x} &=0
\\
\myvec{1 & 1} \vec {x} &=0
\end{align} 
respectively. 
Find the area of the region $R$ consisting of all points $\vec{P}$ lying in the first quadrant of the plane and satisfying 
\begin{align} 
2\leq d_1(\vec{P})+d_2(\vec{P}) \leq 4.
\end{align} 
    \item A triangle with vertices \myvec{4\\0},\myvec{-1\\-1} and \myvec{3\\5} is
    \begin{enumerate}
     \item  isosceles and right angled
     \item  isosceles but not right angled
     \item  right angled but not isosceles
     \item  neither right angled nor isosceles
     \end{enumerate}
    \item Find the locus of mid points of the portion between the axis of 
\begin{align}\myvec{\cos\alpha & \sin\alpha}\vec {x} &= 0\end{align} 
where p is constant
    \item If the pair of the lines \begin{align} \vec{x}^T\myvec{a & 2h \\ 0 & b} \vec {x} + 2\myvec{g & f}\vec {x} + c &=0\end{align}  intersects the y-axis then
    \begin{enumerate}
     \item  $2fgh=bg^2$+c$h^2$
     \item  $bg^2\neq c h^2$
     \item  $abc=2fgh$
     \item  none of these
     \end{enumerate}
    \item A pair of lines represented by \begin{align} \vec{x}^T\myvec{3a & 5 \\ 0 & (a^2-2)} \vec {x} &=0\end{align} are perpendicular to each other for
    \begin{enumerate}
     \item  two values of a 
     \item  $\forall$ a
     \item  for one value of a 
     \item  for no values of a 
     \end{enumerate}
    \item A square of side $ a$ lies above the x-axis and has one vertex at the origin. The side passing through the origin makes an angle $\alpha (0\textless\alpha\textless\pi$/4) with the positive direction of the x-axis. Find the equation of its diagonal not passing through the origin.
    \item If the pair of straight lines \begin{align}\vec{x}^T \myvec{1 & -2p \\ 0 & -1}\vec {x} &= 0\end{align} and \begin{align}\vec{x}^T \myvec{1 & -2q \\ 0 & -1}\vec {x} &= 0\end{align} be such that each pair bisects the angle between the other pair, then find the relation between $p$ and $q$.
    \item Find the locus of the centroid of the triangle whose vertices are \myvec{a \cos t\\a \sin t} , \myvec{b \sin t\\-b \cos t} and \myvec{1\\0}.
    \item If $x_1$,$x_2$,$x_3$ and $y_1$,$y_2$,$y_3$ are both in G.P. with the same common ratio, then the common points $\myvec{x_1\\y_1}, \myvec{x_2\\y_2}$ and $\myvec{x_3\\y_3}$
    \begin{enumerate}
     \item  are verices of a triangle
     \item  lie on a straight line
     \item  lie on a ellipse
     \item  lie on a circle
     \end{enumerate}
    \item If the equation of the locus of a point equidistant from the points $\myvec{a_1\\b_1}, \myvec{a_2\\b_2}$ is \begin{align}\myvec{a_1-b_2 & a_1-b_2}\vec {x} +c &= 0 \end{align} then find the value of c. 
    \item Let $\vec{A}=\myvec{2\\-3}$ and $\vec{B}=\myvec{-2\\3}$ be the vertices of a triangle ABC. If the centroid of the triangle moves on the line \begin{align}\myvec{2 & 3}\vec {x} &= 1 \end{align} then find the locus of the vertex $\vec{C}$.
    \item Find the equation of the straight line passing through the point \myvec{4\\3} and making intercepts on the coordinate axis whose sum is -1.
    \item If the sum of the slopes of the lines given by $\vec {x}^T \myvec{1 & -2c \\ 0 & -1} \vec {x} = 0$ is 4 times their product, then find  $c$.
    \item If one of the lines given by \begin{align}\vec {x}^T \myvec{6 & -1 \\ 0 & 4c} \vec {x} &= 0\end{align} is \begin{align}\myvec{ 3 & 4} \vec {x} &= 0\end{align} then find $c$.
    \item The line parallel to the x-axis and passing through the intersection of the lines \begin{align}\myvec{a & 2b} \vec {x} + 3b &= 0 \\ \myvec{b & -2a} \vec {x} - 3a &= 0, \end{align} where \myvec{a\\b}$\neq$\myvec{0\\0} is
    \begin{enumerate}
     \item  below the x-axis at a distance of 3/2 from it
     \item  below the x-axis at a distance of 2/3 from it
     \item  above the x-axis at a distance of 3/2 from it
     \item  above the x-axis at a distance of 2/3 from it
     \end{enumerate}
    \item If a vertex of a triangle is \myvec{1\\1} and the mid points of two sides through this vertex are \myvec{-1\\2} and \myvec{3\\2}, then find the centroid of the triangle.
    \item A straight line through the point $\vec{A} = \myvec{3\\4}$ is such that its intercepts between the axes is bisected at $\vec{A}$. Find its equation.
    \item If $\myvec{a\\a^2}$ falls inside the angle made by the lines \begin{align}\myvec{-1 & 2} \vec {x} &= 0,  \\ \myvec{-3 & 1} \vec {x} &= 0 \\ \myvec{1 & 0} \vec{x} &> 0,\end{align} then find the range of $a$. 
    \item Let $\vec{A}=\myvec{h\\k}, \vec{B}=\myvec{1\\1}$ and  $\vec{C}=\myvec{2\\1}$ be the vertices of a right angled triangle with AC has its hypotenuse. If the area of the triangle is 1 sq.unit, then find the range of $k$.
    \item Let $\vec{P}=\myvec{-1\\0}, \vec{Q}=\myvec{0\\0}$ and $\vec{R}=\myvec{3\\\sqrt3}$ be three points. Find the equation of the bisector of the angle PQR. 
    \item If one of the lines of \begin{align}\vec{x}^T\myvec{-m & (1-m^2) \\ 0 & m} \vec {x} &= 0\end{align} is a bisector of the angle between the lines  \begin{align}\vec{x}^T\myvec{0 & 0 \\ 1 & 0} \vec {x} &= 0,\end{align} then find m.
    \item The perpendicular bisector of the line segment joining $\vec{P}=\myvec{1\\4}$ and $\vec{Q}=\myvec{k\\3}$ has y-intercept -4. Then a possible value of k is
    \begin{enumerate}
     \item  1 
     \item  2
     \item  -2
     \item  -4
     \end{enumerate}
    \item Find the shortest distance between the line \begin{align}\myvec{-1 & 1}\vec{x} &= 1\end{align} and the curve \begin{align}\vec{x}^T\myvec{0 & 0 \\ 0 & 1} \vec {x} + \myvec{-1 & 0} &= 0\end{align}.
    \item The lines \begin{align}\myvec{(p(p^2+1) & -1}\vec{x} + q &= 0\\ \myvec{(p^2+1)^2 & (p^2+1)}\vec{x} + 2q &= 0\end{align} are perpendicular to a common line for:
    \begin{enumerate}
     \item  exactly one values of p
     \item  exactly two values of p
     \item  more than two values of p
     \item  no value of p
     \end{enumerate}
    \item Three distinct points $\vec{A},\vec{B}$ and $\vec{C}$ are given in the two dimensional coordinates plane such that the ratio of the distance of any one of them from the point \myvec{1\\0} to the distance from the point \myvec{-1\\0} is equal to $\frac{1}{3}$. Find the circumcentre of the triangle ABC.
    \item The line L given by \begin{align}\myvec{1/5 & 1/b}\vec{x} &= 1\end{align}  passes through the point \myvec{13\\32}. The line K is parallel to L and has the equation \begin{align}\myvec{\frac{1}{c} & \frac{1}{3}}\vec{x} = 1\end{align}. Find the distance between L and K. 
    \item If the line \begin{align}\myvec{2 & 1}\vec{x} &= k\end{align} passes through the point which divides the line segment joining the points\myvec{1\\1} and \myvec{2\\4} in the ratio 3:2, then find $k$.
    \item A ray of light along \begin{align}\myvec{1 & \sqrt3}\vec{x} &= \sqrt3\end{align} gets reflected upon reaching the x-axis, then find the equation of the reflected ray. 
    \item Find the x coordinate of the incentre of the triangle that has the coordinates of the mid points of its sides as \myvec{0\\1},\myvec{1\\1} and \myvec{1\\0}
    \item Let PS be the median of the triangle with vertices $\vec{P}= \myvec{2\\2}, \vec{Q}=\myvec{6\\-1} $ and $\vec{R}=  \myvec{7\\3}$. Find the equation of the line passing through \myvec{1\\-1} and parallel to PS. 
    \item Let a,b,c and d be non zero numbers. If the point of  intersection of the lines $\myvec{4a & 2a}\vec{x} + c = 0$ and $\myvec{5b & 2b}\vec{x} + d = 0$ lies in the fourth quadrant and is equidistant from the two axes, then
    \begin{enumerate}
     \item  3bc-2ad=0
     \item  3bc+2ad=0
     \item  2bc-3ad=0
     \item  2bc+3ad=0\\
     \end{enumerate}
    \item The number of points, having both coordinates as integers, that lie in the interior of the triangle with vertices \myvec{0\\0},\myvec{0\\41} and\myvec{41\\0}. 
is:
    \begin{enumerate}
     \item  820 
     \item  780
     \item  901
     \item  861\\
     \end{enumerate}
    \item Two sides of a rhombus are along the lines, \myvec{1&-1} $\vec {x}$+1=0 and \myvec{7&-1}$\vec {x}$-5=0. If its diagonals intersect at \myvec{-1\\-2} then which one of the following is a vertex of the rhombus?
    \begin{enumerate}
     \item  \myvec{\frac{1}{3}\\\frac{8}{3}}
     \item  \myvec{\frac{10}{3}\\\frac{7}{3}}
     \item  \myvec{-3\\-9}
     \item  \myvec{-3\\-8}\\
     \end{enumerate}
    \item A straight the through a fixed point \myvec{2\\3} intersects the coordinate axes at distinct points $\vec{P}$ and $\vec{Q}$. If $\vec{O}$ is the origin and the rectangle OPRQ is completed, then find the locus of $\vec{R}$.
    \item Consider the set of all lines \myvec{p&q} $\vec {x}$+r=0 such that $3p+2q+4r=0$. Which one of the following statements is true?
    \begin{enumerate}
     \item  The lines are concurrent at the point \myvec{\frac{3}{4}\\\frac{1}{2}}
     \item  Each line passes through the origin
     \item  The lines are all parallel
     \item  The lines are not concurrent
     \end{enumerate}
    \item Find the slope of a line passing through $\vec{P} = \myvec{2\\3}$ and intersecting the line \myvec{1&1}
    $\vec {x}$=7 at a distance of 4 units from $\vec{P}$.
\end{enumerate}

%\end{document}
