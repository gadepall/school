\renewcommand{\theequation}{\theenumi}
\begin{enumerate}[label=\arabic*.,ref=\thesubsection.\theenumi]
\item Find the locus of a point which is equidistant from the points $\myvec{6\\-3}$, $\myvec{-4\\7}$.
\item Find the point on the line
\begin{align}
\myvec{2 & 5}\vec{x}+7=0
\end{align}
which is equidistant from the points $\myvec{2\\-3}$, $\myvec{-4\\1}$.
\item Find the coordinates of the circumcentre of the triangle whose corners are at the points $\myvec{4\\3}$, $\myvec{-1\\2}$, $\myvec{2\\-2}$.
\item Find the equations of the lines through $\myvec{3\\1}$ which are respectively parallel and perpendicular to the line joining the points $\myvec{2\\4}$, $\myvec{5\\-6}$.
\item Find the locus of a point at which the join of the points $\myvec{2\\1}$ and $\myvec{-3\\4}$ subtends a right angle.
\item Find the orthocentre of a triangle whose corners are at the points $\myvec{1\\2}$, $\myvec{-3\\-4}$, $\myvec{6\\2}$.
\item Prove that the line joining the points $\myvec{2\\-1}$, $\myvec{-3\\5}$ makes with the axes a triangle of area $\frac{49}{60}$.
\item $ABCD$ is a parallelogram and the coordinates of $\vec{A}$, $\vec{B}$ and $\vec{C}$ are $\myvec{2\\4}$, $\myvec{1\\2}$ and $\myvec{4\\1}$ Find the coordinates of $\vec{D}$.
\item Find the area of the triangle formed by the lines 
\numberwithin{equation}{enumi}
\begin{align}
\myvec{3&-2}\vec{x}=5
\\ 
\myvec{3 &4}\vec{x}=7
\\
\myvec{0 & 1}\vec{x}  +2 = 0
\end{align}
\item Find the centre of the inscribed circle of the triangle whose sides are
\begin{align}
\myvec{3&-4}\vec{x}&=0
\\
\myvec{ 12&-5}\vec{x}&=0,
\\ 
\myvec{4&3}\vec{x}&=8
\end{align}
\renewcommand{\theequation}{\theenumi}
\item The ends of a diagonal of a square are on the coordinate axes at the points $\myvec{2a\\0}$, $\myvec{0\\a}$.  Find the equations of the sides.
\item The sides of a triangle $ABC$ are
\begin{align}
AB=3, BC = 5, CA = 4
\end{align}
and $\vec{A}, \vec{B}$ are on the axes $OX$, $OY$ respectively, while $AC$ makes an angle $\theta$ with $OX$.  Prove that the locus of $\vec{C}$, as $\theta$ varies, is given by the equation
\begin{align}
\vec{x}^T\myvec{16 & -12\\-12 & 25}\vec{x} = 256
%16x^2-24xy+25y^2 = 256
\end{align}
\item Prove that the locus of a point at which the join of the points $\myvec{a\\0}$ and $\myvec{-a\\0}$ subtends an angle of $45\degree$ is
\begin{align}
\vec{x}^T\vec{x}-2a\myvec{0 & 1}\vec{x} = a^2
%x^2+y^2-2ay = a^2
\end{align}
\numberwithin{equation}{enumi}
\item Prove that the line
\begin{align}
\vec{n}^T\vec{x}+c=0
\end{align}
divides the line joining  the points $\vec{x}_1, \vec{x}_2$ in the ratio
\begin{align}
-\frac{\vec{n}^T\vec{x}_1+c}{\vec{n}^T\vec{x}_2+c}
\end{align}
\item Find the equation of the line joining the point $\vec{x}_1$, to the point of intersection of the lines
\begin{align}
\vec{n}^T\vec{x}+c=0
\\
\vec{n}_1^T\vec{x}+c_1=0
\end{align}
\item Find  the equations of the diagonals of the parallelogram whose sides are
\begin{align}
\vec{n}^T\vec{x}+c&=0
\\
\vec{n}^T\vec{x}+d&=0
\\
\vec{n}_1^T\vec{x}+c_1&=0
\\
\vec{n}_1^T\vec{x}+d_1&=0
\end{align}
%are
%{\small
%\begin{align}
%\brak{c_1-d_1}\brak{ax+by+c}-\brak{c-d}\brak{a_1x+b_1y+c_1} = 0
%\\
%\brak{c_1-d_1}\brak{ax+by+c}+\brak{c-d}\brak{a_1x+b_1y+c_1} = 0
%\\
%\brak{c-d}\brak{c_1-d_1}/\brak{ab_1-a_1b}.
%\end{align}
%}
\renewcommand{\theequation}{\theenumi}
\item Prove that for all values of $k$ the line
\begin{align}
\myvec{2+k &1-2k}\vec{x} + 5 = 0
\end{align}
passes through a fixed point, and find its coordinates.
\item Find the angle between the lines
\begin{align}
\vec{x}^T
\myvec{1 & -\sec \theta\\-\sec\theta & 1} 
\vec{x} = 0
\end{align}
\numberwithin{equation}{enumi}
\item Prove that the pairs of straight lines represented by
\begin{align}
\vec{x}^T
\myvec{1 & \frac{1}{2}\\\frac{1}{2} & 0} 
\vec{x} = 0
\\
\vec{x}^T
\myvec{6 & -\frac{1}{2}\\-\frac{1}{2} & -1} 
\vec{x} = 0
\end{align}
are such that the angles between one pair are equal to the angles between the other pair.
\renewcommand{\theequation}{\theenumi}
\item Find the angles between the lines
\begin{align}
x^3-3x^2y-3xy^2+y^3 = 0
\end{align}
\numberwithin{equation}{enumi}
\item Find the area of the triangle whose sides are given by
\begin{align}
\vec{x}^T
\myvec{1 & -{2}\\-{2} & 3} 
\vec{x} = 0
\\
\myvec{3&4}\vec{x}=7
\end{align}
\renewcommand{\theequation}{\theenumi}
\item Show that the equation 
\begin{align}
    \vec{x}^T\myvec{6&-\frac{1}{2}\\-\frac{1}{2}&-15}\vec{x}+\myvec{-11& 31}\vec{x}-10=0\label{eq:solutions/2/6/22/eq:1}
\end{align}
represents two straight lines,and find the equations of the bisectors of the angles between them.
%
%\item Show that the equation
%\begin{align}
%\vec{x}^T
%\myvec{6 & -\frac{1}{2}\\-\frac{1}{2} & -15} 
%\vec{x} 
%+\myvec{-11&31}\vec{x}-10=0
%\end{align}
%represents two straight lines, and find the equations of the bisectors of the angles between them.
\\
\solution
The general second order equation can be expressed as follows,
\begin{align}
\vec{x^T}\vec{V}\vec{x}+2\vec{u^T}\vec{x}+f=0\label{eq:solutions/3/4/11/eq:1}
\end{align}
Comparing \eqref{eq:solutions/3/4/11/eq:0} with \eqref{eq:solutions/3/4/11/eq:1},
\begin{align}
\vec{V} = \myvec{5 & 1\\1 & 5}\label{eq:solutions/3/4/11/eq:2}\\
\vec{u} = \myvec{-7\\ -11}\label{eq:solutions/3/4/11/eq:3}\\
f = 27\label{eq:solutions/3/4/11/eq:4}
\end{align}
Let $\vec{c}$ be the change in the origin. The equation \eqref{eq:solutions/3/4/11/eq:1} can be modified as
\begin{align}
\vec{(x+c)^T}\vec{V}\vec{(x+c)}+2\vec{u^T}\vec{(x+c)}+f=0\label{eq:solutions/3/4/11/eq:5}
\end{align}
Considering \eqref{eq:solutions/3/4/11/eq:5}
\begin{align}
&\implies \vec{(x+c)^T}\vec{V}\vec{(x+c)}\\
&\implies \vec{x^T}\vec{V}\vec{x}+\vec{c^T}\vec{V}\vec{x}+\vec{x^T}\vec{V}\vec{c}+\vec{c^T}\vec{V}\vec{c}\label{eq:solutions/3/4/11/eq:6}
\end{align}
In the above equation
\begin{align}
\vec{c^T}\vec{V}\vec{x} = \vec{x^T}\vec{V}\vec{c}\label{eq:solutions/3/4/11/eq:7}
\end{align}
From \eqref{eq:solutions/3/4/11/eq:6} and \eqref{eq:solutions/3/4/11/eq:7} then \eqref{eq:solutions/3/4/11/eq:5} becomes
\begin{align}
\vec{x^T}\vec{V}\vec{x}+2\vec{c^T}\vec{V}\vec{x}+\vec{c^T}\vec{V}\vec{c}+2\vec{u^T}\vec{x}+2\vec{u^T}\vec{c}+f = 0\label{eq:solutions/3/4/11/eq:8}
\end{align}
Comparing \eqref{eq:solutions/3/4/11/eq:0.1} and \eqref{eq:solutions/3/4/11/eq:8}
\begin{align}
2\vec{c^T}\vec{V}\vec{P}\vec{y} + 2\vec{u^T}\vec{P}\vec{y} =0\\
\vec{c^T}\vec{V}\vec{P}\vec{y} = - \vec{u^T}\vec{P}\vec{y}\\
\vec{c} = -\vec{V^{-1}}\vec{u}\label{eq:solutions/3/4/11/eq:9}
\end{align}
Substituting \eqref{eq:solutions/3/4/11/eq:2} and \eqref{eq:solutions/3/4/11/eq:3} in \eqref{eq:solutions/3/4/11/eq:9}
\begin{align}
\vec{c}=\frac{-1}{24}\myvec{5&-1\\-1&5}\myvec{-7\\-11}=\myvec{1\\2 }\label{eq:solutions/3/4/11/eq:10}
\end{align}
Hence \eqref{eq:solutions/3/4/11/eq:8} becomes
\begin{align}
\vec{x^T}\vec{V}\vec{x}+\vec{c^T}\vec{V}\vec{c}+2\vec{u^T}\vec{c}+f = 0 \label{eq:solutions/3/4/11/eq:11}
\end{align}
Substituting \eqref{eq:solutions/3/4/11/eq:2}, \eqref{eq:solutions/3/4/11/eq:3} and \eqref{eq:solutions/3/4/11/eq:10} the above equation becomes
\begin{align}
\vec{x^T}\myvec{5&1\\1&5}\vec{x}+\myvec{1&2}\myvec{5&1\\1&5}\myvec{1\\2}+2\myvec{-7&-11}\myvec{1\\2}\\+27 = 0
\end{align}
\begin{align}
\vec{x^T}\myvec{5&1\\1&5}\vec{x}+29-58+27=0\\
\vec{x^T}\vec{V}\vec{x}-2 = 0\label{eq:solutions/3/4/11/eq:12}
\end{align}
With change in the origin to point $\vec{c}=\myvec{1\\2}$ but the $\vec{V}$ doesn't change.
\begin{align}
\mydet{\vec{V}} = \mydet{5&1\\1&5} = 24
\end{align}
As $\mydet{\vec{V}} >0$ it represents a ellipse.Hence $\vec{V}$ can be written as,
\begin{align}
\vec{V}=\vec{P}\vec{D}\vec{P^T}\label{eq:solutions/3/4/11/eq:13}
\end{align}
 The characteristic equation of $\vec{V}$ is given by
\begin{align}
\mydet{\vec{V}-\vec{I}\lambda} = 0\\
\mydet{5-\lambda & 1\\1& 5-\lambda} =0\\
\implies \lambda^2 - 10\lambda+24 = 0
\end{align}
Hence the eigen vales are,
\begin{align}
\lambda_1 = 4\\
\lambda_2 = 6
\end{align}
Hence diagonal vector is given by,
\begin{align}
\vec{D}= \myvec{\lambda_1&0\\0&\lambda_2} = \myvec{4&0\\0&6}\label{eq:solutions/3/4/11/eq:14}
\end{align}
The eigen vector $\vec{p}$ is given by
\begin{align}
    \vec{V}\vec{p}&=\lambda\vec{p}\\
    (\vec{V}-\lambda\vec{I})\vec{p}&=0
\end{align}
For $\lambda_1 = 4$  the eigenvector is,
\begin{align}
\vec{V}-\lambda_1\vec{I}= \myvec{1&1\\1&1} \xleftrightarrow[]{R_2 \leftarrow R_2-R_1}\myvec{1&1\\0&0}\\
\vec{p_1}=\myvec{\frac{1}{\sqrt{2}}\\\frac{-1}{\sqrt{2}}}
\end{align}
For $\lambda_1 = 6$  the eigenvector is,
\begin{align}
\vec{V}-\lambda_1\vec{I}= \myvec{-1&1\\1&-1} \xleftrightarrow[]{R_2 \leftarrow R_2+R_1}\myvec{-1&1\\0&0}\\
\vec{p_1}=\myvec{\frac{1}{\sqrt{2}}\\\frac{1}{\sqrt{2}}}
\end{align}
Hence,
\begin{align}
\vec{P} =\myvec{\vec{p_1}&\vec{p_2}} = \myvec{\frac{1}{\sqrt{2}}&\frac{1}{\sqrt{2}}\\\frac{-1}{\sqrt{2}}&\frac{1}{\sqrt{2}}} \label{eq:solutions/3/4/11/eq:15}
\end{align}
Substituting \eqref{eq:solutions/3/4/11/eq:14} and \eqref{eq:solutions/3/4/11/eq:15} in \eqref{eq:solutions/3/4/11/eq:13}
\begin{align}
\vec{V}= \myvec{\frac{1}{\sqrt{2}}&\frac{1}{\sqrt{2}}\\\frac{-1}{\sqrt{2}}&\frac{1}{\sqrt{2}}}\myvec{4&0\\0&6}\myvec{\frac{1}{\sqrt{2}}&\frac{-1}{\sqrt{2}}\\\frac{1}{\sqrt{2}}&\frac{1}{\sqrt{2}}}\label{eq:solutions/3/4/11/eq:16}
\end{align}
Hence substituting \eqref{eq:solutions/3/4/11/eq:16} in \eqref{eq:solutions/3/4/11/eq:12}
\begin{align}
\vec{x^T}\myvec{\frac{1}{\sqrt{2}}&\frac{1}{\sqrt{2}}\\\frac{-1}{\sqrt{2}}&\frac{1}{\sqrt{2}}}\myvec{4&0\\0&6}\myvec{\frac{1}{\sqrt{2}}&\frac{-1}{\sqrt{2}}\\\frac{1}{\sqrt{2}}&\frac{1}{\sqrt{2}}}\vec{x}=2\\
\vec{y^T}\myvec{4&0\\0&6}\vec{y}=2\\
\vec{y^T}\myvec{2&0\\0&3}\vec{y}=1\label{eq:solutions/3/4/11/eq:17}
\end{align}
where $\vec{y}$ is given by Affine transformation
\begin{align}
\vec{x}=\vec{P}\vec{y} \\
\vec{y}=\vec{P^T}\vec{x} \label{eq:solutions/3/4/11/eq:22}
\end{align}
The rotation matrix $\vec{P}$ can be given by,
\begin{align}
\vec{P}=\myvec{\cos{\theta}&\sin{\theta}\\-\sin{\theta}&\cos{\theta}}\label{eq:solutions/3/4/11/eq:18}
\end{align}
Comparing \eqref{eq:solutions/3/4/11/eq:15} and \eqref{eq:solutions/3/4/11/eq:18}
\begin{align}
\cos{\theta} = \frac{1}{\sqrt{2}}\\
\theta = \frac{\pi}{4} 
\end{align}
But given the direction of coordinate axes changes so,
\begin{align}
\theta = \pi +\frac{\pi}{4}\label{eq:solutions/3/4/11/eq:19}
\end{align}
Subtituting \eqref{eq:solutions/3/4/11/eq:19} in \eqref{eq:solutions/3/4/11/eq:18} we get 
\begin{align}
\vec{P}=\myvec{\cos{\brak{\pi +\frac{\pi}{4}}} & \sin{\brak{\pi +\frac{\pi}{4}}}\\-\sin{\brak{\pi +\frac{\pi}{4}}}&\cos{\brak{\pi +\frac{\pi}{4}}}}\\
\vec{P}=\myvec{\frac{-1}{\sqrt{2}}&\frac{-1}{\sqrt{2}}\\\frac{1}{\sqrt{2}}&\frac{-1}{\sqrt{2}}}\label{eq:solutions/3/4/11/eq:20}
\end{align}
From \eqref{eq:solutions/3/4/11/eq:13} we find the diagonal matrix
\begin{align}
\vec{D}=\vec{P^T}\vec{V}\vec{P} = \myvec{\frac{-1}{\sqrt{2}}&\frac{1}{\sqrt{2}}\\\frac{-1}{\sqrt{2}}&\frac{-1}{\sqrt{2}}}\myvec{5&1\\1&5}\myvec{\frac{-1}{\sqrt{2}}&\frac{-1}{\sqrt{2}}\\\frac{1}{\sqrt{2}}&\frac{-1}{\sqrt{2}}}\\
\vec{D}=\myvec{6&0\\0&4}\label{eq:solutions/3/4/11/eq:21}
\end{align}
Hence using \eqref{eq:solutions/3/4/11/eq:20}, \eqref{eq:solutions/3/4/11/eq:21} and \eqref{eq:solutions/3/4/11/eq:13} in \eqref{eq:solutions/3/4/11/eq:12} we get.
\begin{align}
\vec{x^T}\myvec{\frac{-1}{\sqrt{2}}&\frac{-1}{\sqrt{2}}\\\frac{1}{\sqrt{2}}&\frac{-1}{\sqrt{2}}}\myvec{6&0\\0&4}\myvec{\frac{-1}{\sqrt{2}}&\frac{1}{\sqrt{2}}\\\frac{-1}{\sqrt{2}}&\frac{-1}{\sqrt{2}}}\vec{x}=2
\end{align}
using \eqref{eq:solutions/3/4/11/eq:22} the above equation becomes,
\begin{align}
\vec{y^T}\myvec{6&0\\0&4}\vec{y}=2\\
\vec{y^T}\myvec{3&0\\0&2}\vec{y}=1\label{eq:solutions/3/4/11/eq:23}
\end{align} 
Hence from \eqref{eq:solutions/3/4/11/eq:17} and \eqref{eq:solutions/3/4/11/eq:23} proved that change of origin and the directions of the coordinate axes \eqref{eq:solutions/3/4/11/eq:0} can be tranformed to \eqref{eq:solutions/3/4/11/eq:0.1} or \eqref{eq:solutions/3/4/11/eq:0.2}
\begin{figure}[!ht]
\centering
\includegraphics[width=\columnwidth]{./solutions/3/4/11/Ellipse.png}
\caption{Ellipse}
\label{eq:solutions/3/4/11/fig:Ellipse}
\end{figure}


\item For what value of $k$ does the equation
\begin{align}
\vec{x}^T
\myvec{12 & \frac{7}{2}\\\frac{7}{2} & k} 
\vec{x} 
+\myvec{13&-1}\vec{x}+3=0
\end{align}
represent two straight lines? What is the angle between them?

\item For what value of $k$ does the equation 
\begin{equation} \label{eq:solutions/2/6/24/eq:1.1}
\vec{x}^T \myvec{6 && k/2 \\ k/2 && -3} \vec{x} + \myvec{4 && 5}\vec{x} -2 = 0
\end{equation}
represent a pair of straight lines?
\\
%\item For what values of $k$ does the equation
%\begin{align}
%\vec{x}^T
%\myvec{6 & \frac{k}{2}\\\frac{k}{2} & -3} 
%\vec{x} 
%+\myvec{4&5}\vec{x}-2=0
%\end{align}
%represent two straight lines?
\solution
The general second order equation can be expressed as follows,
\begin{align}
\vec{x^T}\vec{V}\vec{x}+2\vec{u^T}\vec{x}+f=0\label{eq:solutions/3/4/11/eq:1}
\end{align}
Comparing \eqref{eq:solutions/3/4/11/eq:0} with \eqref{eq:solutions/3/4/11/eq:1},
\begin{align}
\vec{V} = \myvec{5 & 1\\1 & 5}\label{eq:solutions/3/4/11/eq:2}\\
\vec{u} = \myvec{-7\\ -11}\label{eq:solutions/3/4/11/eq:3}\\
f = 27\label{eq:solutions/3/4/11/eq:4}
\end{align}
Let $\vec{c}$ be the change in the origin. The equation \eqref{eq:solutions/3/4/11/eq:1} can be modified as
\begin{align}
\vec{(x+c)^T}\vec{V}\vec{(x+c)}+2\vec{u^T}\vec{(x+c)}+f=0\label{eq:solutions/3/4/11/eq:5}
\end{align}
Considering \eqref{eq:solutions/3/4/11/eq:5}
\begin{align}
&\implies \vec{(x+c)^T}\vec{V}\vec{(x+c)}\\
&\implies \vec{x^T}\vec{V}\vec{x}+\vec{c^T}\vec{V}\vec{x}+\vec{x^T}\vec{V}\vec{c}+\vec{c^T}\vec{V}\vec{c}\label{eq:solutions/3/4/11/eq:6}
\end{align}
In the above equation
\begin{align}
\vec{c^T}\vec{V}\vec{x} = \vec{x^T}\vec{V}\vec{c}\label{eq:solutions/3/4/11/eq:7}
\end{align}
From \eqref{eq:solutions/3/4/11/eq:6} and \eqref{eq:solutions/3/4/11/eq:7} then \eqref{eq:solutions/3/4/11/eq:5} becomes
\begin{align}
\vec{x^T}\vec{V}\vec{x}+2\vec{c^T}\vec{V}\vec{x}+\vec{c^T}\vec{V}\vec{c}+2\vec{u^T}\vec{x}+2\vec{u^T}\vec{c}+f = 0\label{eq:solutions/3/4/11/eq:8}
\end{align}
Comparing \eqref{eq:solutions/3/4/11/eq:0.1} and \eqref{eq:solutions/3/4/11/eq:8}
\begin{align}
2\vec{c^T}\vec{V}\vec{P}\vec{y} + 2\vec{u^T}\vec{P}\vec{y} =0\\
\vec{c^T}\vec{V}\vec{P}\vec{y} = - \vec{u^T}\vec{P}\vec{y}\\
\vec{c} = -\vec{V^{-1}}\vec{u}\label{eq:solutions/3/4/11/eq:9}
\end{align}
Substituting \eqref{eq:solutions/3/4/11/eq:2} and \eqref{eq:solutions/3/4/11/eq:3} in \eqref{eq:solutions/3/4/11/eq:9}
\begin{align}
\vec{c}=\frac{-1}{24}\myvec{5&-1\\-1&5}\myvec{-7\\-11}=\myvec{1\\2 }\label{eq:solutions/3/4/11/eq:10}
\end{align}
Hence \eqref{eq:solutions/3/4/11/eq:8} becomes
\begin{align}
\vec{x^T}\vec{V}\vec{x}+\vec{c^T}\vec{V}\vec{c}+2\vec{u^T}\vec{c}+f = 0 \label{eq:solutions/3/4/11/eq:11}
\end{align}
Substituting \eqref{eq:solutions/3/4/11/eq:2}, \eqref{eq:solutions/3/4/11/eq:3} and \eqref{eq:solutions/3/4/11/eq:10} the above equation becomes
\begin{align}
\vec{x^T}\myvec{5&1\\1&5}\vec{x}+\myvec{1&2}\myvec{5&1\\1&5}\myvec{1\\2}+2\myvec{-7&-11}\myvec{1\\2}\\+27 = 0
\end{align}
\begin{align}
\vec{x^T}\myvec{5&1\\1&5}\vec{x}+29-58+27=0\\
\vec{x^T}\vec{V}\vec{x}-2 = 0\label{eq:solutions/3/4/11/eq:12}
\end{align}
With change in the origin to point $\vec{c}=\myvec{1\\2}$ but the $\vec{V}$ doesn't change.
\begin{align}
\mydet{\vec{V}} = \mydet{5&1\\1&5} = 24
\end{align}
As $\mydet{\vec{V}} >0$ it represents a ellipse.Hence $\vec{V}$ can be written as,
\begin{align}
\vec{V}=\vec{P}\vec{D}\vec{P^T}\label{eq:solutions/3/4/11/eq:13}
\end{align}
 The characteristic equation of $\vec{V}$ is given by
\begin{align}
\mydet{\vec{V}-\vec{I}\lambda} = 0\\
\mydet{5-\lambda & 1\\1& 5-\lambda} =0\\
\implies \lambda^2 - 10\lambda+24 = 0
\end{align}
Hence the eigen vales are,
\begin{align}
\lambda_1 = 4\\
\lambda_2 = 6
\end{align}
Hence diagonal vector is given by,
\begin{align}
\vec{D}= \myvec{\lambda_1&0\\0&\lambda_2} = \myvec{4&0\\0&6}\label{eq:solutions/3/4/11/eq:14}
\end{align}
The eigen vector $\vec{p}$ is given by
\begin{align}
    \vec{V}\vec{p}&=\lambda\vec{p}\\
    (\vec{V}-\lambda\vec{I})\vec{p}&=0
\end{align}
For $\lambda_1 = 4$  the eigenvector is,
\begin{align}
\vec{V}-\lambda_1\vec{I}= \myvec{1&1\\1&1} \xleftrightarrow[]{R_2 \leftarrow R_2-R_1}\myvec{1&1\\0&0}\\
\vec{p_1}=\myvec{\frac{1}{\sqrt{2}}\\\frac{-1}{\sqrt{2}}}
\end{align}
For $\lambda_1 = 6$  the eigenvector is,
\begin{align}
\vec{V}-\lambda_1\vec{I}= \myvec{-1&1\\1&-1} \xleftrightarrow[]{R_2 \leftarrow R_2+R_1}\myvec{-1&1\\0&0}\\
\vec{p_1}=\myvec{\frac{1}{\sqrt{2}}\\\frac{1}{\sqrt{2}}}
\end{align}
Hence,
\begin{align}
\vec{P} =\myvec{\vec{p_1}&\vec{p_2}} = \myvec{\frac{1}{\sqrt{2}}&\frac{1}{\sqrt{2}}\\\frac{-1}{\sqrt{2}}&\frac{1}{\sqrt{2}}} \label{eq:solutions/3/4/11/eq:15}
\end{align}
Substituting \eqref{eq:solutions/3/4/11/eq:14} and \eqref{eq:solutions/3/4/11/eq:15} in \eqref{eq:solutions/3/4/11/eq:13}
\begin{align}
\vec{V}= \myvec{\frac{1}{\sqrt{2}}&\frac{1}{\sqrt{2}}\\\frac{-1}{\sqrt{2}}&\frac{1}{\sqrt{2}}}\myvec{4&0\\0&6}\myvec{\frac{1}{\sqrt{2}}&\frac{-1}{\sqrt{2}}\\\frac{1}{\sqrt{2}}&\frac{1}{\sqrt{2}}}\label{eq:solutions/3/4/11/eq:16}
\end{align}
Hence substituting \eqref{eq:solutions/3/4/11/eq:16} in \eqref{eq:solutions/3/4/11/eq:12}
\begin{align}
\vec{x^T}\myvec{\frac{1}{\sqrt{2}}&\frac{1}{\sqrt{2}}\\\frac{-1}{\sqrt{2}}&\frac{1}{\sqrt{2}}}\myvec{4&0\\0&6}\myvec{\frac{1}{\sqrt{2}}&\frac{-1}{\sqrt{2}}\\\frac{1}{\sqrt{2}}&\frac{1}{\sqrt{2}}}\vec{x}=2\\
\vec{y^T}\myvec{4&0\\0&6}\vec{y}=2\\
\vec{y^T}\myvec{2&0\\0&3}\vec{y}=1\label{eq:solutions/3/4/11/eq:17}
\end{align}
where $\vec{y}$ is given by Affine transformation
\begin{align}
\vec{x}=\vec{P}\vec{y} \\
\vec{y}=\vec{P^T}\vec{x} \label{eq:solutions/3/4/11/eq:22}
\end{align}
The rotation matrix $\vec{P}$ can be given by,
\begin{align}
\vec{P}=\myvec{\cos{\theta}&\sin{\theta}\\-\sin{\theta}&\cos{\theta}}\label{eq:solutions/3/4/11/eq:18}
\end{align}
Comparing \eqref{eq:solutions/3/4/11/eq:15} and \eqref{eq:solutions/3/4/11/eq:18}
\begin{align}
\cos{\theta} = \frac{1}{\sqrt{2}}\\
\theta = \frac{\pi}{4} 
\end{align}
But given the direction of coordinate axes changes so,
\begin{align}
\theta = \pi +\frac{\pi}{4}\label{eq:solutions/3/4/11/eq:19}
\end{align}
Subtituting \eqref{eq:solutions/3/4/11/eq:19} in \eqref{eq:solutions/3/4/11/eq:18} we get 
\begin{align}
\vec{P}=\myvec{\cos{\brak{\pi +\frac{\pi}{4}}} & \sin{\brak{\pi +\frac{\pi}{4}}}\\-\sin{\brak{\pi +\frac{\pi}{4}}}&\cos{\brak{\pi +\frac{\pi}{4}}}}\\
\vec{P}=\myvec{\frac{-1}{\sqrt{2}}&\frac{-1}{\sqrt{2}}\\\frac{1}{\sqrt{2}}&\frac{-1}{\sqrt{2}}}\label{eq:solutions/3/4/11/eq:20}
\end{align}
From \eqref{eq:solutions/3/4/11/eq:13} we find the diagonal matrix
\begin{align}
\vec{D}=\vec{P^T}\vec{V}\vec{P} = \myvec{\frac{-1}{\sqrt{2}}&\frac{1}{\sqrt{2}}\\\frac{-1}{\sqrt{2}}&\frac{-1}{\sqrt{2}}}\myvec{5&1\\1&5}\myvec{\frac{-1}{\sqrt{2}}&\frac{-1}{\sqrt{2}}\\\frac{1}{\sqrt{2}}&\frac{-1}{\sqrt{2}}}\\
\vec{D}=\myvec{6&0\\0&4}\label{eq:solutions/3/4/11/eq:21}
\end{align}
Hence using \eqref{eq:solutions/3/4/11/eq:20}, \eqref{eq:solutions/3/4/11/eq:21} and \eqref{eq:solutions/3/4/11/eq:13} in \eqref{eq:solutions/3/4/11/eq:12} we get.
\begin{align}
\vec{x^T}\myvec{\frac{-1}{\sqrt{2}}&\frac{-1}{\sqrt{2}}\\\frac{1}{\sqrt{2}}&\frac{-1}{\sqrt{2}}}\myvec{6&0\\0&4}\myvec{\frac{-1}{\sqrt{2}}&\frac{1}{\sqrt{2}}\\\frac{-1}{\sqrt{2}}&\frac{-1}{\sqrt{2}}}\vec{x}=2
\end{align}
using \eqref{eq:solutions/3/4/11/eq:22} the above equation becomes,
\begin{align}
\vec{y^T}\myvec{6&0\\0&4}\vec{y}=2\\
\vec{y^T}\myvec{3&0\\0&2}\vec{y}=1\label{eq:solutions/3/4/11/eq:23}
\end{align} 
Hence from \eqref{eq:solutions/3/4/11/eq:17} and \eqref{eq:solutions/3/4/11/eq:23} proved that change of origin and the directions of the coordinate axes \eqref{eq:solutions/3/4/11/eq:0} can be tranformed to \eqref{eq:solutions/3/4/11/eq:0.1} or \eqref{eq:solutions/3/4/11/eq:0.2}
\begin{figure}[!ht]
\centering
\includegraphics[width=\columnwidth]{./solutions/3/4/11/Ellipse.png}
\caption{Ellipse}
\label{eq:solutions/3/4/11/fig:Ellipse}
\end{figure}


%
\end{enumerate}
