The equations for the circle and line in \eqref{eq:solutions/18/3/Latex/eq:1} and \eqref{eq:solutions/18/3/Latex/eq:2} can be rewritten in vector form as:
\begin{align}
  \norm{\vec{x}}^2 = 4 \\
  \myvec{1 & 2}\vec{x} = 6 \label{eq:solutions/18/3/Latex/eq:3}
\end{align}
The center of the circle happens to be $(0, 0)$ \\
The equation of a line is of the form:
\begin{align}
  \vec{n}^T\vec{x} = c \label{eq:solutions/18/3/Latex/eq:4}
\end{align}
Where n is the normal to the line. \\
Comparing \eqref{eq:solutions/18/3/Latex/eq:4} to \eqref{eq:solutions/18/3/Latex/eq:3},
\begin{align}
  \vec{n} = \myvec{1 \\ 2}
\end{align}
Since the tangent is parallel to the line in \eqref{eq:solutions/18/3/Latex/eq:3}, it will also have the same normal.

The point of contact for a conic is given by:
\begin{align}
  \vec{v} = \vec{V}^{-1}(\kappa\vec{n}-\vec{u}) \label{eq:solutions/18/3/Latex/eq:5}
\end{align}
where,
\begin{align}
  \kappa = \pm \sqrt[]{\frac{\vec{u}^T\vec{V^{-1}u}-f}{\vec{n}^T\vec{V}^{-1}\vec{n}}} \label{eq:solutions/18/3/Latex/eq:6}
\end{align}
For a circle,
\begin{align}
  \vec{V} = \vec{I}
\end{align}
Using properties of identity matrix, we get:
\begin{align}
  \vec{I}^{-1} = \vec{I} \\
  \vec{IX} = \vec{X}
\end{align}
Therefore \eqref{eq:solutions/18/3/Latex/eq:5} and \eqref{eq:solutions/18/3/Latex/eq:6} simplify to:
\begin{align}
  \kappa = \pm \sqrt[]{\frac{\vec{u}^T\vec{u}-f}{\vec{n}^T\vec{n}}} \\
  \implies \vec{v} = \kappa\vec{n-u}
\end{align}
Substituting the values, we get:
\begin{align}
  \kappa = \pm \sqrt[]{\frac{4}{\myvec{1 & 2}\myvec{1 \\ 2}}} \\
  \implies \kappa = \pm \sqrt[]{\frac{4}{5}} \\
  \vec{q} = \pm \sqrt[]{\frac{4}{5}}\myvec{1 \\ 2} \\
  \implies \vec{q_1} = \myvec{\sqrt[]{\frac{4}{5}} \\[0.2cm] \sqrt[]{\frac{16}{5}}}, \vec{q_2} = -\myvec{\sqrt[]{\frac{4}{5}} \\[0.2cm] \sqrt[]{\frac{16}{5}}}
\end{align}
Since there are two points of contact, there are two tangents parallel to \eqref{eq:solutions/18/3/Latex/eq:3} that have the same normal vector.
\begin{align}
  \implies \vec{n}^T\vec{q_1} = c_1 \\
  \vec{n}^T\vec{q_2} = c_2
\end{align}
Substituting the values, we get:
\begin{align}
  c_1 = 2\sqrt[]{5}, c_2 = -2\sqrt[]{5}
\end{align}
Therefore, the equation of the tangents are:
\begin{align}
  \myvec{1 & 2}\vec{x} = 2\sqrt[]{5} \\
  \myvec{1 & 2}\vec{x} = -2\sqrt[]{5}
\end{align}
The plot of the circle with the tangents is given below:
\begin{figure}[h]
\centering
    \includegraphics[width=\columnwidth]{solutions/18/3/Latex/Figure_1.png}
    \caption{Circle centered at $(0, 0)$ with tangents parallel to line $x+2y=6$.}
    \label{eq:solutions/18/3/Latex/fig:1}
\end{figure}
