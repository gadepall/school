\renewcommand{\theequation}{\theenumi}
\begin{enumerate}[label=\arabic*.,ref=\thesubsection.\theenumi]
\numberwithin{equation}{enumi}
\item The angles of quadrilateral are in the ratio 3 : 5 : 9 : 13. Find all the angles of the quadrilateral.
\item If the diagonals of a parallelogram are equal, then show that it is a rectangle. 
\item Show that if the diagonals of a quadrilateral bisect each other at right angles, then it is a rhombus.
\item Show that the diagonals of a square are equal and bisect each other at right angles. 
\item Show that if the diagonals of a quadrilateral are equal and bisect each other at right angles, then it is a square.
\item Diagonal $AC$ of a parallelogram $ABCD$ bisects $\angle A$ . Show that
(i) it bisects  $\angle C$  also, (ii) $ABCD$ is a rhombus.
\item $ABCD$ is a rhombus. Show that diagonal $AC$ bisects $\angle A$ as well as  $\angle C$  and diagonal $BD$ bisects  $\angle B$  as well as  $\angle D$ .
\item $ABCD$ is a rectangle in which diagonal $AC$ bisects $\angle A$ as well as  $\angle C$ . Show that: (i) $ABCD$ is a square (ii) diagonal $BD$ bisects  $\angle B$  as well as  $\angle D$ .
% $SP$   $parallel$   $QR$  ( $SP$  is a part of DP and  $QR$  is a part of QB)  $SQ$   $parallel$   $PR$ 
%Similarly, quadrilateral $DPBQ$ is a parallelogram, because  $DQ$   $parallel$   $PB$  and  $DQ$  =  $PB$ 
%1 2
%CD (Since AB  $parallel$  CD) (1)
%(Given) (Why?)
%(2) [From (1) and (2) and Theorem 8.8] 2019-2020QUADRILATERALS 147
\item In parallelogram $ABCD$, two points $P$ and $Q$ are taken on diagonal $BD$ such that $DP = BQ$. Show that: (i)  $\triangle  APD  \cong   \triangle  CQB$ (ii) $AP = CQ$ (iii)  $\triangle  AQB  \cong   \triangle  CPD$ (iv) $AQ = CP$ (v) $APCQ$ is a parallelogram
\item $ABCD$ is a parallelogram and $AP$ and $CQ$ are perpendiculars from vertices $A$ and $C$ on diagonal $BD$ . Show that (i)  $\triangle  APB  \cong   \triangle  CQD $ (ii) $AP = CQ$
\item In  $\triangle  ABC$ and  $\triangle  DEF, AB = DE, AB  parallel  DE, BC = EF$ and $BC  parallel  EF$. Vertices $A, B$ and $C$ are joined to vertices $D, E$ and $F$ respectively. Show that
(i) quadrilateral $ABED$ is a parallelogram (ii) quadrilateral $BEFC$ is a parallelogram (iii) $AD  parallel  CF$ and $AD = CF$ (iv) quadrilateral $ACFD$ is a parallelogram (v) $AC$ = $DF$ (vi)  $\triangle  ABC  \cong   \triangle  DEF$.
\item $ABCD$ is a trapezium in which $AB$  $parallel$  $CD$ and $AD = BC$. Show that (i)$\angle A$ =  $\angle B$  (ii)  $\angle C  =  \angle D$  (iii)  $\triangle  ABC  \cong   \triangle  BAD$ (iv) diagonal $AC$ = diagonal $BD$ 
\item $ABCD$ is a quadrilateral in which $P, Q, R$ and $S$ are mid-points of the sides $AB, BC, CD$ and $DA$ $AC$ is a diagonal. Show that :
(i) $SR$  $\parallel$  $AC$ and $SR =\frac{1}{ 2}AC$
(ii) $PQ = SR$ (iii)  $PQRS$  is a parallelogram.
\item $ABCD$ is a rhombus and  $P, Q, R$ and $S$  are the mid-points of the sides  $AB, BC, CD$ and $DA$ respectively. Show that the quadrilateral  $PQRS$  is a rectangle.
\item $ABCD$ is a rectangle and  $P, Q, R$ and $S$  are mid-points of the sides  $AB, BC, CD$ and $DA$ respectively. Show that the quadrilateral  $PQRS$  is a rhombus.
\item $ABCD$ is a trapezium in which $AB  \parallel  DC, BD$ is a diagonal and $E$ is the mid-point of $AD$. A line is drawn through $E$ $\parallel$  $AB$ intersecting $BC$ at $F$. Show that $F$ is the mid-point of $BC$.
\item In a parallelogram $ABCD$, $E$ and $F$ are the mid-points of sides $AB$ and $CD$ respectively . Show that the line segments $AF$ and $EC$ trisect the diagonal $BD$.
\item Show that the line segments joining the mid-points of the opposite sides of a quadrilateral bisect each other.
\item ABC is a triangle right angled at $C$. A line through the mid-point $M$ of hypotenuse $AB$ and parallel to $BC$ intersects $AC$ at $D$. Show that (i) $D$ is the mid-point of $AC$
(ii) $MD \perp AC$ (iii) $CM = MA = \frac{1}{2}AB$

\item $ABCD$ is a cyclic quadrilateral with 
\begin{align}
\angle A &= 4y+20
\\
\angle B &= 3y-5
\\
\angle C &= -4x
\\
\angle D &= -7x+5
\end{align}
%
Find its angles.
\item Draw a quadrilateral in the Cartesian plane, whose vertices are \myvec{– 4\\ 5}, \myvec{0\\ 7}, \myvec{5\\ – 5} and \myvec{– 4\\ –2}. Also, find its area.
\item Find the area of a rhombus if its vertices are \myvec{3\\0}, \myvec{4\\5}, \myvec{-1\\4} and \myvec{-2\\-1} taken in order.
\item Without using distance formula, show that points \myvec{– 2\\ – 1}, \myvec{4\\ 0}, \myvec{3\\ 3} and \myvec{–3\\ 2} are the vertices of a parallelogram.
\item  Find the area of the quadrilateral whose vertices, taken in order, are 
 \myvec{-4\\2},  \myvec{-3\\-5},  \myvec{3\\-2},  \myvec{2\\3}. 
\item The two opposite vertices of a square are \myvec{-1\\2},  \myvec{3\\2}. Find the coordinates of the other two vertices.
\item $ABCD$ is a rectangle formed by the points $\vec{A} = \myvec{-1\\-1}, \vec{B} = \myvec{-1\\4}, \vec{C} = \myvec{5\\4}, \vec{D} = \myvec{5\\-1}$. $ \vec{P}, \vec{Q}, \vec{R}, \vec{S}$ are the mid points of $AB, BC, CD, DA$ respectively.  Is the quadrilateral $PQRS$ a 
\begin{enumerate}
\item square?
\item rectangle?
\item rhombus?
\end{enumerate}
\item Find the area of a parallelogram whose adjacent sides are given by the vectors \myvec{3\\1\\4} and \myvec{1\\-1\\1}.
\item Find the area of a parallelogram whose adjacent sides are determined by the vectors $\vec{a} = \myvec{1\\-1\\3}$ and $\vec{b}=\myvec{2\\-7\\1}$.
\item Find the area of a rectangle $ABCD$ with vertices
$\vec{A} = \myvec{-1\\\frac{1}{2}\\ 4},
 \vec{B} = \myvec{1\\\frac{1}{2}\\ 4},
\vec{C} = \myvec{1\\-\frac{1}{2}\\ 4},
\vec{D} = \myvec{-1\\-\frac{1}{2}\\ 4}.
$
\item The two adjacent sides of a parallelogram are \myvec{2\\ -4 \\ -5} and  \myvec{1\\-2\\ -3}. Find the unit vector parallel to its diagonal.  Also, find its area.
%
\item A park, in the shape of a quadrilateral ABCD, has $\angle C = 90\degree, AB = 9 m, BC = 12 m, CD = 5 m$ and$ AD = 8 m$. How much area does it occupy?
2. Find the area of a quadrilateral ABCD in which $AB = 3 cm, BC = 4 cm, CD = 4 cm, DA = 5 cm$ and $AC = 5 cm$.
\item A triangle and a parallelogram have the same base and the same area. If the sides of the triangle are 26 cm, 28 cm and 30 cm, and the parallelogram stands on the base 28 cm, find the height of the parallelogram.
\item A rhombus shaped field has green grass for 18 cows to graze. If each side of the rhombus is 30 m and its longer diagonal is 48 m, how much area of grass field will each cow be getting?
\item A field is in the shape of a trapezium whose parallel sides are 25 m and 10 m. The non-parallel sides are 14 m and 13 m. Find the area of the field.
\end{enumerate}
%
