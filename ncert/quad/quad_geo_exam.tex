\renewcommand{\theequation}{\theenumi}
\begin{enumerate}[label=\arabic*.,ref=\thesubsection.\theenumi]
\numberwithin{equation}{enumi}


\item Show that the points $\vec{A} = \myvec{1\\7}, \vec{B} = \myvec{4\\2}, \vec{C}=\myvec{-1\\-1},\vec{D}= \myvec{-4\\4} $  are the vertices of a square.
\\
\solution By inspection, 
%
\begin{align}
\frac{\vec{A}+\vec{C}}{2}=\frac{\vec{B}+\vec{D}}{2} = \myvec{0\\3}
\end{align}
%
Hence, the diagonals $AC$ and $BD$ bisect each other.
%
Also, 
\begin{align}
\brak{\vec{A}-\vec{C}}^T
\brak{\vec{B}-\vec{D}} = 0
\end{align}
%
$\implies AC \perp BD $.  Hence $ABCD$ is a square.
\item If the points
$
\vec{A} = \myvec{6\\1}, 
\vec{B} = \myvec{8\\2}, 
\vec{C} = \myvec{9\\4}, 
\vec{D} = \myvec{p\\3}
$
are the vertices of a parallelogram, taken in order, find the value of $p$.
\\
\solution In the parallelogram $ABCD$, $AC$ and $BD$ bisect each other.  This can be used to find $p$.
\item If $\vec{A} = \myvec{-5\\7}, \vec{B} = \myvec{-4\\-5}, \vec{C} = \myvec{-1\\-6}, \vec{D} = \myvec{4\\5}$, find the area of the quadrilateral $ABCD$.
%
\\
\solution The area of  $ABCD$ is the sum of the areas of trianges ABD and CBD and is given by 
\begin{multline}
\frac{1}{2}\norm{\brak{\vec{A}-\vec{B}}\times \brak{\vec{A}-\vec{D}}}
\\
+
\frac{1}{2}\norm{\brak{\vec{C}-\vec{B}}\times \brak{\vec{C}-\vec{D}}}
\end{multline}
\item Show that the points 
$\vec{A} = \myvec{1\\2\\3},
 \vec{B} = \myvec{-1\\-2\\-1},
\vec{C} = \myvec{2\\3\\2},
\vec{D} = \myvec{4\\7\\6}.
$
are the vertices of a parallelogram $ABCD$ but it is not a rectangle.
%
\\
\solution Since the direction vectors
%
\begin{align}
\vec{A}-\vec{B}&= \vec{D}-\vec{C}
\\
\vec{A}-\vec{D}&= \vec{B}-\vec{C}
\end{align}
%
$AB \parallel CD$ and $AD \parallel BC$.  Hence $ABCD$ is a parallelogram.  However, 
%
\begin{align}
\brak{\vec{A}-\vec{B}}^T\brak{ \vec{A}-\vec{D}}\ne 0
\end{align}
%
Hence, it is not a rectangle.
The following code plots Fig. \ref{fig:quad_3d}
%
\begin{lstlisting}
codes/triangle/quad_3d.py
\end{lstlisting}
%
\begin{figure}[!ht]
\includegraphics[width=\columnwidth]{./triangle/figs/quad_3d.eps}
\caption{}
\label{fig:quad_3d}
\end{figure}
%

\item Find the area of a parallelogram whose adjacent sides are given by the vectors \myvec{3\\1\\4} and \myvec{1\\-1\\1}.
%
\\
\solution  The area is given by 
%
\begin{align}
\frac{1}{2}\norm{\myvec{3\\1\\4} \times \myvec{1\\-1\\1}}
\end{align}
%
\item Kamla has a triangular field with sides 240 m, 200 m, 360 m, where she grew wheat. In another triangular field with sides 240 m, 320 m, 400 m adjacent to the previous field, she wanted to grow potatoes and onions. She divided the field in two parts by joining the mid-point of the longest side to the opposite vertex and grew patatoes in one part and onions in the other part. Draw the figure for this problem.  How much area (in hectares) has been used for wheat, potatoes and onions? (1 hectare = 10000 $m^2$).
\item Students of a school staged a rally for cleanliness campaign. They walked through the lanes in two groups. One group walked through the lanes AB, BC and CA; while the other through AC, CD and DA. Then they cleaned the area enclosed within their lanes. If AB = 9 m, BC = 40 m, CD = 15 m, DA = 28 m and $\angle B = 90\degree$, which group cleaned more area and by how much? Draw the corresponding figure.  Find the total area cleaned by the students (neglecting the width of the lanes). 
%
\item Sanya has a piece of land which is in the shape of a rhombus. She wants her one daughter and one son to work on the land and produce different crops. She divided the land in two equal parts. If the perimeter of the land is 400 m and one of the diagonals is 160 m, how much area each of them will get for their crops? Draw the rhombus.
\end{enumerate}
%
