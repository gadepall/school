\renewcommand{\theequation}{\theenumi}
\begin{enumerate}[label=\arabic*.,ref=\thesubsection.\theenumi]
\numberwithin{equation}{enumi}
%
%
\item Express the problem of finding the distance of the point $\vec{P}=\myvec{3\\-5}$ from the line 
\label{prob:opt_line_dist}
\begin{align}
\label{eq:opt_line_nor}
L: \quad \myvec{3 & – 4}\vec{x}  = 26
\end{align}
%
as an optimization problem.
\\
\solution The given problem can be expressed as
%
\begin{align}
\label{eq:opt_line_dist}
\min_{\vec{x}}g(\vec{x}) &= \norm{\vec{x}-\vec{P}}^2
\\
\text{s.t.} \quad \vec{n}^T\vec{x} &= c
\label{eq:opt_line_dist_nor}
\end{align}
%
where 
%
\begin{align}
\vec{n} &= \myvec{3\\-4}
\\
c&=26
\end{align}
%
\item Convert  \eqref{eq:opt_line_dist} to an {\em unconstrained} optimization problem.
%
\\
%
\solution $L$ in \eqref{eq:opt_line_nor} can be expressed in terms of the direction vector $\vec{m}$ as
\begin{align}
\label{eq:opt_line_dir}
\vec{x} = \vec{A} + \lambda \vec{m}, 
\end{align}
where $\vec{A}$ is any point on the line and 
%
\begin{align}
\label{eq:opt_line_orth}
\vec{m}^T\vec{n} = 0
%\vec{m} &= \vec{Q}\vec{n} = \myvec{4\\3},
%\\
%\vec{Q} &= \myvec{0 & -1\\1 & 0}
\end{align}
%
Substituting \eqref{eq:opt_line_dir} in \eqref{eq:opt_line_dist}, an unconstrained optimization problem 
\begin{align}
\label{eq:opt_line_dist_uncon}
\min_{\lambda}f(\lambda) = \norm{\vec{A} + \lambda \vec{m}-\vec{P}}^2 
\end{align}
%
is obtained.
%
\item Solve \eqref{eq:opt_line_dist_uncon}.
%
\\
\solution 
\begin{align}
f(\lambda) 
%&=\norm{\vec{x}-\vec{P}}^2 = \norm{\vec{A} + \lambda \vec{m}-\vec{P}}^2 
%\\
& = \brak{ \lambda \vec{m}+\vec{A} -\vec{P}}^T \brak{ \lambda \vec{m}+\vec{A} -\vec{P}}
\\
&= \lambda^2 \norm{\vec{m}}^2 +2\lambda \vec{m}^T\brak{\vec{A} -\vec{P}} 
\nonumber \\
&\quad + \norm{\vec{A} -\vec{P}}^2
\label{eq:opt_line_dist_uncon_dist}
\end{align}
\begin{align}
\because f^{(2)}\lambda = 2\norm{\vec{m}}^2 > 0
\end{align}
%
the minimum value of $f(\lambda)$ is obtained when 
%
\begin{align}
 f^{(1)}(\lambda) &= 2\lambda\norm{\vec{m}}^2 + 2 \vec{m}^T\brak{\vec{A} -\vec{P}} =0
\\
\implies \lambda_{\min} &= -\frac{\vec{m}^T\brak{\vec{A} -\vec{P}}}{\norm{\vec{m}}^2}
\label{eq:opt_line_dist_uncon_lam_min}
\end{align}
%
Choosing $\vec{A}$ such that 
%
\begin{align}
\vec{m}^T\brak{\vec{A} -\vec{P}} &= 0,
\label{eq:opt_line_dist_uncon_trick}
\end{align}
%
substituting in \eqref{eq:opt_line_dist_uncon_lam_min},
%
\begin{align}
\label{eq:opt_line_dist_uncon_lam0}
\lambda_{\min} &= 0 \quad \text{and}
\\
\vec{A} -\vec{P} &= \mu \vec{n}
\label{eq:opt_line_dist_uncon_mu}
\end{align}
for some constant $\mu$. \eqref{eq:opt_line_dist_uncon_mu}
 is a consequence of \eqref{eq:opt_line_orth} and \eqref{eq:opt_line_dist_uncon_trick}. Also, from 
\eqref{eq:opt_line_dist_uncon_mu},
%
\begin{align}
\vec{n}^T\brak{\vec{A} -\vec{P} } &= \mu \norm{\vec{n}}^2
\\
\implies \mu & = \frac{\vec{n}^T\vec{A} -\vec{n}^T\vec{P} }{\norm{\vec{n}}^2} = \frac{c -\vec{n}^T\vec{P} }{\norm{\vec{n}}^2}
\label{eq:opt_line_dist_uncon_mu_sol}
\end{align}
%
from \eqref{eq:opt_line_dist_nor}.
%, where $\mu$ is some constant.
Substituting $\lambda_{\min} = 0$ in \eqref{eq:opt_line_dist_uncon},
%\label
%Thus, the shortest distance from $\vec{P}$ to $L$ is
%
\begin{align}
\min_{\lambda}f(\lambda) =  \norm{\vec{A} -\vec{P}}^2 = \mu^2\norm{\vec{n}}^2
\label{eq:opt_line_dist_uncon_f}
\end{align}
upon substituting from \eqref{eq:opt_line_dist_uncon_mu}. The distance between $\vec{P}$ and ${L}$ is then obtained from \eqref{eq:opt_line_dist_uncon_f} as
% obtained as 
\begin{align}
\norm{\vec{A} -\vec{P}} &= \abs{\mu}\norm{\vec{n}}
\\
&= \frac{\abs{\vec{n}^T\vec{P} -c }}{\norm{\vec{n}}}
\end{align}
after substituting for $\mu$ from  \eqref{eq:opt_line_dist_uncon_mu_sol}.

%\\
%\implies d &=  \norm{\vec{A} -\vec{P}} = \abs{\mu} \norm{\vec{n}}= \frac{\abs{ \vec{n}^T\vec{P}-c} }{\norm{\vec{n}}}
%\end{align}
%%
%after substituting from \eqref{eq:opt_line_dist_nor}.

%\item Find the equation of the line whose perpendicular distance from the origin is 4 units and the angle which the normal makes with the positive direction of x-axis is $15\degree$.
%\\
%\solution 
%%
%\item Find the 
%distance between the lines 
%\begin{align}
%L_1: \quad \vec{x} &= \myvec{1\\2\\-4} + \lambda_1\myvec{2 \\ 3 \\6}
%\\
%L_2: \quad \vec{x} &= \myvec{3\\3\\-5} + \lambda_2\myvec{2 \\ 3 \\6}
%\end{align}
%\label{prob:line_dist_parallel}
%%
%
%
%\item Find the shortest distance between the lines 
%\begin{align}
%L_1: \quad \vec{x} &= \myvec{1\\1\\0} + \lambda_1\myvec{2 \\ -1 \\1}
%\\
%L_2: \quad \vec{x} &= \myvec{2\\1\\-1} + \lambda_2\myvec{3 \\ -5 \\2}
%\end{align}
%\label{prob:line_dist_skew}
%%
%
%
%\item Find the distance of the plane 
%\begin{align}
%\myvec{2 & -3 & 4}\vec{x}-6  = 0
%\end{align}
%%
%from the origin.
%
%
%\item Find the equation of a plane which is at a distance of $\frac{6}{\sqrt{29}}$ from the origin and has  normal vector $\vec{n}=\myvec{2\\-3\\4}$.
%%
%\\
%\solution From the previous problem, the desired equation is
%%
%\begin{align}
%\myvec{2 & -3 & 4}\vec{x}-6  = 0
%\end{align}
%%
%
%\item Find the unit normal vector of the plane 
%\begin{align}
%\myvec{6 & -3 & -2}\vec{x}  = 1.
%\end{align}
%%
%\solution The normal vector is 
%%
%\begin{align}
%\vec{n} = \myvec{6 & -3 & -2}
%\\
%\because \norm{\vec{n}} = 7,
%\end{align}
%%
%the unit normal vector is 
%%
%\begin{align}
%\frac{\vec{n}}{\norm{\vec{n}}} = \frac{1}{7}\myvec{6 & -3 & -2}
%\end{align}
%%
%\item Find the coordinates of the foot of the perpendicular drawn from the origin to the plane 
%\begin{align}
%\label{eq:line_foot_perp}
%\myvec{2 & -3 & 4}\vec{x}-6  = 0
%\end{align}
%%
%\solution The normal vector is 
%%
%\begin{align}
%\vec{n}=\myvec{2 \\ -3 \\ 4}
%\end{align}
%%
%Hence, the foot of the perpendicular from the origin is $\lambda \vec{n}$.  Substituting in \eqref{eq:line_foot_perp},
%\begin{align}
%\lambda \norm{\vec{n}}^2 = 6  \implies \lambda = \frac{6}{\norm{\vec{n}}^2} = \frac{6}{29}
%\end{align}
%%
%Thus, the foot of the perpendicular is
%%
%\begin{align}
%\frac{6}{29}\myvec{2 \\ -3 \\ 4}
%\end{align}
%%
%\item Find the equation of the plane which passes through the point $\vec{A}=\myvec{5\\2\\-4}$ and perpendicular to the line with direction vector $\vec{n}=\myvec{2\\3\\-1}$.
%%
%\\
%\solution  The normal vector to the plane is $\vec{n}$. Hence from \eqref{eq:line_norm_vec}, the equation of the plane is 
%%
%\begin{align}
%\vec{n}^T\brak{\vec{x}-\vec{A}} &= 0
%\\
%\implies \myvec{2\\3\\-1}\vec{x} &=\myvec{2&3&-1}\myvec{5\\2\\-4}
%\\
%&=20
%\end{align}
%%
%%The following 
%%%
%%\begin{lstlisting}
%%codes/line/plane_3d.py
%%\end{lstlisting}
%%%
%%\begin{figure}[!ht]
%%\includegraphics[width=\columnwidth]{./line/figs/plane_3d.eps}
%%\caption{}
%%\label{fig:plane_3d}
%%\end{figure}
%%
%
%\item Find the equation of the plane passing through 
%$
%\bm{R} = \myvec{2\\5\\-3},
%\bm{S}= \myvec{-2\\-3\\5}
%$ 
%and 
%$
%\bm{T}= \myvec{5\\3\\-3}.
%$
%\label{prob:plane_3pts}
%\\
%\solution  If the equation of the plane be 
%\begin{align}
%\vec{n}^T\vec{x} &= c,
%\\
%\vec{n}^T\vec{R}=\vec{n}^T\vec{S}=\vec{n}^T\vec{T}&= c,
%\\
%\implies \myvec{\vec{R}-\vec{S} & \vec{S}-\vec{T}}^T\vec{n} &= 0
%\end{align}
%%
%after some algebra.
%Using row reduction on the above matrix, 
%\begin{align}
%\myvec{4 & 8 &-8 \\ -7  & -6 & 8} \xleftrightarrow[]{R_1\leftarrow \frac{R_1}{4}}\myvec{1 & 2 &-2 \\ -7  & -6 & 8}
%\\
%\xleftrightarrow[]{R_2\leftarrow R_2 + 7R_1}
%\myvec{
%1 & 2 &-2 
%\\ 
%0  & 8 & -6
%}
%\xleftrightarrow[]{R_2\leftarrow \frac{R_2}{2}}
%\myvec{
%1 & 2 &-2 
%\\ 
%0  & 4 & -3
%}
%\\
%\xleftrightarrow[]{R_1\leftarrow 2R_1-R_2}
%\myvec{
%2 & 0 &-1 
%\\ 
%0  & 4 & -3
%}
%\end{align}
%%
%Thus, 
%\begin{align}
%\vec{n} &= \myvec{\frac{1}{2}\\\frac{3}{4}\\1} = \myvec{2\\3\\4} \text{ and}
%\\
%c = \vec{n}^{T}\vec{T} = 7
%\end{align}
%%
%Thus, the equation of the plane is 
%%
%\begin{align}
%\myvec{2 & 3 & 4}\vec{n} = 7
%\end{align}
%%
%Alternatively, the normal vector to the plane can be obtained as
%%
%\begin{align}
%\vec{n} = \brak{\vec{R}-\vec{S}} \times \brak{\vec{S}-\vec{T}}
%\end{align}
%%
%The equation of the plane is then obtained from \eqref{eq:line_norm_vec} as 
%%
%\begin{align}
%\label{eq:plane_3pts_cross}
%\vec{n}^T\brak{\vec{x}-\vec{T}} = \sbrak{\brak{\vec{R}-\vec{S}} \times \brak{\vec{S}-\vec{T}}}^T\brak{\vec{x}-\vec{T}} = 0
%\end{align}
%%
%
%\item Find the equation of the plane with intercepts 2, 3 and 4 on the x, y and z axis respectively.
%\\
%\solution From the given information, the plane passes through the points \myvec{2\\0\\0}, \myvec{0\\3\\0} and \myvec{0\\0\\4} respectively. The equation can be obtained using Problem \ref{prob:plane_3pts}.
%
%\item Find the equation of the plane passing through the intersection of the planes 
%%
%\begin{align}
%\label{eq:plane_2p_1pt_1}
%\myvec{1 & 1 & 1}\vec{x}&=6  
%\\
%\myvec{2 & 3 & 4}\vec{x}&=-5
%\label{eq:plane_2p_1pt_2}
%\end{align}
%%
%and the point \myvec{1\\1\\1}.
%\\
%\solution The intersection of the planes is obtained by row reducing the augmented matrix as
%%
%\begin{align}
%\myvec{
%1 & 1 & 1 & 6
%\\
%2 & 3 & 4 & -5
%}
%\xleftrightarrow[]{R_2 = R_2 -2R_1}
%\myvec{
%1 & 1 & 1 & 6
%\\
%0 & 1 & 2 & -17
%}
%\\
%\xleftrightarrow[]{R_1 = R_1 -R_2}
%\myvec{
%1 & 0 & -1 & 23
%\\
%0 & 1 & 2 & -17
%}
%\\
%\implies 
%\vec{x} = \myvec{23\\-17\\0}+\lambda\myvec{1\\-2\\1}
%\end{align}
%%
%Thus, \myvec{23\\-17\\0} is another point on the plane.  The normal vector to the plane is then obtained as
%The normal vector to the plane is then obtained as
%%
%\begin{align}
%\brak{ \myvec{1\\1\\1}-\myvec{23\\-17\\0}}\times \myvec{1\\-2\\1} 
%\end{align}
%%
%which can be obtained by row reducing the matrix
%\begin{align}
%\myvec{
%1 & -2 & 1
%\\
%-22 & 18 & 1
%}
%\xleftrightarrow[]{R_2 = R_2+22R_1}
%\myvec{
%1 & -2 & 1
%\\
%0  & -26 & 23
%}
%\\
%\xleftrightarrow[]{R_1 = 13R_1-R_2}
%\myvec{
%13 & 0 & -10
%\\
%0  & -26 & 23
%}
%\\
%\implies \vec{n} = \myvec{\frac{10}{13}\\\frac{23}{26}\\1} = \myvec{20\\23\\26}
%\end{align}
%%
%Since the plane passes through \myvec{1\\1\\1}, using 
% \eqref{eq:line_norm_vec},
%%
%\begin{align}
%\myvec{20 & 23 & 26}\brak{\vec{x}- \myvec{1\\1\\1}} &= 0
%\\
%\implies 
%\myvec{20 & 23 & 26}\vec{x} &= 69
%\end{align}
%%
%Alternatively, the plane passing through the intersection of \eqref{eq:plane_2p_1pt_1} and 
%\eqref{eq:plane_2p_1pt_2} has the form 
%%
%\begin{align}
%\label{eq:plane_2p_1pt_lam}
%\myvec{1 & 1 & 1}\vec{x} + \lambda \myvec{2 & 3 & 4}\vec{x} &=6 -5\lambda  
%\end{align}
%%
%Substituting \myvec{1\\1\\1} in the above, 
%%
%\begin{align}
%\myvec{1 & 1 & 1}\myvec{1\\1\\1} + \lambda \myvec{2 & 3 & 4}\myvec{1\\1\\1} &=6 -5\lambda  
%\\
%\implies 3 + 9\lambda &= 6-5\lambda
%\\
%\implies &\lambda = \frac{3}{14}
%\end{align}
%%
%Substituting this value of $\lambda $ in \eqref{eq:plane_2p_1pt_lam} yields the equation of the plane.
%
%\item Show that the lines 
%\label{prob:line_coplanar}
%%
%\begin{align}
%\frac{x+3}{-3} = \frac{y-1}{1} &= \frac{z-5}{5}, 
%\\
%\frac{x+1}{-1} = \frac{y-2}{2} &= \frac{z-5}{5} 
%\end{align}
%%
%are coplanar.
%
%\item Find the distance of a point \myvec{2\\5\\-3} from the plane
%\begin{align}
%\myvec{6 & -3 & 2}\vec{x}=4
%\end{align}
%%
%
%\item Find the distance between the point $\vec{P}=\myvec{6\\5\\9}$ and the plane determined by the points $\bm{A}=\myvec{3\\-1\\2}, \bm{B}=\myvec{5\\2\\4}$ and $\bm{C}=\myvec{-1\\-1\\6}$.
%
%
\end{enumerate}
