\renewcommand{\theequation}{\theenumi}
\begin{enumerate}[label=\arabic*.,ref=\thesubsection.\theenumi]
\numberwithin{equation}{enumi}

\item Romila went to a stationery shop and purchased 2 pencils and 3 erasers for \rupee 9. Her friend Sonali saw the new variety of pencils and erasers with Romila, and she also bought 4 pencils and 6 erasers of the same kind for \rupee 18. Represent this situation algebraically and graphically.
\item Two rails are represented by the equations 
\begin{align}
\myvec{1 & 2}\vec{x} – 4 &= 0 \text{ and}
\\
\myvec{ 2 & 4}\vec{x} – 12 &= 0. 
\end{align}
%
Represent this situation geometrically.
%
\item Check graphically whether the pair of equations 
\begin{align}
\myvec{1 & 3}\vec{x}  &= 6 \text{ and}
\\
\myvec{ 2 & -3}\vec{x} &= 12 
\end{align}
%
is consistent. If so, solve them graphically.
%
\item Graphically, find whether the following pair of equations has no solution, unique solution or infinitely many solutions: 
%
\begin{align}
\myvec{5 & -8}\vec{x}  &= -1 \text{ and}
\\
\myvec{ 3 & -\frac{24}{5}}\vec{x} &= -\frac{3}{5}
\end{align}
%

\end{enumerate}
%
