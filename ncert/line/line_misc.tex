\renewcommand{\theequation}{\theenumi}
\begin{enumerate}[label=\arabic*.,ref=\thesubsection.\theenumi]
\numberwithin{equation}{enumi}
%
\item If 
$
\myvec{l_1\\m_1\\n_1}
$
and
$
\myvec{l_2\\m_2\\n_2}
$
are the unit direction vectors of two mutually perpendicular lines, the shown that the unit direction vector of the line perpendicular to both of these is
$
\myvec{m_1n_2-m_2n_1\\n_1l_2-n_2l_1\\l_1m_2-l_2m_1}.
$
\item A line makes angles $\alpha, \beta, \gamma, \delta$ with the diagonals of a cube, prove that \begin{align}
\cos^2\alpha + \cos^2\beta + \cos^2\gamma +\cos^2\delta = \frac{4}{3}.
\end{align}
\item Show that the lines 
\begin{align}
\frac{x-a+d}{\alpha-\delta} = \frac{y-a}{\alpha} &= \frac{z-a-d}{\alpha+\delta}, 
\\
\frac{x-b+c}{\beta-\gamma} = \frac{y-b}{\beta} &= \frac{z-b-c}{\beta+\gamma} 
\end{align}
%
are coplanar.
\item If 
\begin{align}
\vec{P} = 3\vec{a}-2\vec{b}
\\
\vec{Q} = \vec{a}+\vec{b}
\end{align}
%
find $\vec{R}$, which divides $PQ$ 
\begin{enumerate}
\item internally,
\item externally.
\end{enumerate}
\item Find $\vec{R}$ which divides the line joining the points 
\begin{align}
\vec{P} = 2\vec{a}+\vec{b}
\\
\vec{Q} = \vec{a}-\vec{b}
\end{align}
externally in the ratio $1:2$.
\item Find $\norm{\vec{a}}$ and $\norm{\vec{b}}$ if 
\begin{align}
\brak{\vec{a}+\vec{b}}^T\brak{\vec{a}-\vec{b}} &= 8
\\
\norm{\vec{a}}&=8\norm{\vec{b}}
\end{align}
\item Evaluate the product 
\begin{align}
\brak{3\vec{a}-5\vec{b}}^T\brak{2\vec{a}+7\vec{b}} 
\end{align}
\item Find $\norm{\vec{a}}$ and $\norm{\vec{b}}$, if
\begin{align}
\norm{\vec{a}} &= \norm{\vec{b}},
\\
\vec{a}^T\vec{b} = \frac{1}{2} 
\end{align}
and the angle between $\vec{a}$ and $\vec{b}$ is $60\degree$.
\item Show that 
\begin{align}
\brak{\norm{\vec{a}}\vec{b}+\norm{\vec{b}}\vec{a}}\perp \brak{\norm{\vec{a}}\vec{b}-\norm{\vec{b}}\vec{a}}
\end{align}
\item If $\vec{a}^T\vec{a}=0$ and  $\vec{a}\vec{b}=0$, what can be concluded about the vector $\vec{b}$?
\item If $\vec{a},\vec{b},\vec{c}$ are unit vectors such that 
\begin{align}
\vec{a}+\vec{b}+\vec{c} = 0,
\end{align}
find the value of 
\begin{align}
\vec{a}^T\vec{b}+\vec{b}^T\vec{c}+\vec{c}^T\vec{a}.
\end{align}
\item If $\vec{a} \ne \vec{0}, \lambda \ne 0$, then $\norm{\lambda \vec{a}} = 1$ if
\begin{enumerate}
\item $\lambda =1$
\item $\lambda = -1$
\item $\norm{\vec{a}}=\abs{\lambda}$
\item $\norm{\vec{a}}=\frac{1}{\abs{\lambda}}$
\end{enumerate}
\item If a unit vector $\vec{a}$ makes angles $\frac{\pi}{3}$ with the x-axis and $\frac{\pi}{4}$ with the y-axis and an acute angle $\theta$ with the z-axis, find $\theta$ and $\vec{a}$.
\item Show that 
\begin{align}
\brak{\vec{a}-\vec{b}}\times \brak{\vec{a}+\vec{b}} = 2\brak{\vec{a}\times\vec{b}}
\end{align}
\item If $\vec{a}^T\vec{b} = 0$ and $\vec{a}\times \vec{b}$ = 0, what can you conclude about $\vec{a}$ and $\vec{b}$?
\item Find $\vec{x}$ if  $\vec{a}$ is a unit vector such that
\begin{align}
\brak{\vec{x}-\vec{a}}^T\brak{\vec{x}+\vec{a}} = 12.
\end{align}
\item If $\norm{\vec{a}} = 3, \norm{\vec{b}} =\frac{\sqrt{2}}{3}$, then $\vec{a}\times \vec{b}$ is a unit vector if the angle between $\vec{a}$ and $\vec{b}$ is 
\begin{enumerate}[itemsep = 2pt]
\begin{multicols}{2}
\item $\frac{\pi}{6}$
\item $\frac{\pi}{4}$
\item $\frac{\pi}{3}$
\item $\frac{\pi}{2}$
\end{multicols}
\end{enumerate}
\item Prove that 
\begin{align}
\brak{\vec{a}+\vec{b}}^T\brak{\vec{a}+\vec{b}} &= \norm{\vec{a}}^2+\norm{\vec{b}}^2
\\
\iff \vec{a}&\perp\vec{b}.
\end{align}
\item If $\theta$ is the angle between two vectors $\vec{a}$ and $\vec{b}$, then $\vec{a}^T\vec{b} \ge $ only when 
\begin{enumerate}[itemsep = 2pt]
\begin{multicols}{2}
\item $0 < \theta < \frac{\pi}{2}$
\item $0 \le \theta \le \frac{\pi}{2}$
\item $0 < \theta < {\pi}$
\item $0 \le \theta \le {\pi}$
\end{multicols}
\end{enumerate}
\item Let $\vec{a}$ and $\vec{b}$ be two unit vectors and $\theta$ be the angle between them.  Then $\vec{a}+\vec{b}$ is a unit vector if 
\begin{enumerate}[itemsep = 2pt]
\begin{multicols}{2}
\item $\theta = \frac{\pi}{4}$
\item $\theta = \frac{\pi}{3}$
\item $\theta = \frac{\pi}{2}$
\item $\theta = \frac{2\pi}{3}$
\end{multicols}
\end{enumerate}
\item If $\theta$ is the angle between any two vectors $\vec{a}$ and $\vec{b}$, then 
$\norm{\vec{a}^T\vec{b}} = \norm{\vec{a} \times \vec{b}}$ when $\theta$ is equal to 
\begin{enumerate}[itemsep = 2pt]
\begin{multicols}{2}
\item 0
\item $\frac{\pi}{4}$
\item $\frac{\pi}{2}$
\item $\pi$.
\end{multicols}
\end{enumerate}
\item Let $\norm{\vec{a}} = 3, \norm{\vec{b}}= 4, \norm{\vec{c}} = 5$ such that each vector is perpendicular to the other two.  Find $\norm{\vec{a}+\vec{b}+\vec{c}}$.
\item Given 
\begin{align}
 \vec{a}+\vec{b}+\vec{c} = \vec{0}, 
\end{align}
evaluate 
\begin{align}
 \vec{a}^T\vec{b}+\vec{b}^T\vec{c}+\vec{c}^T\vec{a},
\end{align}
given that $\norm{ \vec{a}}=3, \norm{ \vec{b}}= 4$ and $\norm{ \vec{c}} = 2 $.
%
\item Find the angle between the lines whose direction vectors are $\myvec{a\\b\\c}$ and $\myvec{b-c\\c-a\\a-b}$.
\item Find the equation of a line parallel to the x-axis and passing through the origin.
\item 
%
\end{enumerate}
