\renewcommand{\theequation}{\theenumi}
\begin{enumerate}[label=\arabic*.,ref=\thesubsection.\theenumi]
\numberwithin{equation}{enumi}


\item Verify if $\vec{A} = \myvec{3\\1}, \vec{B} = \myvec{6\\4}, \vec{C} = \myvec{8\\6}$ are points on a line.
\\
\solution Refer to Problem \ref{prob:tri_exam_coll_pts}.

\item Find the condition for $\vec{x} = \myvec{x_1\\x_2}$ to be equidistant from the points $\myvec{7\\1}, \myvec{3\\5}$.
%
\\
\solution From the given information,
%
\begin{align}
\norm{\vec{x}-\myvec{7\\1}}^2&=\norm{\vec{x}-\myvec{3\\5}}^2
\end{align}
\begin{multline}
\implies \norm{\vec{x}}^2 + \norm{\myvec{7\\1}}^2-2\myvec{7&1}\vec{x} 
\\= 
 \norm{\vec{x}}^2 + \norm{\myvec{3\\5}}^2-2\myvec{3&5}\vec{x} 
\end{multline}
%
which can be simplified to obtain
\begin{align}
\label{eq:line_perp_bisect}
\myvec{1 & -1}\vec{x} = 2
\end{align}
%
which is the desired condition.  
The following code plots Fig. \ref{fig:line_perp_bisect}
%
\begin{lstlisting}
codes/line/line_perp_bisect.py
\end{lstlisting}
%
cleearly showing that \eqref{eq:line_perp_bisect} is the perpendicular bisector of $AB$.
\begin{figure}[!ht]
\includegraphics[width=\columnwidth]{./line/figs/line_perp_bisect.eps}
\caption{}
\label{fig:line_perp_bisect}
\end{figure}
%

\item Find a point on the $y$-axis which is equidistant from the points $\vec{A} = \myvec{6\\5}, \vec{B} = \myvec{-4\\3}$.
\item Draw a line segement of length 7.6 cm and divide it in the ratio $5:8$.
\\
\solution Let the end points of the line be 
\begin{align}
\vec{A} = \myvec{0\\0}, \vec{B} = \myvec{7.6\\0}
\end{align}
Then the point $\vec{C}$
\begin{align}
\vec{C} = \frac{k \vec{A} + \vec{B}}{k+1}
\end{align}
divides $AB$ in the ration $k:1$. For the given problem, $k = \frac{5}{8}$.
The following code plots Fig. \ref{fig:section}
\begin{lstlisting}
codes/line/draw_section.py
\end{lstlisting}
\begin{figure}[!ht]
\includegraphics[width=\columnwidth]{./line/figs/section.eps}
\caption{}
\label{fig:section}
\end{figure}
\item Find the coordinates of the point which divides the line segment joining the points \myvec{4\\-3} and \myvec{8\\5} in the ratio $3:1$ internally.
\item In what ratio does the point \myvec{-4\\6} divide the line segment joining the points 
%
\begin{align}
\vec{A} = \myvec{-6\\10},
\vec{B} = \myvec{3\\-8}
\end{align}
%
\item Find the coordinates of the points of trisection of the line segement joining the points
%
\begin{align}
\vec{A} = \myvec{2\\-2},
\vec{B} = \myvec{-7\\4}
\end{align}
%
\item Find the ratio in which the y-axis divides the line segment joining the points \myvec{5\\-6} and \myvec{-1\\-4}.
\item Find the value of $k$ if the points $\vec{A}=\myvec{2\\3}, \vec{B}=\myvec{4\\k}$ and $\vec{C}=\myvec{6\\-3}$ are collinear.
\item Find the direction vectors and slopes of the lines passing through the points
%
\begin{enumerate}
\item \myvec{3\\-2} and \myvec{-1\\4}.
\item \myvec{3\\-2} and \myvec{7\\-2}.
\item \myvec{3\\-2} and \myvec{3\\4}.
\item Making an inclination of $60\degree$ with the positive direction of the x-axis.
\end{enumerate}
%
\item If the angle between two lines is $\frac{\pi}{4}$ and the slope of one of the lines is $\frac{1}{4}$ find the slope of the other line.
\item The line through the points \myvec{-2\\6} and \myvec{4\\8} is perpendicular to the line through the points \myvec{8\\12} and $\myvec{x\\24}$.  Find the value of $x$.
\item Two positions of time and distance are recorded as, when $T = 0, D = 2$ and when $T = 3, D = 8$. Using the concept of slope, find law of motion, i.e., how distance depends upon time.
\item Find the equations of the lines parallel to the axes and passing through \myvec{-2\\3}.
\item Find the equation of the line through \myvec{– 2\\ 3} with slope –4.
\item Find the equations of the lines parallel to axes and passing through \myvec{– 2, 3}.
\item Write the equation of the line through the points \myvec{1\\-1} and \myvec{3\\5}.
\item Wrire the equation of the lines for which $\tan \theta = \frac{1}{2}$, where $\theta$ is the inclination of the line and 
\begin{enumerate}
\item y-intercept is $-\frac{3}{2}$
\item x-intercept is 4.
\end{enumerate}
\item Find the equation of the line, which makes intercepts -3 and 2 on the x and y axes respectively.
\item Find the equation of the line whose perpendicular distance from the origin is 4 units and the angle which the normal makes with the positive direction of x-axis is $15\degree$.
\item The Farenheit temperature $F$ and absolute temperature $K$ satisfy a linear equation.  Given $K=273$ when $F=32$ and that $K=373$  when $F=212$, express $K$ in terms of $F$ and find the value of $F$, when $K=0$.
\item Equation of a line is 
\begin{align}
\myvec{3 & – 4} + 10 = 0. 
\end{align}
Find its 
\begin{enumerate}
\item  slope, 
\item  x - and y-intercepts.
\end{enumerate}
\item Find the angle between the lines 
\begin{align}
\myvec{1 & – \sqrt{3}}\vec{x}  = 5
\\
\myvec{\sqrt{3} & –1}\vec{x}  = -6
. 
\end{align}
\item Find the equation of a line perpendicular to the line 
\begin{align}
\myvec{1 & – 2}\vec{x}  = 3
\end{align}
%
and passes through the point \myvec{1\\-2}.
\item Find the distance of the point \myvec{3\\-5} from the line 
\begin{align}
\myvec{3 & – 4}\vec{x}  = 26
\end{align}
\item If the lines 
\begin{align}
\myvec{2 & 1}\vec{x}  = 3
\\
\myvec{5 & k}\vec{x}  = 3
\\
\myvec{3 & 1}\vec{x}  = 2
\end{align}
%
are concurrent, find the value of $k$.
%
\item Find the distance of the line
\begin{align}
\myvec{4 & 1}\vec{x}  = 0
\end{align}
%
from the point \myvec{4\\1} measured along the line making an angle of $135\degree$ with the positive x-axis.
\item Assuming that straight lines work as a plane mirror for a point, find the image of the point \myvec{1\\2} in the line 
%
\begin{align}
\myvec{1 & -3}\vec{x}  = -4.
\end{align}
%
\item A line is such that its segment between the lines %
\begin{align}
\myvec{5 & -1}\vec{x}  &= -4
\\
\myvec{3 & 4}\vec{x}  &= 4
\end{align}
%
is bisected at the point \myvec{1\\5}.  Obtain its equation.
%
\item Show that the path of a moving point such that its distances from two lines
%
\begin{align}
\myvec{3 & -2}\vec{x}  &= 5
\\
\myvec{3 & 2}\vec{x}  &= 5
\end{align}
%
are  equal is a straight line.
%
\item Find the distance between the points
%
\begin{align}
\vec{P} = \myvec{1\\-3\\4},
\vec{Q} = \myvec{-4\\1\\2}
\end{align}
%
\item Show that the points 
$
\vec{A}=\myvec{-2\\3\\5}, 
\vec{B}=\myvec{1\\2\\3}$ 
and 
$ \vec{C}=\myvec{7\\0\\-1}$ 
are collinear.
%
\item Find the equation of set of points $\vec{P}$ such that
\begin{align}
PA^2+PB^2 =2k^2,
\end{align}
%
\begin{align}
\vec{A} = \myvec{3\\4 \\5},
\vec{B} = \myvec{-1\\3 \\-7},
\end{align}
%
respectively.
%
\item Find the coordinates of a point which divides the line segment joining the points \myvec{1\\-2\\3} and \myvec{3\\4\\-5} in the ratio $2:3$
\begin{enumerate}
\item internally, and
\item externally.
\end{enumerate}
%
\item Using section formular, prove that the three points \myvec{-4\\6\\10}, \myvec{2\\4\\6} and \myvec{14\\0\\-2} are collinear.
%
\item Find the ratio in which the line segment joining the points \myvec{4\\8\\10} and \myvec{6\\10\\-8} is divided by the YZ-plane.
%
\item Find the equation of the set of points $\vec{P}$ such that its distances from the points
$
\vec{A}=\myvec{3\\4\\-5}, 
\vec{B}=\myvec{-2\\1\\4}
$ 
%
\item Find the values of $x, y, z$ such that 
\begin{align}
\myvec{x\\2\\z}= \myvec{2\\y\\1}
\end{align}
%
\item If
\begin{align}
\vec{a} = \myvec{1\\2}, \vec{b} = \myvec{2\\1},
\end{align}
verify if  
\begin{enumerate}
\item $\norm{\vec{a}}=\norm{\vec{b}}$

\item $\vec{a}=\vec{b}$
\end{enumerate}
%
\item Find a unit vector in the  direction of \myvec{2\\3\\1}.
%
\item Find a vector $\vec{x}$ in the direction of \myvec{1\\-2} such that $\norm{\vec{x}} = 7$.
%
\item Find a unit vector in the direction of $\vec{a}+\vec{b}$, where 
%
\begin{align}
\vec{a} = \myvec{2\\2\\-5}, \vec{b} = \myvec{2\\1\\3}.
\end{align}
%
\item Find a unit vector in the direction of 
%
\begin{align}
\myvec{1\\1\\-2}.
\end{align}
%
\item Find the direction vector of $PQ$, where 
\begin{align}
\vec{P} = \myvec{2\\3\\0},
\vec{Q} = \myvec{-1\\-2\\-4}
\end{align}
%
\item If 
\begin{align}
\vec{P} = 3\vec{a}-2\vec{b}
\\
\vec{Q} = \vec{a}+\vec{b}
\end{align}
%
find $\vec{R}$, which divides $PQ$ 
\begin{enumerate}
\item internally,
\item externally.
\end{enumerate}
%
\item Find the angle between two vectors $\vec{a}$ and $\vec{b}$ where 
%
\begin{align}
\norm{\vec{a}} = 1,
\norm{\vec{b}} = 2,
\vec{a}^T\vec{b} = 1.
\end{align}
%
\item Find the angle between the vectors 
$\vec{a}=\myvec{1\\1\\-1}$
  and 
$\vec{b}=\myvec{1\\-1\\1}$.
\item If 
$\vec{a}=\myvec{5\\-1\\-3}$
  and 
$\vec{b}=\myvec{1\\3\\-5}$,
%
then show that the vectors $\vec{a}+\vec{b}$ and $\vec{a}-\vec{b}$ are perpendicular.
%
\item Find the projection of the vector 
\begin{align}
\vec{a} = \myvec{2\\3\\2}
\end{align}
on the vector
\begin{align}
\myvec{1\\2\\1}.
\end{align}
%
\item Find $\norm{\vec{a}-\vec{b}}$, if 
\begin{align}
\norm{\vec{a}} = 2, 
\norm{\vec{b}} = 3,
\vec{a}^T\vec{b} = 4.
\end{align}
%
\item If $\vec{a}$ is a unit vector and 
%
\begin{align}
\brak{\vec{x}-\vec{a}}\brak{\vec{x}+\vec{a}} = 8, 
\end{align}
%
then find $\vec{x}$.
%
\item Given
\begin{align}
\vec{a}=\myvec{2\\1\\3},
\vec{b}=\myvec{3\\5\\-2},
\end{align}
find $\norm{\vec{a} \times \vec{b}}$.
%
\item Find a unit vector perpendicular to each of the vectors
$\vec{a}+\vec{b}$ and $\vec{a}-\vec{b}$, where 
\begin{align}
\vec{a}=\myvec{1\\1\\1},
\vec{b}=\myvec{1\\2\\3}.
\end{align}
%
\item Show that 
$\vec{A}=\myvec{1\\1\\1}, \vec{B}=\myvec{2\\5\\0}, \vec{C}=\myvec{3\\2\\-3}$  and $ \vec{D}=\myvec{1\\-6\\-1}$, are collinear.
%
\item Let $\norm{\vec{a}} = 3, \norm{\vec{b}}= 4, \norm{\vec{c}} = 5$ such that each vector is perpendicular to the other two.  Find $\norm{\vec{a}+\vec{b}+\vec{c}}$.
%
\item Given 
\begin{align}
 \vec{a}+\vec{b}+\vec{c} = \vec{0}, 
\end{align}
evaluate 
\begin{align}
 \vec{a}^T\vec{b}+\vec{b}^T\vec{c}+\vec{c}^T\vec{a},
\end{align}
given that $\norm{ \vec{a}}=3, \norm{ \vec{b}}= 4$ and $\norm{ \vec{c}} = 2 $.
%
\item Let $\bm{\alpha} = \myvec{3\\-1\\0}, \bm{\beta} = \myvec{2\\1\\-3}$.  Find $\bm{\beta}_1, \bm{\beta}_2 $ such that $\bm{\beta}=\bm{\beta}_1+\bm{\beta}_2, \bm{\beta}_1 \parallel  \bm{\alpha} $ and $\bm{\beta}_2 \perp \bm{\alpha} $.
%
\item Find a unit vector that makes an angle of $90\degree, 60\degree$ and $30\degree$ with the positive x, y and z axis respectively.
%
\item Find a unit vector in the direction of \myvec{2\\-1\\-2}.
%
\item Find a unit vector in the direction of the line passing through \myvec{-2\\4\\-5} and $\myvec{1\\2\\3}$.
%
\item Show that 
$
\vec{A}=\myvec{2\\3\\-4}, 
\vec{B}=\myvec{1\\-2\\3} \text{ and } 
\vec{C}=\myvec{3\\8\\-11}$  
are collinear.
%
\item Find the equation of a line through the point \myvec{5\\2\\-4} and parallel to the vector \myvec{3\\2\\-8}.
\item Find the equation of a line passing through the points \myvec{-1\\0\\2} and \myvec{3\\4\\6}.
\item If
\begin{align}
\frac{x+3}{2} = \frac{y-5}{4} = \frac{z+6}{2}, 
\end{align}
%
find the equation of the line.
%
\item Find the angle between the pair of lines given by 
\begin{align}
\bm{x} &= \myvec{3\\2\\-4} + \lambda_1\myvec{1 \\ 2 \\2}
\\
\bm{x} &= \myvec{5\\-2\\0} + \lambda_2\myvec{3 \\ 2 \\6}
\end{align}
%
\item Find the angle between the pair of lines
\begin{align}
\frac{x+3}{3} = \frac{y-1}{5} &= \frac{z+3}{4}, 
\\
\frac{x+1}{1} = \frac{y-4}{1} &= \frac{z-5}{2} 
\end{align}
%
\item Find the shortest distance between the lines 
\begin{align}
L_1: \quad \bm{x} &= \myvec{1\\1\\0} + \lambda_1\myvec{2 \\ -1 \\1}
\\
L_2: \quad \bm{x} &= \myvec{2\\1\\-1} + \lambda_2\myvec{3 \\ -5 \\2}
\end{align}
\item Find the 
distance between the lines 
\begin{align}
L_1: \quad \bm{x} &= \myvec{1\\2\\-4} + \lambda_1\myvec{2 \\ 3 \\6}
\\
L_2: \quad \bm{x} &= \myvec{3\\3\\-5} + \lambda_2\myvec{2 \\ 3 \\6}
\end{align}
%
\item Find the equation of a plane which is at a distance of $\frac{6}{\sqrt{29}}$ from the origin and has  normal vector \myvec{2\\-3\\4}.
\item Find the unit normal vector of the plane 
\begin{align}
\myvec{6 & -3 & -2}\bm{x}  = 1.
\end{align}
\item Find the distance of the plane 
\begin{align}
\myvec{2 & -3 & 4}\bm{x}-6  = 0
\end{align}
%
from the origin.
\item Find the coordinates of the foot of the perpendicular drawn from the origin to the plane 
\begin{align}
\myvec{2 & -3 & 4}\bm{x}-6  = 0
\end{align}
%
\item Find the equation of the plane which passes through the point \myvec{5\\2\\-4} and perpendicular to the line with direction vector \myvec{2\\3\\-1}.
\item Find the equation of the plane passing through 
$
\bm{R} = \myvec{2\\5\\-3},
\bm{S}= \myvec{-2\\-3\\5}
$ 
and 
$
\bm{T}= \myvec{5\\3\\-3}.
$
\item Find the equation of the plane with intercepts 2, 3 and 4 on the x, y and z axis respectively.
\item Find the equation of the plane passing through the intersection of the planes 
\begin{align}
\myvec{1 & 1 & 1}\bm{x}&=6  
\\
\myvec{2 & 3 & 4}\bm{x}&=-5
\end{align}
%
and the point \myvec{1\\1\\1}.
\item Show that the lines 
\begin{align}
\frac{x+3}{-3} = \frac{y-1}{1} &= \frac{z-5}{5}, 
\\
\frac{x+1}{-1} = \frac{y-2}{2} &= \frac{z-5}{5} 
\end{align}
%
are coplanar.
\item Find the angle between the two planes
\begin{align}
\myvec{2 & 1 & -2}\bm{x}&=5
\\
\myvec{3 &-6 & -2}\bm{x}&=7.
\end{align}
%
\item Find the angle between the two planes
\begin{align}
\myvec{2 & 2 & -2}\bm{x}&=5
\\
\myvec{3 &-6 & 2}\bm{x}&=7.
\end{align}
%
Find the distance of a point \myvec{2\\5\\-3} from the plane
\begin{align}
\myvec{6 & -3 & 2}\bm{x}=4
\end{align}
%
Find the angle between the line 
%
\begin{align}
\frac{x+1}{2} = \frac{y}{3} = \frac{z-3}{6} 
\end{align}
%
and
%
the plane 
\begin{align}
\myvec{10 & 2 & -11}\bm{x}=3
\end{align}
%
\item Find the equation of the plane that contains the point $\myvec{1\\-1\\2}$ and is perpedicular to each of the planes
\begin{align}
\myvec{2 & 3 & -2}\bm{x}&=5
\\
\myvec{1 & 2 & -3}\bm{x}&=8
\end{align}
%
\item Find the distance between the point $\bm{P}=\myvec{6\\5\\9}$ and the plane determined by the points $\bm{A}=\myvec{3\\-1\\2}, \bm{B}=\myvec{5\\2\\4}$ and $\bm{C}=\myvec{-1\\-1\\6}$.
\item Find the coordinates of the point where the lines through the points
$
\bm{A}=\myvec{3\\4\\1}, 
\bm{B}=\myvec{5\\1\\6}
$
crosses the XY plane.
%
\end{enumerate}
%
