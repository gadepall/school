\renewcommand{\theequation}{\theenumi}
\begin{enumerate}[label=\arabic*.,ref=\thesubsection.\theenumi]
\numberwithin{equation}{enumi}
%\chapter{The Optimum Receiver}
\item Angles opposite to equal sides of a triangle are equal. 
\label{prob:tri_ang_side_eq}
\\
\solution Using the sine formula in \eqref{eq:tri_sin_form},%
\begin{align}
\frac{\sin A}{a} = \frac{\sin B}{b}
\end{align}
%
Thus, if $A=B$, $\sin A = \sin B \implies a =b$.
\item  Sides opposite to equal angles of a triangle are equal. 
%\\
%\solution Use \eqref{eq:tri_sin_form} and the argument in Problem \ref{prob:tri_ang_side_eq}
%
\item  Each angle of an equilateral triangle is of 60$\degree$. 
%\\
%\solution In an equilateral $\triangle$, 
%%
%\begin{align}
%A=B=C.&
%\\
%\because A+B+C = 180\degree, 3A = 180\degree&
%\\
%\implies A = 60\degree&
%\end{align}
%


%\subsection{Problem}
\item Triangles on the same base (or equal bases) and between the same parallels are equal in area.
\item Triangles on the same base (or equal bases) and having equal areas lie between the same parallels.
\item In $\triangle ABC, D, E$ and $F$ are respectively the mid-points of sides $AB, BC$ and $CA $. Show that $\triangle ABC$ is divided into four congruent triangles by joining $D, E$ and $F$.
\item  The line-segment joining the mid-points of any two sides of a triangle is parallel to the third side and is half of it.
\label{prob:tri_mid_similar}
%\label{prob:quad_similar}
%%
%\\
%\solution If $DE$ is the lie joining he mid points of $\triangle ABC$,  use cosine formula to find the lengths of $DE$ and $BC$. Then use cosine formula to show that all angles of $\triangle ADE$ are equal to the corresponding angles of $\triangle ABC$.
%
\item  A line through the mid-point of a side of a triangle parallel to another side bisects the third side.
%\\
%\solution Use cosine formula.
\item ABC is a triangle right angled at $C$. A line through the mid-point $M$ of hypotenuse $AB$ and parallel to $BC$ intersects $AC$ at $D$. Show that (i) $D$ is the mid-point of $AC$
(ii) $MD \perp AC$ (iii) $CM = MA = \frac{1}{2}AB$

\end{enumerate}


