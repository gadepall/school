\renewcommand{\theequation}{\theenumi}
\begin{enumerate}[label=\arabic*.,ref=\thesubsection.\theenumi]
\numberwithin{equation}{enumi}
%\chapter{The Optimum Receiver}
%\item Angles opposite to equal sides of a triangle are equal. 
%\label{prob:tri_ang_side_eq}
%\\
%\solution Using the sine formula in \eqref{eq:tri_sin_form},%
%\begin{align}
%\frac{\sin A}{a} = \frac{\sin B}{b}
%\end{align}
%%
%Thus, if $A=B$, $\sin A = \sin B \implies a =b$.
\item  Sides opposite to equal angles of a triangle are equal. 
%\\
%\solution Use \eqref{eq:tri_sin_form} and the argument in Problem \ref{prob:tri_ang_side_eq}
%
\item  Each angle of an equilateral triangle is of 60$\degree$. 
%\\
%\solution In an equilateral $\triangle$, 
%%
%\begin{align}
%A=B=C.&
%\\
%\because A+B+C = 180\degree, 3A = 180\degree&
%\\
%\implies A = 60\degree&
%\end{align}
%


%\subsection{Problem}
\item Triangles on the same base (or equal bases) and between the same parallels are equal in area.
\item Triangles on the same base (or equal bases) and having equal areas lie between the same parallels.
\item In $\triangle ABC$, the bisector $AD$ of $\angle  A$ is perpendicular to side $BC$. Show that $AB = AC$ and $\triangle ABC$ is isosceles.
\item $E$ and $F$ are respectively the mid-points of equal sides $AB$ and AC of $\triangle ABC$. Show that $BF = CE$. 
\item In an isosceles $\triangle ABC$ with $AB$ = AC, D and E are points on $BC$ such that $BE = CD$. Show that $AD = AE$. 
%
\item $AB$ is a line-segment. $P$ and $Q$ are points on opposite sides of $AB$ such that each of them is equidistant from the points $A$ and $B$. Show that the line $PQ $ is the perpendicular bisector of $AB$.
%
\item $P$ is a point equidistant from two lines $l$ and $m$ intersecting at point $A$.  Show that the line  $AP$  bisects the angle between them.
%
\item $D$ is a point on side $BC$ of $\triangle  ABC$ such that $AD = AC$. Show that $AB > AD$

%
\item $AB$ is a line segment and line $l$ is its perpendicular bisector. If a point $P$ lies on $l$, show that $P$ is equidistant from $A$ and $B$.
\item Line-segment $AB$ is parallel to another line-segment $CD$. $O$ is the mid-point of $AD$. Show that 
\begin{enumerate}
\item  $\triangle AOB \cong \triangle DOC$ 
\item  $O$ is also the mid-point of $BC$.
\end{enumerate}
%
\item In quadrilateral $ACBD, AC = AD$ and $AB$ bisects $\angle  A$. Show that $\triangle  ABC \cong \triangle  ABD$. What can you say about $BC$ and $BD$?
%
\item $ABCD$ is a quadrilateral in which $AD = BC$ and $\angle  DAB = \angle  CBA$ . Prove that
\begin{enumerate}
\item  $\triangle  ABD \cong  \triangle  BAC $
\item $ BD = AC $
\item  $\angle  ABD = \angle  BAC$.
\end{enumerate}
%
\item $l$ and $m$ are two parallel lines intersected by another pair of parallel lines p and q 
to form the quadrilateral $ABCD$. Show that $\triangle  ABC \cong  \triangle  CDA$.
%
\item Line $l$ is the bisector of $ \angle  A$ and $B$ is any point on $l$. $BP$ and $BQ$ are perpendiculars from $B$ to the arms of $\angle  A$ (see Fig. 7.20). Show that: 
\begin{enumerate}
\item  $\triangle  APB \cong  \triangle  AQB$ 
\item  $BP = BQ$ or $B$ is equidistant from the arms of $\angle  A$.
\end{enumerate}
%
\item $ABCE$ is a quadrilateral and $D$ is a point on $BC$ such that, $AC = AE, AB = AD$ and $\angle  BAD = \angle  EAC$. Show that $BC = DE$.
%
\item In right triangle $ABC$, right angled at $C, M$ is the mid-point of hypotenuse $AB$. $C$ is joined to $M$ and produced to a point $D$ such that $DM = CM$. Point $D$ is joined to point $B$.
Show that: 
\begin{enumerate}
\item $ \triangle  AMC \cong  \triangle  BMD $
\item $\angle  DBC$ is a right angle. 
\item $\triangle  DBC \cong  \triangle  ACB$
\item $ CM = \frac{1}{ 2} AB$
\end{enumerate}
%
\item In an isosceles $\triangle ABC$, with $AB = AC$, the bisectors of $\angle B$ and $\angle C$ intersect each other at $O$. Join $A$ to $O$. Show that :
\begin{enumerate} 
\item $OB = OC$ 
\item $AO$ bisects $\angle A$
\end{enumerate}
\item In $\triangle ABC$, $AD$ is the perpendicular bisector of $BC$. Show that $\triangle ABC$ is an isosceles triangle in which $AB = AC$.
\item $ABC$ is an isosceles triangle in which altitudes $BE$ and $CF$ are drawn to equal sides $AC$ and $AB$ respectively . Show that these altitudes are equal.
%
\item $ABC$ is a triangle in which altitudes $BE$ and $CF$ to sides $AC$ and $AB$ are equal. Show that
%
\begin{enumerate} 
\item $\triangle  ABE \cong  \triangle  ACF $
\item  $AB = AC$, i.e., $ABC$ is an isosceles triangle.
\end{enumerate}
%
\item $ABC$ and $DBC$ are two isosceles triangles on the same base $BC$. Show that $\angle ABD = \angle ACD$.
%
\item  $\triangle  ABC$ and $\triangle  DBC$ are two isosceles triangles on the same base $BC$ and vertices $A$ and $D$ are on the same side of $BC$. If $AD$ is extended to intersect $BC$ at $P$, show that
\begin{enumerate}
\item $\triangle  ABD \cong  \triangle  ACD $
\item $\triangle  ABP \cong  \triangle  ACP $
\item $AP$ bisects $\angle  A$ as well as $\angle  D$. 
\item $AP$ is the perpendicular bisector of $BC$.
\end{enumerate}
\item $AD$ is an altitude of an isosceles $\triangle ABC$ in which $AB = AC$. Show that 
\begin{enumerate}
\item $AD$ bisects $BC$
\item $AD$ bisects $\angle  A$. 
\end{enumerate}

\item  Two sides $AB$ and $BC$ and median $AM$ of one triangle $ABC$ are respectively equal to sides $PQ$ and $QR$ and median $PN$ of $\triangle  PQR$. Show that: 
\begin{enumerate}
\item $\triangle  ABM \cong  \triangle  PQN $
\item $\triangle  ABC \cong  \triangle  PQR$
\end{enumerate}
\item  $BE$ and $CF$ are two equal altitudes of a triangle $ABC$. Using RHS congruence rule, prove that the triangle $ABC$ is isosceles.
\item  $ABC$ is an isosceles triangle with $AB = AC$. Draw $AP \perp BC$ to show that $\angle  B = \angle  C$.
%
\item $\triangle ABC$ is an isosceles triangle in which $AB = AC$. Side $BA$ is produced to $D$ such that $AD = AB$. Show that $\angle BCD$ is a right angle.
%
\item $ABC$ is a right angled triangle in which $\angle A$ = 90$\degree$ and $AB = AC$. Find $\angle B$ and $\angle C$.
%
\item Show that in a right angled triangle, the hypotenuse is the longest side.
\item Sides AB and AC of $\triangle  ABC$ are extended to points P and Q respectively. Also, $\angle  PBC < \angle  QCB$. Show that $AC > AB$.

\item Line segments $AD$ and $BC$ intersect at $O$ and form $\triangle OAB$ and $\triangle ODC$. $\angle  B < \angle  A$ and $\angle  C < \angle  D$. Show that $AD < BC$.

\item $AB$ and $CD$ are respectively the smallest and longest sides of a quadrilateral $ABCD$. Show that $\angle  A > \angle  C$ and $\angle  B > \angle  D$.
%
\item In $\triangle PQR,  PR > PQ$ and $PS$ bisects $\angle  QPR$. Prove that $\angle  PSR > \angle  PSQ$.
%
\item Show that of all line segments drawn from a given point not on it, the perpendicular line segment is the shortest.
%


\end{enumerate}


