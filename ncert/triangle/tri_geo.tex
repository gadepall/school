\renewcommand{\theequation}{\theenumi}
\begin{enumerate}[label=\arabic*.,ref=\thesubsection.\theenumi]
\numberwithin{equation}{enumi}

\item Find the area of a triangle whose vertices are \myvec{1\\-1}, \myvec{-4\\6} and \myvec{-3\\-5}.
\item Find the area of a triangle formed by the vertices $\vec{A}=\myvec{5\\2}, \vec{B}=\myvec{4\\7}, \vec{C}=\myvec{7\\-4}$.
\item Find the area of a triangle formed by the points $\vec{P}=\myvec{-1.5\\3}, \vec{Q}=\myvec{6\\-2}, \vec{R}=\myvec{-3\\4}$.
\item Find the area of the triangle whose vertices are
\begin{enumerate}
\item \myvec{2\\3}, \myvec{-1\\0},  \myvec{2\\-4}
\item  \myvec{-5\\-1},  \myvec{3\\-5},  \myvec{5\\2}
\end{enumerate}
\item Find the area of the triangle formed by joining the mid points o the sides of a triangle whose vertices are  \myvec{0\\-1},  \myvec{2\\1},  \myvec{0\\3}.
\item Verify that the median of $\triangle ABC$ with vertices $\vec{A}=\myvec{4\\-6},  \vec{B}=\myvec{3\\-2}$ and  $\vec{C} =  \myvec{5\\2}$ divides it into two triangles of equal areas.
\item The vertices of $\triangle ABC$ are $\vec{A}=\myvec{4\\6},  \vec{B}=\myvec{1\\5}$ and  $\vec{C} =  \myvec{7\\2}$.  A line is drawn to intersect sides $AB$ and $AC$ at $D$ and $E$ respectively, such that
\begin{align}
\frac{AD}{AB}=\frac{AE}{AC}= \frac{1}{4}
\end{align}
%
Find 
\begin{align}
\frac{\text{area of }\triangle ADE}{\text{area of }\triangle ABC}.
\end{align}
\item Let $\vec{A}=\myvec{4\\2},  \vec{B}=\myvec{6\\5}$ and  $\vec{C} =  \myvec{1\\4}$ be the vertices of $\triangle ABC$.
\begin{enumerate}
\item The median from $\vec{A}$ meets $BC$ at $\vec{D}$.  Find the coordinates of the point $\vec{D}$.
\item Find the coordinates of the point $\vec{P}$ on $AD$ such that $AP:PD = 2:1$.
\item Find the coordinates of the points $\vec{Q}$ and $\vec{R}$ on medians $BE$ and $CF$ respectively such that $BQ:QE = 2:1$ and $CR:RF = 2:1$.
\end{enumerate}
\item In $\triangle ABC$, Show that the centroid 
\begin{align}
\vec{O} = \frac{\vec{A}+\vec{B}+\vec{C}}{3}
\end{align}
\item Show that the points 
\begin{align}
\vec{A} = \myvec{2\\-1 \\1},
\vec{B} = \myvec{1\\-3 \\-5},
\vec{C} = \myvec{3\\ -4\\-4}
\end{align}
%
are the vertices of a right angled triangle.
\end{enumerate}
%
