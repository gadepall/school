Let X be the random variable representing the outcome when the dice is thrown
\begin{align}
X \epsilon \{1,2,3,4,5,6\}
\end{align}
Since all events are equally likely,
\begin{align}
\pr{X=x}=\begin{cases}
	\frac{1}{6}& x=1,2,3,4,5,6\\
	0 & otherwise
	\end{cases}
\end{align}
\brak{i}The probability that the outcome is a prime number is 
\begin{align}
\pr{X=2}+\pr{X=3}+\pr{X=5} \\
\implies 3 \times \frac{1}{6}=\frac{1}{2}
\end{align}

\brak{ii}Probability of occurance of number between 2 and 6 is
\begin{align}
\pr{X=3}+\pr{X=4}+\pr{X=5}
 \\
\implies 3 \times \frac{1}{6}=\frac{1}{2}
\end{align}
\brak{iii}Probability of occurance of odd number is
\begin{align}
\pr{X=1}+\pr{X=3}+\pr{X=5}
 \\
\implies 3 \times \frac{1}{6}=\frac{1}{2}
\end{align}

The python code for the distribution is
\begin{lstlisting}
./prob/codes/dice123.py
\end{lstlisting}
The above code checks number of times each of the above events occur when the dice is thrown 100,000 times.
