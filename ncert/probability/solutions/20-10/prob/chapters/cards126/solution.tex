(i)The sample size is equal to number of cards 
\begin{align}
S=5
\end{align}
number of queens in the cards are 
\begin{align}
Q=1
\end{align}
The probability that a queen is picked is 
\begin{align}
\pr{Q} = \frac{Q}{S} &= \frac{1}{5}
\end{align}
(ii) After a queen is drawn and put aside, the new sample space is
\begin{align}
S'=4
\end{align}
(a)number of aces in the remaining cards are 
\begin{align}
A=1
\end{align}
The probability that an ace is picked is 
\begin{align}
\pr{A} = \frac{A}{S'} &= \frac{1}{4}
\end{align}

(a)number of queens in the remaining cards are 
\begin{align}
Q'=0
\end{align}
The probability that a queen is picked from the remaining cards is 
\begin{align}
\pr{Q'} = \frac{Q'}{S'} &= 0
\end{align}

The python code below calculates the above probabilities for 100000 picks
\begin{lstlisting}
./prob/codes/cards126.py
\end{lstlisting}
