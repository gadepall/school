\documentclass[journal,12pt,twocolumn]{IEEEtran}
%
\usepackage{setspace}
\usepackage{gensymb}
\usepackage{xcolor}
%\usepackage{esint}
\usepackage{caption}
%\usepackage{subcaption}
%\doublespacing
\singlespacing
\usepackage{multicol}

\usepackage{iithtlc}
%\usepackage{graphicx}
%\usepackage{amssymb}
%\usepackage{relsize}
\usepackage[cmex10]{amsmath}
\usepackage{mathtools}
%\usepackage{amsthm}
%\interdisplaylinepenalty=2500
%\savesymbol{iint}
%\usepackage{txfonts}
%\restoresymbol{TXF}{iint}
%\usepackage{wasysym}
\usepackage{amsthm}
\usepackage{mathrsfs}
\usepackage{txfonts}
\usepackage{stfloats}
\usepackage{cite}
\usepackage{cases}
\usepackage{subfig}
%\usepackage{xtab}
\usepackage{longtable}
\usepackage{multirow}
%\usepackage{algorithm}
%\usepackage{algpseudocode}
\usepackage{enumerate}
\usepackage{mathtools}
%\usepackage{stmaryrd}

\usepackage{listings}
%    \usepackage[latin]{inputenc}                                 %%
    \usepackage{color}                                            %%
    \usepackage{array}                                            %%
    \usepackage{longtable}                                        %%
    \usepackage{calc}                                             %%
    \usepackage{multirow}                                         %%
    \usepackage{hhline}                                           %%
    \usepackage{ifthen}                                           %%
  %optionally (for landscape tables embedded in another document): %%
    \usepackage{lscape}   
    \usepackage{tikz}
\usetikzlibrary{calc}
\usepackage{tkz-euclide}
\usetkzobj{all}
  
%\usepackage{tikz}
%\usepackage{tkz-euclide}
%\usetkzobj{all}
%%\usepackage[utf8]{inputenc}
%\usepackage{rotating}
%\usetikzlibrary{arrows,shapes,decorations.markings,positioning}
%\usepackage{amsmath}
%\usepackage{ulem}
%\usepackage{soul}

%\usepackage{wasysym}
%\newcounter{MYtempeqncnt}
\DeclareMathOperator*{\Res}{Res}
%\renewcommand{\baselinestretch}{2}
\renewcommand\thesection{\arabic{section}}
\renewcommand\thesubsection{\thesection.\arabic{subsection}}
\renewcommand\thesubsubsection{\thesubsection.\arabic{subsubsection}}

\renewcommand\thesectiondis{\arabic{section}}
\renewcommand\thesubsectiondis{\thesectiondis.\arabic{subsection}}
\renewcommand\thesubsubsectiondis{\thesubsectiondis.\arabic{subsubsection}}

%\DeclareUnicodeCharacter{01B5}{\zst}
%\DeclareRobustCommand\zst{%
%\unskip\nobreak\thinspace\zbar\allowbreak\thinspace\ignorespaces}
%\newcommand{\zbar}{\raisebox{0.2ex}{--}\kern-0.6em Z}


% correct bad hyphenation here
\hyphenation{op-tical net-works semi-conduc-tor}

\def\inputGnumericTable{}  

\lstset{
language=python,
frame=single, 
breaklines=true
}

\begin{document}
%

\theoremstyle{definition}
\newtheorem{theorem}{Theorem}[section]
\newtheorem{problem}{Problem}
\newtheorem{proposition}{Proposition}[section]
\newtheorem{lemma}{Lemma}[section]
\newtheorem{corollary}[theorem]{Corollary}
\newtheorem{example}{Example}[section]
\newtheorem{definition}{Definition}[section]
%\newtheorem{algorithm}{Algorithm}[section]
%\newtheorem{cor}{Corollary}
\newcommand{\BEQA}{\begin{eqnarray}}
\newcommand{\EEQA}{\end{eqnarray}}
\newcommand{\define}{\stackrel{\triangle}{=}}

\bibliographystyle{IEEEtran}
%\bibliographystyle{ieeetr}



\providecommand{\pr}[1]{\ensuremath{\Pr\left(#1\right)}}
\providecommand{\qfunc}[1]{\ensuremath{Q\left(#1\right)}}
\providecommand{\sbrak}[1]{\ensuremath{{}\left[#1\right]}}
\providecommand{\lsbrak}[1]{\ensuremath{{}\left[#1\right.}}
\providecommand{\rsbrak}[1]{\ensuremath{{}\left.#1\right]}}
\providecommand{\brak}[1]{\ensuremath{\left(#1\right)}}
\providecommand{\lbrak}[1]{\ensuremath{\left(#1\right.}}
\providecommand{\rbrak}[1]{\ensuremath{\left.#1\right)}}
\providecommand{\cbrak}[1]{\ensuremath{\left\{#1\right\}}}
\providecommand{\lcbrak}[1]{\ensuremath{\left\{#1\right.}}
\providecommand{\rcbrak}[1]{\ensuremath{\left.#1\right\}}}
\theoremstyle{remark}
\newtheorem{rem}{Remark}
\newcommand{\sgn}{\mathop{\mathrm{sgn}}}
\providecommand{\abs}[1]{\left\vert#1\right\vert}
\providecommand{\res}[1]{\Res\limits_{#1}} 
\providecommand{\norm}[1]{\lVert#1\rVert}
\providecommand{\mtx}[1]{\mathbf{#1}}
\providecommand{\mean}[1]{E\left[ #1 \right]}
\providecommand{\fourier}{\overset{\mathcal{F}}{ \rightleftharpoons}}
%\providecommand{\hilbert}{\overset{\mathcal{H}}{ \rightleftharpoons}}
\providecommand{\system}{\overset{\mathcal{H}}{ \longleftrightarrow}}
\providecommand{\gauss}[2]{\mathcal{N}\ensuremath{\left(#1,#2\right)}}
	%\newcommand{\solution}[2]{\textbf{Solution:}{#1}}
\newcommand{\solution}{\noindent \textbf{Solution: }}
\providecommand{\dec}[2]{\ensuremath{\overset{#1}{\underset{#2}{\gtrless}}}}
%\numberwithin{equation}{section}
%\numberwithin{problem}{section}
\makeatletter
\@addtoreset{figure}{problem}
\makeatother

\let\StandardTheFigure\thefigure
%\renewcommand{\thefigure}{\theproblem.\arabic{figure}}
\renewcommand{\thefigure}{\theproblem}

\def\putbox#1#2#3{\makebox[0in][l]{\makebox[#1][l]{}\raisebox{\baselineskip}[0in][0in]{\raisebox{#2}[0in][0in]{#3}}}}
     \def\rightbox#1{\makebox[0in][r]{#1}}
     \def\centbox#1{\makebox[0in]{#1}}
     \def\topbox#1{\raisebox{-\baselineskip}[0in][0in]{#1}}
     \def\midbox#1{\raisebox{-0.5\baselineskip}[0in][0in]{#1}}


% paper title
% can use linebreaks \\ within to get better formatting as desired
\title{
\logo{
JEE 2018 through Python
}
}
%
%
% author names and IEEE memberships
% note positions of commas and nonbreaking spaces ( ~ ) LaTeX will not break
% a structure at a ~ so this keeps an author's name from being broken across
% two lines.
% use \thanks{} to gain access to the first footnote area
% a separate \thanks must be used for each paragraph as LaTeX2e's \thanks
% was not built to handle multiple paragraphs
%

%\author{Y Aditya, A Rathnakar and G V V Sharma$^{*}$% <-this % stops a space
%\author{G V V Sharma$^{*}$% <-this % stops a space
%\thanks{*The author is with the Department
%of Electrical Engineering, Indian Institute of Technology, Hyderabad
%502205 India e-mail:  gadepall@iith.ac.in.All content in this manual is released under GNU GPL.  Free and open source.}% <-this % stops a space
%%\thanks{J. Doe and J. Doe are with Anonymous University.}% <-this % stops a space
%%\thanks{Manuscript received April 19, 2005; revised January 11, 2007.}}
%}
% note the % following the last \IEEEmembership and also \thanks - 
% these prevent an unwanted space from occurring between the last author name
% and the end of the author line. i.e., if you had this:
% 
% \author{....lastname \thanks{...} \thanks{...} }
%                     ^------------^------------^----Do not want these spaces!
%
% a space would be appended to the last name and could cause every name on that
% line to be shifted left slightly. This is one of those "LaTeX things". For
% instance, "\textbf{A} \textbf{B}" will typeset as "A B" not "AB". To get
% "AB" then you have to do: "\textbf{A}\textbf{B}"
% \thanks is no different in this regard, so shield the last } of each \thanks
% that ends a line with a % and do not let a space in before the next \thanks.
% Spaces after \IEEEmembership other than the last one are OK (and needed) as
% you are supposed to have spaces between the names. For what it is worth,
% this is a minor point as most people would not even notice if the said evil
% space somehow managed to creep in.



% The paper headers
%\markboth{Journal of \LaTeX\ Class Files,~Vol.~6, No.~1, January~2007}%
%{Shell \MakeLowercase{\textit{et al.}}: Bare Demo of IEEEtran.cls for Journals}
% The only time the second header will appear is for the odd numbered pages
% after the title page when using the twoside option.
% 
% *** Note that you probably will NOT want to include the author's ***
% *** name in the headers of peer review papers.                   ***
% You can use \ifCLASSOPTIONpeerreview for conditional compilation here if
% you desire.




% If you want to put a publisher's ID mark on the page you can do it like
% this:
%\IEEEpubid{0000--0000/00\$00.00~\copyright~2007 IEEE}
% Remember, if you use this you must call \IEEEpubidadjcol in the second
% column for its text to clear the IEEEpubid mark.



% make the title area
\maketitle

%\tableofcontents

\begin{abstract}
This manual is a collection of math problems from the JEE 2018 mains paper. These problems can be solved
using the Python scripts in the JEE 2016 manual.  This will give the student enough practice in Python programming. 
%
\end{abstract}
% IEEEtran.cls defaults to using nonbold math in the Abstract.
% This preserves the distinction between vectors and scalars. However,
% if the journal you are submitting to favors bold math in the abstract,
% then you can use LaTeX's standard command \boldmath at the very start
% of the abstract to achieve this. Many IEEE journals frown on math
% in the abstract anyway.

% Note that keywords are not normally used for peerreview papers.
%\begin{IEEEkeywords}
%Cooperative diversity, decode and forward, piecewise linear
%\end{IEEEkeywords}



% For peer review papers, you can put extra information on the cover
% page as needed:
% \ifCLASSOPTIONpeerreview
% \begin{center} \bfseries EDICS Category: 3-BBND \end{center}
% \fi
%
% For peerreview papers, this IEEEtran command inserts a page break and
% creates the second title. It will be ignored for other modes.
\IEEEpeerreviewmaketitle


%\newpage
%\section{Two Variable}
%
%\subsection{Multivariate Gaussian}
%
%\renewcommand{\thefigure}{Q\theenumi(\theenumii)}
%\renewcommand{\thefigure}{\theenumi}
\begin{enumerate}[1.]
\item If $\alpha, \beta \in \textbf{C} $ are the distinct roots, of the equation $ x^2-x+1=0$, thean $ \alpha^{101} + \beta^{107} $ is equal to: 

\begin{enumerate}[(1)]
 
\item $
-1
$

\item $
0
$

\item $
1
$

\item $
2
$


\end{enumerate}

\item Let $A$ be the sum of the first $20$ terms and $B$ be the sum of the first $40$ terms of the series

$1^2+2.2^2+3^2+2.4^2+5^2+2.6^2+.....$

If $B-2A=100 \lambda$, then $\lambda $ is equal to:

\begin{enumerate}[(1)]
 
\item $
232
$

\item $
248
$

\item $
464
$

\item $
496
$


\end{enumerate}
 
\item If the curves $y^2=6x, \  9x^2+by^2=16 $ intersect each other at right angles, then the value of $b$ is:

\begin{enumerate}[(1)]
 
\item $
6
$

\item $
\frac{7}{2}
$

\item $
4
$

\item $
\frac{9}{2}
$


\end{enumerate}


\item Let $f(x)=x^2+\frac{1}{x^2}$ and $g(x)=x-\frac{1}{x}$,

$ x \in \textbf{R}-\lbrace -1,0,1 \rbrace $. If $h(x)=\frac{f(x)}{g(x)}$, then the local minimum value of $h(x)$ is :

\begin{enumerate}[(1)]
 
\item $
3
$

\item $
-3
$

\item $
-2\sqrt{2}
$

\item $
2\sqrt{2}
$


\end{enumerate}


\item Let $g(x)=cos x^2, \ f(x)=\sqrt{x} $, and $\alpha,\beta (\alpha < \beta)$ be the roots of the quadratic equation $18x^2-9\pi x+{\pi}^2=0$. Then the area (in sq. units) bounded by the curve $y=(g\circ f)(x)$ and the lines $x=\alpha , x=\beta$ and $y=0$, is :

\begin{enumerate}[(1)]
 
\item $
\frac{1}{2}(\sqrt{3}-1)
$

\item $
\frac{1}{2}(\sqrt{3}+1)
$

\item $
\frac{1}{2}(\sqrt{3}-\sqrt{2})
$

\item $
\frac{1}{2}(\sqrt{2}-1)
$


\end{enumerate}

\item A straight line through a fixed point $(2,3)$ intersects the coordinate axes at distinct points $P$ 
and $Q$. If $O$ is the origin and the rectangle $OPRQ$ is completed, then the locus of $R$ is:

\begin{enumerate}[(1)]
 
\item $
3x+2y=6
$

\item $
2x+3y=xy
$

\item $
3x+2y=xy
$

\item $
3x+2y=6xy
$


\end{enumerate}

\item Let the orthocentre and centroid of a triangle be $A(-3,5)$ and $B(3,3)$ respectively. If $C$ is the circumcentre of this triangle, then the radius of the circle having line segment $AC$ as diameter, is :

\begin{enumerate}[(1)]
 
\item $
\sqrt{10}
$

\item $
2 \sqrt{10}
$

\item $
3 \sqrt{\frac{5}{2}}
$

\item $
\frac{3 \sqrt{5}}{2}
$


\end{enumerate}

\item If the tangent at $(1,7)$ to the curve $x^2=y-6$ touches the circle $x^2+y^2+16x+12y+c=0$ then the value of $c$ is :

\begin{enumerate}[(1)]
 
\item $
195
$

\item $
185
$

\item $
85
$

\item $
95
$

\end{enumerate}

\item Tangent and normal are drawn at $P(16,16)$ on the parabola $y^2=16x$, which intersect the axis of the parabola at $A$ and $B$, respectively. If $C$ is the centre of the circle through the points $P$, $A$ and $B$ and $ \angle CPB=\theta$, then a value of $\tan \theta $ is :


\begin{enumerate}[(1)]
 
\item $
\frac{1}{2}
$

\item $
2
$

\item $
3
$

\item $
\frac{4}{3}
$


\end{enumerate}
 
\item Tangents are drawn to the hyperbola $4x^2-y^2=36$ at the points $P$ and $Q$. If these tangents intersect at the point $T(0,3)$ then the area (in sq. units) of $\vartriangle PTQ $ is:

\begin{enumerate}[(1)]
 
\item $
45 \sqrt{5}
$

\item $
54 \sqrt{3}
$

\item $
60 \sqrt{3}
$

\item $
36 \sqrt{5}
$


\end{enumerate}

\item If sum of all the solutions of the equation 

$ 8\cos x\cdot \Big( \cos \Big( \frac{\pi}{6}+x \Big) \cdot \cos \Big( \frac{\pi}{6}-x \Big) - \frac{1}{2} \Big)=1 $

in $[0,\pi]$ is $k\pi$, then $k$ is equal to:

\begin{enumerate}[(1)]
 
\item $
\frac{2}{3}
$

\item $
\frac{13}{9}
$

\item $
\frac{8}{9}
$

\item $
\frac{20}{9}
$


\end{enumerate}

\item The Boolean expression

$\sim (p \vee q) \vee (\sim p \wedge q) $ is equivalent to:

\begin{enumerate}[(1)]
 
\item $
\sim p
$

\item $
p
$

\item $
q
$

\item $
\sim q
$

\end{enumerate}

\item If the tangent at $(1,7)$ to the curve $x^2=y-6$ touches the circle $x^2+y^2+16x+12y+c=0$ then the value of $c$ is :

\begin{enumerate}[(1)]
 
\item $
85
$

\item $
95
$

\item $
195
$

\item $
185
$


\end{enumerate}


\item Let $S= \lbrace x \in \textbf{R} : x \geqslant 0 $ and 

$ 2 \mid \sqrt{x}-3 \mid + \sqrt{x}(\sqrt{x}-6)+6=0 \rbrace $. Then $S$ :

\begin{enumerate}[(1)]

\item contains exactly two elements.

\item contains exactly four elements.

\item is an empty set.

\item contains exactly one element.


\end{enumerate}


\item Two sets $A$ and $B$ are as under:

$A= \lbrace (a,b) \in \textbf{R} \times \textbf{R} \ : \ \mid a-5 \mid \ < \ 1 \ and \mid b-5 \mid < 1 \rbrace ; $

$B= \lbrace (a,b) \in \textbf{R} \times \textbf{R} \ : \ 4(a-6)^2+9(b-5)^2 \leq 36 \rbrace . $ Then :

\begin{enumerate}[(1)]

\item $
A \cap B = \phi $ (an empty set)

\item neither $ A \subset B $ nor $ B \subset A $

\item $ B \subset A $

\item $ A \subset B $

\end{enumerate}

\item Let $ S= \lbrace  t \in \textbf{R} \ : \ f(x)=\mid x - \pi \mid \cdot (e^{\mid x \mid}-1) sin \mid x \mid $ is not differentiable at $t \rbrace $. Then the set $S$ is equal to:

\begin{enumerate}[(1)]

\item $ \lbrace \pi \rbrace $

\item $ \lbrace 0, \pi \rbrace $

\item $ \phi $ (an empty set)

\item $ \lbrace 0 \rbrace $

\end{enumerate}


\item  $\lim\limits_{x \to 0}\frac{x \tan 2x - 2x \tan x}{(1-\cos 2x)^2} $ equals: 


\begin{enumerate}[(1)]
 
\item $
\frac{1}{4}
$

\item $
1
$

\item $
\frac{1}{2}
$

\item $
-\frac{1}{2}
$


\end{enumerate}

\item Let 
$ 
f(x)=\begin{cases}
(x-1)^{\frac{1}{2-x}},& x>1,x \neq2 \\
k,& x=2
\end{cases}
$ \\


The value of $k$ for which $f$ is continuous at $x=2$ is:


\begin{enumerate}[(1)]
 
\item $
1
$

\item $
e
$

\item $
e^{-1}
$

\item $
e^{-2}
$


\end{enumerate}

\item The sides of a rhombus $ABCD$ are parallel to the lines,$x-y+2=0$ and $7x-y+3=0$. If the diagonals of the rhombus intersect at $P(1,2)$ and the vertex $A$(different from the origin) is on the y-axis,then the ordinate of $A$ is :


\begin{enumerate}[(1)]
 
\item $
\frac{5}{2}
$

\item $
\frac{7}{4}
$

\item $
2
$

\item $
\frac{7}{2}
$


\end{enumerate}

\item The tangent to the circle $C_{1} : x^2+y^2-2x-1=0 $ at the point $(2,1) $ cuts off a chord of length $ 4$ from a circle $C_{2}$ whose centre is $ (3,-2). $ The radius of $C_{2}$ is : 



\begin{enumerate}[(1)]
 
\item $
2
$

\item $
\sqrt{2}
$

\item $
3
$

\item $
\sqrt{6}
$


\end{enumerate}


\item Tangents drawn from the point $(-8,0)$ to the parabola $y^2=8x$ touch the parabola at $P$ and $Q$. If $F$ is the focus of the parabola, then the area of the triangle $PFQ$ (in sq.units) is equal to :


\begin{enumerate}[(1)]
 
\item $
24
$

\item $
32
$

\item $
48
$

\item $
64
$


\end{enumerate}
 
\item A normal to the hyperbola, $4x^2-9y^2=36$ meets the co-ordinate axes $x$ and $y$ at $A$ and $B$, respectively. If the parallelogram $OABP$ ($O$ being the origin) is formed, then the locus of $P$ is :


\begin{enumerate}[(1)]
 
\item $
4x^2+9y^2=121
$

\item $
9x^2+4y^2=169
$

\item $
4x^2-9y^2=121
$

\item $
9x^2-4y^2=169
$


\end{enumerate}



\item If the mean of the data : $7,8,9,7,8,7,\lambda,8$ is $8$, then the variance of this data is:


\begin{enumerate}[(1)]
 
\item $
\frac{7}{8}
$

\item $
1
$

\item $
\frac{9}{8}
$

\item $
2
$

\end{enumerate}


\item The number of solutions of $\sin 3x=\cos 2x$, in the interval $ \Big( \frac{\pi}{2},\pi \Big) $ is :


\begin{enumerate}[(1)]
 
\item $
1
$

\item $
2
$

\item $
3
$

\item $
4
$


\end{enumerate}

\item The value of integral $ \int\limits_{\frac{\pi}{4}} ^ {\frac{3 \pi}{4}} \frac{x}{1+\sin x}dx $ is


\begin{enumerate}[(1)]
 
\item $
\pi \sqrt{2}
$

\item $
\pi (\sqrt{2}-1)
$

\item $
\frac{\pi}{2}(\sqrt{2}+1)
$

\item $
2 \pi (\sqrt{2}-1)  
$


\end{enumerate}


\item n-digit numbers are formed using only three digits $2,5$ and $7$.The smallest value of $n$ for which $900$ such distinct numbers can be formed, is :


\begin{enumerate}[(1)]
 
\item $
6
$

\item $
7
$

\item $
8
$

\item $
9
$


\end{enumerate}

\item A circle passes through the points $(2,3)$ and $(4,5)$. If its centre lies on the line, $y-4x+3=0$,then its radius is equal to 


\begin{enumerate}[(1)]
 
\item $
2
$

\item $
\sqrt{5}
$

\item $
\sqrt{2}
$

\item $
1
$


\end{enumerate}

\item Two parabolas with a common vertex and with axes along x-axis and y-axis, respectively, intersect each other in the first quadrant. If the length of the latus rectum of each parabola is $3$, then the equation of the common tangent to the two parabolas is :


\begin{enumerate}[(1)]
 
\item $
4(x+y)+3=0
$

\item $
3(x+y)+4=0
$

\item $
8(2x+y)+3=0
$

\item $
x+2y+3=0
$


\end{enumerate}

\item If the tangents drawn to the hyperbola $4y^2=x^2+1$ intersect co-ordinate axes at the distinct points $A$ and $B$, then the locus of the mid point of $AB$ is :


\begin{enumerate}[(1)]
 
\item $
x^2-4y^2+16x^2y^2=0
$

\item $
x^2-4y^2-16x^2y^2=0
$

\item $
4x^2-y^2+16x^2y^2=0
$

\item $
4x^2-y^2-16x^2y^2=0
$


\end{enumerate}


\item If \ $tanA$ and $tanB$ are the roots of the quadratic equation, $3x^2-10x-25=0$, then the value of $3$ $\sin^2(A+B)-10 \sin(A+B) \cdot \cos(A+B)-25 \cos^2(A+B) $ is :


\begin{enumerate}[(1)]
 
\item $
-10
$

\item $
10
$

\item $
-25
$

\item $
25
$


\end{enumerate}


\item If $ (p \wedge \sim q) \wedge (p \wedge r) \rightarrow \sim p \vee q $ is false, then the truth values of $p,q$ and $r$ are, respectively :


\begin{enumerate}[(1)]
 
\item F,T,F

\item T,F,T

\item T,T,T

\item F,F,F


\end{enumerate}

\item The least positive integer $n$ for which $ {\Big(\frac{1+i\sqrt{3}}{1-i\sqrt{3}} \Big)}^{n}=1 $, is :


\begin{enumerate}[(1)]
 
\item $
2
$

\item $
3
$

\item $
5
$

\item $
6
$


\end{enumerate}


\item Let $A= \left[{\begin{array}{ccc}
1 & 0 & 0\\
1 & 1 & 0\\
1 & 1 & 1\\

\end{array}}\right]$  and $B=A^{20}$. Then the sum of the elements of the first column of $B$ is :


\begin{enumerate}[(1)]
 
\item $
210
$

\item $
211
$

\item $
231
$

\item $
251
$


\end{enumerate}

\item The number of numbers between $2,000$ and $5,000$ that can be formed with the digits $0,1,2,3,4$(repetition of digits is not allowed) and are multiple of $3$ is :



\begin{enumerate}[(1)]
 
\item $
24
$

\item $
30
$

\item $
36
$

\item $
48
$


\end{enumerate}


\item The coefficient of $x^2$ in the expansion of the product

$(2-x^2) \cdot ({(1+2x+3x^2)}^6+{(1-4x^2)}^6) $ is :


\begin{enumerate}[(1)]
 
\item $
107
$

\item $
106
$

\item $
108
$

\item $
155
$


\end{enumerate}

\item The sum of the first $20$ terms of the series

$1+\frac{3}{2}+\frac{7}{4}+\frac{15}{8}+\frac{31}{16}+...,$ is :


\begin{enumerate}[(1)]
 
\item $
38+\frac{1}{2^{19}}
$

\item $
38+\frac{1}{2^{20}}
$

\item $
39+\frac{1}{2^{20}}
$

\item $
39+\frac{1}{2^{19}}
$


\end{enumerate}

\item $\lim\limits_{x \to 0} \frac{(27+x)^{\frac{1}{3}}-3}{9-(27+x)^{\frac{2}{3}}} $ equals :


\begin{enumerate}[(1)]
 
\item $
\frac{1}{3}
$

\item $
-\frac{1}{3}
$

\item $
-\frac{1}{6}
$

\item $
\frac{1}{6}
$

\end{enumerate}


\item Let $M$ and $m$ be respectively the absolute maximum ad the absolute minimum values of the function, $f(x)=2x^3-9x^2+12x+5$ in the interval $[0,3]$. Then $M-m$ is equal to :

\begin{enumerate}[(1)]
 
\item $
5
$

\item $
9
$

\item $
4
$

\item $
1
$


\end{enumerate}

\item If the area of the region bounded by the curves, $y=x^2$, $y=\frac{1}{x}$ and the lines $y=0$ and $x=t(t>1)$ is $1$ sq.unit, then $t$ is equal to :

\begin{enumerate}[(1)]
 
\item $
e^{\frac{3}{2}}
$

\item $
\frac{4}{3}
$

\item $
\frac{3}{2}
$

\item $
e^\frac{2}{3}
$


\end{enumerate}

\item If a circle $C$, whose radius is $3$, touches externally the circle, 

$x^2+y^2+2x-4y-4=0$ at the point $(2,2)$, then the length of the intercept cut by this circle $C$, on the x-axis is equal to :

\begin{enumerate}[(1)]
 
\item $
2 \sqrt{5}
$

\item $
3 \sqrt{2}
$

\item $
\sqrt{5}
$

\item $
2 \sqrt{3}
$


\end{enumerate}

\item Let $P$ be a point on the parabola, $x^2=4y$. If the distance of $P$ from the centre of the circle, $x^2+y^2+6x+8=0$ is minimum, then the equation of the tangent to the parabola at $P$, is :

\begin{enumerate}[(1)]
 
\item $
x+4y-2=0
$

\item $
x-y+3=0
$

\item $
x+y+1=0
$

\item $
x+2y=0
$


\end{enumerate}

\item If an angle $A$ of a $ \bigtriangleup ABC$ satisfies $ 5 \cos A + 3 =0 $, then the roots of the quadratic equation, $9x^2+27x+20=0$ are :

\begin{enumerate}[(1)]
 
\item $
\sec A , \cot A
$

\item $
\sin A , \sec A
$

\item $
\sec A ,\tan A
$

\item $
\tan A , \cos A
$


\end{enumerate}

\item If $ p \rightarrow ( \sim p \vee \sim q )$ is false, then the truth values of $p$ and $q$ are respectively :

\begin{enumerate}[(1)]
 
\item 
F,F


\item 
T,F


\item 
F,T


\item 
T,T



\end{enumerate}
\end{enumerate}


\end{document}


